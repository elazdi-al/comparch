\documentclass[11pt,openany]{book}
\raggedbottom

% Load packages
\usepackage[utf8]{inputenc} % for unicode input
\usepackage{microtype}
\usepackage{bm}
\usepackage{enumitem}
\usepackage{geometry} % for page layout
\usepackage{hyperref} % for hyperlinks
\usepackage{tocbibind} % includes the bibliography in the table of contents
\usepackage{amsmath, amsfonts, amssymb, amsthm} % for advanced math formatting
\usepackage{lipsum} % generates filler text
\usepackage{fancyhdr}
\usepackage[table]{xcolor} % for cell coloring
\usepackage{graphicx} % for including images
\usepackage{booktabs} % for professional looking tables
\usepackage[normalem]{ulem} % for underlining
\usepackage[document]{ragged2e} % for text alignment
\usepackage{tikz} % for drawing
\usepackage{algorithm}
\usepackage{algpseudocode}
\usepackage{wrapfig}
\usepackage{circuitikz}
\usepackage{caption}
\usepackage{venndiagram}
\usepackage{multicol}
\usepackage{listings}
\usepackage{adjustbox}
\usepackage{multirow}
\usepackage{media9}



% TikZ libraries
\usetikzlibrary{circuits.logic.US, arrows.meta, positioning, calc, fit, decorations.markings, math}

% Document geometry (page size, margins)
\geometry{a4paper, left=20mm, right=20mm, top=25mm, bottom=30mm}

% Custom page style for centered page numbers
\pagestyle{fancy}
\fancyhf{} % Clear all header and footer fields
\fancyhead[RE]{\leftmark} % Left Even pages - Chapter number and name
\fancyhead[RO]{Notes by Ali EL AZDI} % Right Odd pages - Custom message
\fancyfoot[CE,CO]{\thepage} % Centered page number in the footer for both even and odd pages
\renewcommand{\headrulewidth}{0pt}
\renewcommand{\footrulewidth}{0pt}

% Custom commands
\newcommand*\xor{\oplus}
\newcommand{\minidash}{\text{-}}

% Custom spacing command
\makeatletter
\newcommand{\vspacer}[1]{%
  \ifvmode
    \vskip#1\relax
  \else
    \@bsphack
    \vadjust{\vskip#1\relax}
    \@esphack
  \fi
}
\makeatother

%%%%%%%%%%%%%%%%%%% Verilog CODE STYLING %%%%%%%%%%%%%%%%%%%%%%%%%%%%%
\definecolor{keywordcolor}{rgb}{0.5,0.0,0.33}
\definecolor{backgroundcolor}{rgb}{0.95,0.95,1.0}
\definecolor{commentcolor}{rgb}{0.129,0.384,0.529}
\definecolor{stringcolor}{rgb}{0.16,0.00,1.00}
\definecolor{rulecolor}{rgb}{0.46,0.43,0.5}
\definecolor{codegray}{rgb}{0.5,0.5,0.5}

\lstdefinestyle{verilogstyle}{
  language=Verilog,
  basicstyle=\ttfamily\footnotesize,
  backgroundcolor=\color{backgroundcolor},
  commentstyle=\color{commentcolor}\ttfamily, % Add \ttfamily to ensure comments are in typewriter font
  morecomment=[l][\color{commentcolor}\ttfamily]{//}, % Line comment in Verilog
  morecomment=[s][\color{commentcolor}\ttfamily]{/*}{*/}, % Block comments in Verilog
  morekeywords={module, input, output, wire, endmodule, endcase, default, tri, assign, always, if, else, begin, end, case, endcase, parameter}, % Add Verilog keywords
  keywordstyle=\color{keywordcolor},
  stringstyle=\color{stringcolor},
  showstringspaces=false,
  frame=single,
  rulecolor=\color{rulecolor}, % Frame color
  breaklines=true,
  numbers=left,
  numberstyle=\tiny\color{codegray},
  tabsize=2
}

\lstnewenvironment{verilog}
  {\lstset{style=verilogstyle}}
  {}

%%%%%%%%%%%%%%%%%%% C CODE STYLING %%%%%%%%%%%%%%%%%%%%%%%%%%%%%
\definecolor{ckeywordcolor}{rgb}{0.8,0.1,0.1}
\definecolor{cbackgroundcolor}{rgb}{0.95,0.95,0.95}
\definecolor{ccommentcolor}{rgb}{0.0,0.5,0.0}
\definecolor{cstringcolor}{rgb}{0.1,0.1,0.8}
\definecolor{crulecolor}{rgb}{0.5,0.5,0.5}
\definecolor{ccodegray}{rgb}{0.6,0.6,0.6}

\lstdefinestyle{cstyle}{
  language=C,
  basicstyle=\ttfamily\footnotesize,
  backgroundcolor=\color{cbackgroundcolor},
  commentstyle=\color{ccommentcolor}\ttfamily,
  keywordstyle=\color{ckeywordcolor},
  stringstyle=\color{cstringcolor},
  showstringspaces=false,
  frame=single,
  rulecolor=\color{crulecolor},
  breaklines=true,
  numbers=left,
  numberstyle=\tiny\color{ccodegray},
  tabsize=2
}

\lstnewenvironment{cc}
  {\lstset{style=cstyle}}
  {}

%%%%%%%%%%%%%%%%%%% Assembly CODE STYLING %%%%%%%%%%%%%%%%%%%%%%%%%%%%%
\definecolor{akeywordcolor}{rgb}{0.0, 0.2, 0.4}
\definecolor{abackgroundcolor}{rgb}{0.98, 0.99, 1.0}
\definecolor{acommentcolor}{rgb}{0.0, 0.4, 0.6}
\definecolor{astringcolor}{rgb}{0.2, 0.4, 0.8}
\definecolor{arulecolor}{rgb}{0.6, 0.7, 0.8}
\definecolor{acodegray}{rgb}{0.3, 0.4, 0.5}

\lstdefinestyle{assembly}{
  language=[x86masm]Assembler,
  basicstyle=\ttfamily\footnotesize,
  backgroundcolor=\color{abackgroundcolor},
  commentstyle=\color{acommentcolor}\ttfamily,
  keywordstyle=\color{akeywordcolor},
  stringstyle=\color{astringcolor},
  showstringspaces=false,
  frame=single,
  rulecolor=\color{arulecolor},
  breaklines=true,
  numbers=left,
  numberstyle=\tiny\color{acodegray},
  tabsize=2,
  morekeywords={li, and, add, addi, srli, bne}
}

\lstnewenvironment{assembly}
  {\lstset{style=assembly}}
  {}

%%%%%%%%%%%%%%%%%%%%%%%%%%%%%%%%%%%%%%%%%%%%%%%%%%%%%%%%
%%%%%%%%%%%%%%%%%%% JAVA CODE STYLING %%%%%%%%%%%%%%%%%%%%%%%%%%%%%
\definecolor{javakeywordcolor}{rgb}{0.0, 0.0, 0.5}
\definecolor{javabackgroundcolor}{rgb}{0.95, 0.95, 0.95}
\definecolor{javacommentcolor}{rgb}{0.0, 0.5, 0.0}
\definecolor{javastringcolor}{rgb}{0.6, 0.0, 0.0}
\definecolor{javarulecolor}{rgb}{0.5, 0.5, 0.5}
\definecolor{javagray}{rgb}{0.6, 0.6, 0.6}

\lstdefinestyle{javastyle}{
  language=Java,
  basicstyle=\ttfamily\footnotesize,
  backgroundcolor=\color{javabackgroundcolor},
  commentstyle=\color{javacommentcolor}\ttfamily,
  keywordstyle=\color{javakeywordcolor},
  stringstyle=\color{javastringcolor},
  showstringspaces=false,
  frame=single,
  rulecolor=\color{javarulecolor},
  breaklines=true,
  numbers=left,
  numberstyle=\tiny\color{javagray},
  tabsize=2,
  morekeywords={class, public, private, protected, extends, implements, interface, import, package, new, return, void, static}
}

\lstnewenvironment{java}
  {\lstset{style=javastyle}}
  {}

%%%%%%%%%%%%%%%%%%%%%%%%%%%%%%%%%%%%%%%%%%%%%%%%%%%%%%%%
% Document begins
\begin{document}


% Title Page
\begin{titlepage}
    \centering
    \vspace*{1cm}
    \Huge
    Computer Architecture
    \vspace{10px}
    \newline
    \Large IN BA3 - Paolo IENNE
    \vfill
    \large
    September 11, 2024
\end{titlepage}

\begin{center}
    \vspace*{1cm}
    \textbf{Introduction}
    \newline
    \paragraph[short]{}{This document is designed to offer a LaTeX-styled overview of the Computer Architecture course, emphasizing brevity and clarity. Should there be any inaccuracies or areas for improvement, please reach out at ali.elazdi@epfl.ch for corrections. For the latest version, check my GitHub repository.}
\end{center}

% Table of Contents
\tableofcontents

 
\chapter{Part I(a) - ISA Reminder, Assembly Language, Compiler - W 1.1}
\textbf{hum...welcome back} \newline
\textit{In the first part of the course, professor introduced (for motivational purposes) how computer architecture, specifically processors, have become essential to our lives, and how the field is growing exponentially. (didn't think it was essential to mention here...)}

\section{From High Level Languages to Assembly Language}
\subsection{High Level Languages}
\textit{When talking about programming we usually think of programs that look like this\dots} \newline \vspace*{5px}

\begin{minipage}[htp]{0.4\textwidth} % Use \textwidth to ensure it spans the page width
\begin{cc}
int data = 0x00123456;
int result = 0;
int mask = 1;
int count = 0;
int temp = 0;
int limit = 32;
do {
    temp = data & mask;
    result = result + temp;
    data = data >> 1;
    count = count + 1;
} while (count != limit);
\end{cc}
\end{minipage}
\hfill
\vline
\hfill
\begin{minipage}[htp]{0.4\textwidth}
    \centering
    \begin{tabular}{|c|c|}
        \hline
        \textbf{name} & \textbf{value} \\ \hline
        data       & 0x00123456  \\ \hline
        result     & 0           \\ \hline
        mask       & 1           \\ \hline
        count      & \dots       \\ \hline
        temp       &             \\ \hline
        limit      &             \\ \hline
        \dots      &             \\ \hline
        my\_float  & 3.141529    \\ \hline
        a\_string  & Hello world! \\ \hline
        \end{tabular}
\end{minipage}

\subsection{Assembly Language}
We use this code because it enables us to build a \textit{Finite State Machine}, which isn't feasible with C code. This language provides a more rigid format with a sequence of numbered instructions, an \textit{opcode}, predefined variable names, and the ability to \textbf{jump between lines}.
\newpage
\begin{center}
    \begin{assembly}
    li x1, 0x00123456
    li x2, 0
    li x3, 1
    li x4, 0
    li x5, 0
    li x6, 32
loop: and x5, x1, x3
    add x2, x2, x5
    srli x1, x1, 1
    addi x4, x4, 1
    bne x4, x6, loop
    \end{assembly}
\end{center}

\section{Processors}
\textbf{Remember, a processor can be decomposed into five components:} \newline
\begin{itemize}[noitemsep]
    \item[-] \textbf{ALU (Arithmetic and Logic Unit)}: Performs arithmetic and logical operations.
    \item[-] \textbf{Register File}: Stores data temporarily for quick access during processing.
    \item[-] \textbf{Memory}: Holds data and instructions needed by the processor.
    \item[-] \textbf{Control Logic}: Directs the operation of the processor by coordinating the other components.
    \item[-] \textbf{PC (Program Counter)}: Keeps track of the address of the next instruction to be executed.
    \item[-] \textbf{Instruction Memory}: Stores the program instructions that the processor will execute.
\end{itemize}
\begin{center}
    \includegraphics[width=0.5\textwidth]{chapters/chapter1a/images/processor.png}
\end{center}

We may distinguish three types of general operations made by the processor: \newline
\subsubsection*{Encoding}
\begin{center}
    \includegraphics[width=0.65\textwidth]{chapters/chapter1a/images/encoding.png}
\end{center}
\subsubsection*{Fetching}
\begin{center}
    \includegraphics[width=0.65\textwidth]{chapters/chapter1a/images/fetching.png}
\end{center}
\subsubsection*{Executing}
\begin{center}
    \includegraphics[width=0.65\textwidth]{chapters/chapter1a/images/executing.png}
\end{center}


\section{Joint or Disjoint Program and Data Memories}
\textit{There are two main types of architectures one called the Harvard Architecture (Where the data and the memory are seperate) and pne called Unified Architecture (where data is shared with the program memory)} \newline
\vspace*{10px}
\begin{minipage}[htp]{0.4\textwidth}
    \texttt{Harvard Architecture} \newline
    \vspace*{2px}
    \centering
    \includegraphics[width=0.6\textwidth]{chapters/chapter1a/images/harvard.png}
\end{minipage}
\hfill
\vline
\hfill
\begin{minipage}[htp]{0.4\textwidth}
    \texttt{Unified Architecture} \newline    
    \vspace*{2px}

    \centering
    \includegraphics[width=0.6\textwidth]{chapters/chapter1a/images/unified.png}
\end{minipage}
\newpage
\section{The Encoding problem}
\textit{We may ask ourselves how we encode assembly written instructions into actual 0s and 1s.} \newline
\subsection{The Stupid Solution}
\textit{Now, the professor throws out the "stupid idea"(his words) of just counting all possible instructions, assigning a number to each one, and writing the numbers in binary. The problem with such a method is that the number of instructions could grow exponentially, requiring an unmanageable number of bits to represent each one, leading to inefficiency.} \newline 
\begin{center}
    \includegraphics[width=0.7\textwidth]{chapters/chapter1a/images/encoding.png}
    \centering
    \textbf{"stupid solution"}
\end{center}

\subsection{RISC-V Encoding (The Solution)}
\textbf{Instead, the chosen solution is to use an instruction set encoding where instructions are grouped into classes, each with a fixed format optimizing both memory usage and processing speed by limiting the number of bits required to represent instructions.} \newline
\begin{center}
    \includegraphics[width=0.7\textwidth]{chapters/chapter1a/images/riscv.png}
    \centering
    \textbf{RISC-V encoding}
\end{center}

\newpage
\subsection{Automating this process}
\textit{Now to automate the processes of decoding assembler code into machine code we use an \textbf{Assembler}, and to automate the process of decoding a higher level language to assembler we use a \textbf{Compiler}}. \newline
\subsubsection{Assembler}
\textit{The program that does this is called an assembler. It takes the assembly code and converts it into machine code.} \newline
\begin{center}
    \includegraphics[width=0.7\textwidth]{chapters/chapter1a/images/assembler.png}
    \centering
    \textbf{Assembly}
\end{center}
\subsubsection{Compiler}
A compiler is a program that translates high-level source code written in languages like C or Java into machine code or an intermediate representation. 
\begin{center}
    \includegraphics[width=0.7\textwidth]{chapters/chapter1a/images/compiler.png}
    \centering
    \textbf{Compilation}
\end{center}

\newpage
\section{ISA (Instruction Set Architecture)}
\textit{The ISA is the interface between the hardware and the software. It defines the instructions that a processor can execute, as well as the format of those instructions.} \newline
\begin{center}
    \includegraphics[width=0.65\textwidth]{chapters/chapter1a/images/contract.png}
    \centering
\end{center} 
\chapter{Part I(b) - ISA, Functions, and Stack - W 1.2}
\section{Arithmetic and Logic Instructions in RISCV}
\textit{Bellow some examples of RISCV instructions:} \\
\textbf{Two Operands Instructions} \\
\vspace*{10px}
\begin{minipage}{0.4\textwidth}
\begin{assembly}
sll  x5, x5, x9
add  x6, x5, x7
xor  x6, x6, x8
slt  x8, x6, x7
\end{assembly}
\end{minipage}%
\hfill
\vline
\hfill
\begin{minipage}{0.5\textwidth}
\small
\textit{Shift x5 left by x9 positions $\rightarrow$ x5} \\
\textit{Add x5 and x7 $\rightarrow$ x6} \\
\textit{Logic XOR bitwise x6 and x8 $\rightarrow$ x6} \\
\textit{Set x8 to 1 if x6 is lower than x7, otherwise to 0}
\end{minipage}

\textbf{Arithmetic Instructions} \\
\vspace*{10px}
\begin{minipage}{0.4\textwidth}
\begin{assembly}
slli x5, x5, 3
addi x6, x5, 72
xori x6, x6, -1
slti x8, x6, 321
\end{assembly}
\end{minipage}%
\hfill
\vline
\hfill
\begin{minipage}{0.5\textwidth}
\small
\textit{Shift x5 left of 3 positions $\rightarrow$ x5} \\
\textit{Add 72 to x5 $\rightarrow$ x6} \\
\textit{Logic XOR bitwise x6 and 0xFFFFFFFF $\rightarrow$ x6} \\
\textit{Set x8 to 1 if x6 is lower than 321, to 0 otherwise} \\
\end{minipage} \\
\textbf{Here, you may ask yourself, why are all immediates (constants) writtent on a maximum of 12bits?} \\
\subsection{Constants must be encoded on 12 bits}
\textit{As you may see here, all instructions encode immediates on 12 bits.}
\begin{center}
    \includegraphics[width=0.8\textwidth]{chapters/chapter1b/images/riscv.png}
\end{center}

\subsection{Assembler Directives} \textit{Assembler directives help write cleaner and more readable code. The code snippets on the left and right below are equivalent.}

\begin{center} \includegraphics[width=0.65\textwidth]{chapters/chapter1b/images/directives.png} \end{center}

The left-hand side code snippet shows an assembly sequence where a 32-bit constant value (\texttt{0x12345678}) is loaded into a register (\texttt{x5}). Since immediate values are 16-bit limited, this requires splitting the 32-bit value into two instructions: 

\begin{itemize}
    \item[-] The first instruction, \texttt{lui}, loads the upper 20 bits (\texttt{0x12345}) into the register \texttt{x5}.
    \item[-] The second instruction, \texttt{addiu}, adds the lower 12 bits (\texttt{0x678}) to \texttt{x5}, completing the full 32-bit value in the register.
\end{itemize}

\textit{This approach, while functional, can become cumbersome when dealing with multiple constants, making the code less readable and harder to maintain. \\
} 
\vspace*{5px}
The right-hand side shows the same functionality but makes use of assembler directives, specifically the \texttt{.equ} directive to define a label (\texttt{something}) for the constant \texttt{0x12345678}. Using the \texttt{\%hi()} and \texttt{\%lo()} pseudo-instructions, the assembler automatically splits the constant into its upper and lower parts:

\begin{itemize}
    \item[-] The \texttt{\%hi(something)} loads the upper 20 bits into \texttt{x5}.
    \item[-] The \texttt{\%lo(something)} adds the lower 12 bits to \texttt{x5}.
\end{itemize}

This method enhances code clarity and maintainability, especially when working with multiple constants, by using human-readable labels instead of raw numeric values. The assembler handles the details of splitting the 32-bit constant into its upper and lower parts.
\begin{center} \includegraphics[width=0.65\textwidth]{chapters/chapter1b/images/directives2.png} \end{center}

\subsection{The \texttt{x0} Register} 
\textit{The \texttt{x0} register is hardwired to 0 and cannot be changed.} \ \textit{Any attempt to write into \texttt{x0} will have no effect.}

\texttt{Why is this useful?} \\
One common application is in introducing wait delays during program execution. By leveraging the fixed nature of \texttt{x0}, it simplifies certain instructions that require an immediate zero value.

\section{PseudoInstructions}
\textit{PseudoInstructions simplify commands involving the \texttt{x0} register by creating easier-to-use alternatives.} \newline
\begin{center}
        \begin{tabular}{|c|c|c|}
        \hline
        \textbf{Pseudoinstruction} & \textbf{Base Instruction(s)} & \textbf{Meaning} \\ \hline
        \texttt{nop}               & \texttt{addi x0, x0, 0}      & No operation     \\ \hline
        \texttt{li rd, immediate}  & Myriad sequences             & Load immediate   \\ \hline
        \texttt{mv rd, rs}         & Myriad sequences             & Copy register    \\ \hline
        \texttt{not rd, rs}        & \texttt{xori rd, rs, -1}     & One's complement \\ \hline
        \texttt{neg rd, rs}        & \texttt{sub rd, x0, rs}      & Two's complement \\ \hline
        \texttt{seqz rd, rs}       & \texttt{sltiu rd, rs, 1}     & Set if = zero    \\ \hline
        \texttt{snez rd, rs}       & \texttt{sltu rd, x0, rs}     & Set if $\neq$ zero    \\ \hline
        \texttt{sltz rd, rs}       & \texttt{slt rd, rs, x0}      & Set if < zero    \\ \hline
        \texttt{sgtz rd, rs}       & \texttt{slt rd, x0, rs}      & Set if > zero    \\ \hline
        \end{tabular}
\end{center} 
The term \textit{myriad sequences} refers to a series of instructions that together achieve the functionality of a single pseudoinstruction, such as using \texttt{lui} and \texttt{addi} to implement \texttt{li rd, immediate}.

\textbf{According to the professor li should be called \texttt{mvi} (as move immediate).}

\subsection{Control flow instructions}
\textit{Control flow instructions are used to change the order of execution of instructions are a kind of pseudo-instructions.}
\begin{assembly}
    li x1, 0x00123456
    li x2, 0
    li x3, 1
    li x4, 0
    li x5, 0
    li x6, 32
loop: and x5, x1, x3
    add x2, x2, x5
    srli x1, x1, 1
    addi x4, x4, 1
    bne x4, x6, loop
\end{assembly}

\subsection{If-Then-Else}
\begin{minipage}[htp]{0.4\textwidth}
\begin{cc}
if (x5 == 72) {
    x6 = x6 + 1;
    } else {
    x6 = x6 - 1;
}
\end{cc}    
\end{minipage}
\hfill
\vline
\hfill
\begin{minipage}[htp]{0.4\textwidth}
\begin{assembly}
.text
    li x7, 72
    beq x5, x7, then_clause
else_clause:
    addi x6, x6, -1
    j end_if
then_clause:
    addi x6, x6, 1
end_if:
\end{assembly}
\end{minipage} \\
\vspace*{5px}
\textit{As seen here, beqi does not exist in RISCV, instead we use \texttt{beq} and \texttt{li} to achieve the same result.}
\subsection{Jumps and Branches}
A common but not universal distinction exists between \emph{jumps} and \emph{branches}. In RISC-V (inherited from MIPS and used by SPARC, Alpha, etc.), jumps refer to unconditional control transfer instructions, while branches refer to conditional control transfer instructions. However, not all architectures follow this convention. For instance, in x86, all control transfer instructions are considered jumps, such as \texttt{JMP}, \texttt{JZ}, \texttt{JC}, and \texttt{JNO}.

\subsection{Comparaisions}
\textit{The processor implements only $<$ and $>$, and the assembler “creates” $\leq$ and $\geq$.}

\begin{center}
    \includegraphics[width=0.75\textwidth]{chapters/chapter1b/images/comp.png}
\end{center}
\subsection{Do-While}
\textit{Do-while loops look like this (we obviously use control flow instructions here).} \\
\begin{minipage}[htp]{0.4\textwidth}
\begin{cc}
do {
    x5 = x5 >> 1;
    x6 = x6 + 1;
} while (x5 != 0);
\end{cc}    
\end{minipage}
\hfill
\vline
\hfill
\begin{minipage}[htp]{0.4\textwidth}
\begin{assembly}
.text
loop:
    srli x5, x5, 1
    addi x6, x6, 1
    bnez x5, loop
\end{assembly}
\end{minipage}

\section{Functions}
\textit{In higher-level programming languages, functions (routines, subroutines, procedures, methods, etc.) are used to encapsulate code and make it reusable. } \\
\textbf{Calling a function involves these steps:}
\begin{enumerate}
    \item Place arguments where the called function can access them.
    \item Jump to the function.
    \item Acquire storage resources the function needs.
    \item Perform the desired task of the function.
    \item Communicate the result value back to the calling program.
    \item Release any local storage resources.
    \item Return control to the calling program.
\end{enumerate}
\subsection{Jump to the Function/Retun control to the calling program}
\subsubsection{The too simple not working approach}
A simple (not working) approach for creating functions would be to do this: 
\begin{center}
    \includegraphics[width=0.75\textwidth]{chapters/chapter1b/images/function.png}
\end{center}
\textit{With this approach the function doesn't know where to return to after being called (back2 or back)}
\textbf{For the next part, remember, the Program Counter is distinct from general-purpose registers. It is dedicated to managing the flow of instruction execution, while general registers are used for data manipulation. }
\subsubsection{The Good Approach}
\textit{The right approach involves using the Jump and Link instruction \texttt{jal}, here loading PC + 4 (remember 4 bytes per Instruction) into x1 as a way to come back from the function.} \\
\begin{minipage}[htp]{0.4\textwidth}
\begin{assembly}
main:
    ...
    jal x1, sqrt
    ...
    ...
    jal x1, sqrt
\end{assembly}    
\end{minipage}
\hfill
\vline
\hfill
\begin{minipage}[htp]{0.4\textwidth}
\begin{assembly}
sqrt:
    ...
    ...
    jr x1
\end{assembly}
\end{minipage} \\
\textit{Both times x1 was used to store the return adress, and there is a reason for that (Register Conventions Sections).}

\subsection{Jump Instructions}
\textit{There are only two core real jump instructions in RISCV, \texttt{jal} (jump and link) and \texttt{jalr} (jump and link register), the rest are pseudo instructions using them.} \\

\begin{center}
    \includegraphics[width=0.75\textwidth]{chapters/chapter1b/images/jump.png}
\end{center}
\newpage
\subsection{Register Conventions}
\textit{Register conventions are rules that dictate how registers are used in a program, here are the ones we've seen for now} \\
\begin{center}
    \includegraphics[width=0.75\textwidth]{chapters/chapter1b/images/conventions.png}
\end{center}

\subsection{Back to the good (not so good) approach}
\textit{There's still a problem with the previous approach, say for example you want to call a function from another function.}
\begin{center}
    \includegraphics[width=0.75\textwidth]{chapters/chapter1b/images/function2.png}
\end{center}
\textbf{Here the allocated space for the return address is overwritten by the second function call, and the first function can't return to the right place.}
\subsection{One simple solution (still not good)}
\textit{One solution would be to say that a range of registers are used for certain functions and that they can't be used by other functions.}
\begin{center}
    \includegraphics[width=0.75\textwidth]{chapters/chapter1b/images/function3.png}
\end{center}
\textbf{The problem here is that it's still not very scalable.}
\subsection{Acquire storage resources the function needs (still not it)}
One simple solution to our problem would be to allocate memory for the function at in the data section of the program. \\
\begin{minipage}[htp]{0.4\textwidth}
\begin{assembly}
.data
sqrt_save_ra: .word 0
sqrt_save_x5: .word 0 
\end{assembly}
\end{minipage}
\hfill
\vline
\hfill
\begin{minipage}[htp]{0.4\textwidth}
\begin{assembly}
.text
sqrt:
...
add x5, x7, x8
sw ra, sqrt_save_ra
sw x5, sqrt_save_x5
jal round
lw ra, sqrt_save_ra
lw x5, sqrt_save_x5
sub x6, x6, x5
...
ret
\end{assembly}
\end{minipage}
\subsubsection{Problem: Recursive Functions}
\textit{The problem here is that the return address is overwritten by the recursive call.}
\begin{center}
\begin{assembly}
.data
    find_child_save_ra: .word 0
.text
    find_child:
    ...
    sw ra, find_child_save_ra
    jal find_child
    lw ra, find_child_save_ra
    ...
    ret
\end{assembly}
\end{center}
\subsection{The Stack}
\textit{The Solution to our Problem is this, the Stack.} \\
\textbf{The Stack is a region of memory that grows and shrinks as needed.} \\
We may use a register (e.g \texttt{x2}) to point to the first used word after the end of the used region.
\begin{center}
    \includegraphics[width=0.4\textwidth]{chapters/chapter1b/images/stack.png}
\end{center}

\subsubsection{Dynamic Memory Allocation}
The Stack, contrary to the Data Section, is dynamic and can be used to allocate memory when needed. This means that during program execution, variables or temporary data can be stored in the stack, which grows or shrinks depending on the operations performed. \\
 The \texttt{stack pointer}, typically register x2, is used to manage the allocation and deallocation of memory.

\begin{minipage}[htp]{0.4\textwidth}
\textit{In this instruction, for example, we allocate 12 bytes in the stack. We achieve this by decrementing the stack pointer (x2) by 12. This ensures that the new memory space is available for temporary storage.}
\begin{assembly}
addi x2, x2, -12
\end{assembly}
\end{minipage}
\hfill
\vline
\hfill
\begin{minipage}[htp]{0.4\textwidth}
\begin{center}
\includegraphics[width=0.75\textwidth]{chapters/chapter1b/images/stack2.png}
\end{center}
\end{minipage}

\subsubsection{Retrieving Data from the Stack}
Once memory has been allocated on the stack, we can store or retrieve data from it. In this case, we are retrieving data that was previously saved in the stack. The lw (load word) instruction is used to load the values stored at different offsets in the stack.

\begin{minipage}[htp]{0.4\textwidth}
\textit{In this case, we retrieve three different values from the stack using the lw instruction, which loads a 4-byte value into the specified registers (ra, x5, and x6). The offsets (0, 4, and 8) refer to different positions in the 12 bytes we allocated earlier.}
\begin{assembly}
lw ra, 0(x2)
lw x5, 4(x2)
lw x6, 8(x2)
\end{assembly}
\end{minipage}
\hfill
\vline
\hfill
\begin{minipage}[htp]{0.4\textwidth}
\begin{center}
\includegraphics[width=0.75\textwidth]{chapters/chapter1b/images/stack3.png}
\end{center}
\end{minipage}
\newpage

\subsubsection{Memory Deallocation}
After the data has been used or is no longer needed, it is good practice to deallocate the memory to ensure proper management of the stack. We deallocate memory by adjusting the stack pointer (x2) back to its original position.

\begin{minipage}[htp]{0.4\textwidth}
\textit{In this instruction, we restore the stack to its previous state by adding 12 back to the stack pointer (x2).} \\ \textit{This effectively "frees" the 12 bytes of memory we had allocated earlier.}
\begin{assembly}
addi x2, x2, 12
\end{assembly}
\end{minipage}
\hfill
\vline
\hfill
\begin{minipage}[htp]{0.4\textwidth}
\begin{center}
\includegraphics[width=0.75\textwidth]{chapters/chapter1b/images/stack4.png}
\end{center}
\end{minipage}

\subsubsection{The Stack Pointer}
\textit{The Stack Pointer is a register that points to the top of the stack, by convention it corresponds to the x2 register} \\
\begin{center}
\includegraphics[width=0.75\textwidth]{chapters/chapter1b/images/conventions2.png}
\end{center}
\small
\textit{Other architectures have special instructions to place stuff on
the stack (push) and to retrieve it (pop)} \\
\vspace*{10px}
\begin{minipage}[htp]{0.4\textwidth}
\begin{lstlisting}
PUSH AX
\end{lstlisting}
\end{minipage}
\hfill
\vline
\hfill
\begin{minipage}[htp]{0.4\textwidth}
\begin{assembly}
add sp, sp, -4
sw x5, 0(sp)
\end{assembly}
\end{minipage}

\subsection{Spilling Registers to Memory}
\textit{Spilling registers to memory involves saving register values to the stack when more registers are needed or to prevent overwriting important data, allowing the registers to be reused. This technique is also used in function calls to save the return address, ensuring the program can correctly return control after the function finishes.}
\begin{center}
    \includegraphics[width=0.75\textwidth]{chapters/chapter1b/images/spilling.png}
\end{center}

\subsection{Register across functions}
In assembly programming, handling registers across functions can be managed in two main ways: either functions \textbf{change registers} and expect the caller to save their values, or functions \textbf{preserve registers} and ensure that the register values remain the same across function calls.

\begin{itemize}
    \item On the left, the function \texttt{sqrt} changes the value of register \texttt{x20}, requiring the caller to save and restore its value.
    \item On the right, the function \texttt{sqrt} preserves the value of \texttt{x20}, ensuring that the caller does not need to manage the saving and restoring.
\end{itemize}

This distinction is important, but it does not cause issues as long as there is agreement on how registers are handled. \\
\textit{In case it's still not clear, we're looking at the \texttt{sw} instruction}

\begin{center}
    \includegraphics[width=0.7\textwidth]{chapters/chapter1b/images/registers.png}
\end{center}

\subsection{Preserving Registers}
In RISC-V, register preservation is managed through a combination of callee-saved and caller-saved registers. \\
Callee-saved registers (such as \texttt{s0}, \texttt{s1}, and \texttt{s2-11}) are preserved by the called function, ensuring that their values remain unchanged after the function call.  \\
Caller-saved registers (such as \texttt{t0}, \texttt{t1-2}, and \texttt{t3-6}) are temporary and do not need to be preserved by the called function, meaning the caller must save them if their values are important. \\
\begin{center}
    \begin{tabular}{|c|c|c|c|}
        \hline
        \textbf{Register} & \textbf{ABI Name} & \textbf{Description} & \textbf{Preserved across call?} \\ \hline
        x0  & zero  & Hard-wired zero                        & \textemdash    \\ \hline
        x1  & ra    & Return address                         & No             \\ \hline
        x2  & sp    & Stack pointer                          & Yes            \\ \hline
        x5  & t0    & Temporary/alternate link register      & No             \\ \hline
        x6--7 & t1--2 & Temporaries                          & No             \\ \hline
        x8  & s0/fp & Saved register/frame pointer           & Yes            \\ \hline
        x9  & s1    & Saved register                        & Yes            \\ \hline
        x18--27 & s2--11 & Saved registers                   & Yes            \\ \hline
        x28--31 & t3--6 & Temporaries                        & No             \\ \hline
        \end{tabular}
\end{center}

\section{Passing Arguments in RISC-V}

In RISC-V, there are two main ways to pass arguments to functions:

\subsection{Option 1: Using Registers}
- Specific registers are used to pass arguments and return results. \\
\vskip 0.1in
- This can be done in a straightforward way, where each function uses different registers (e.g., passing an argument in \texttt{x5} and returning the result in \texttt{x6}).
\vskip 0.1in
- A more structured approach is to follow a convention where arguments are passed in registers \texttt{x10} to \texttt{x17}, with results returned in \texttt{x10}.  \\
\vskip 0.1in
- The limitation: if there are more arguments than available registers (e.g., more than 8 arguments), this approach is insufficient.  \\

\subsection{Option 2: Using the Stack}
- When registers are not enough, extra arguments are placed on the stack.  \\
\vskip 0.1in
- The stack offers a universal solution because it has no practical limit on size.  \\
\vskip 0.1in
- However, using the stack is more complex and requires additional work compared to using registers.  \\

\subsection{The RISC-V Approach}
- RISC-V uses a combination of both methods.  \\
\vskip 0.1in
- Registers \texttt{x10} to \texttt{x17} are used to pass arguments, with \texttt{x10} and \texttt{x11} also handling return values. \\
\vskip 0.1in
- If more arguments are needed beyond what these registers can handle, they are passed via the stack. 

\begin{center}
    \includegraphics[width=0.75\textwidth]{chapters/chapter1b/images/arguments.png}
\end{center}
\textit{Register reserved for arguments and return values in RISC-V.}

\section{Summary of RISC-V Register Conventions}
\begin{center}
    \includegraphics[width=0.75\textwidth]{chapters/chapter1b/images/summary.png}
\end{center} 
\chapter{Part I(c) - ISA Memory and Addressing Modes - W 2.1}

\section{Memory}
\textit{Memory is a really important component of a computing system, we store our programs in it, we store our data in it, and it's through memory that we receive and send data.} \\ \vspace*{5px}
\textbf{Though memory is very useful it has three main drawbacks:} \\ \vspace*{5px}
\begin{itemize}
    \item[-] It's \textbf{slow} $\rightarrow$ Caches
    \item[-] It's \textbf{finite} $\rightarrow$ Virtual Memory
    \item[-] It can make an ISA \textbf{too complex} $\rightarrow$ Pipelining
\end{itemize}
\textit{no worries we'll cover each one of these in this chapter.}

\subsection{Address and Data}
\textit{Data in Memory can be accessed by an adress, meaning i's a \textit{Random Access} (it can access a memory value without going through the preceding ones).} \\ \vspace*{5px}
\textit{Professor Remark: "There's not anything random about this memory, we'd better call it and abitrary access memory."} \\ \vspace*{5px}
\vspace*{5px}
\begin{minipage}[htp]{0.45\textwidth}
\begin{center}
    \begin{tabular}{|c|c|}
    \hline
    \textbf{Address} & \textbf{Value} \\
    \hline
    \texttt{0} & 12 \\
    \hline
    \texttt{1} & 6 \\
    \hline
    \texttt{2} & 4 \\
    \hline
    \texttt{3} & 1 \\
    \hline
    \texttt{4} & 0 \\
    \hline
    \texttt{5} & 3 \\
    \hline
    \texttt{6} & 1 \\
    \hline
    \texttt{7} & 13 \\
    \hline
    \texttt{8} & 15 \\
    \hline
    \texttt{9} & 9 \\
    \hline
    \texttt{10} & 3 \\
    \hline
    \texttt{11} & 5 \\
    \hline
    \texttt{12} & 0 \\
    \hline
    \texttt{13} & 0 \\
    \hline
    \texttt{14} & 0 \\
    \hline
    \texttt{15} & 0 \\
    \hline
    \end{tabular}
\end{center}
\end{minipage}
\hfill
\vline
\hfill
\begin{minipage}[htp]{0.45\textwidth}
\begin{center}
    \begin{tabular}{|c|c|}
    \hline
    \textbf{Write} & \textbf{Read} \\
    \hline
    \texttt{\textbf{Memory[5] = 3}} & \texttt{Memory[5]?} \\
    \hline
    \end{tabular}
\end{center}
\end{minipage}

\section{Many Types of Memories}
We may distinguish between different types of memories based on their \textbf{technology}, such as SRAM, DRAM, EPROM, and Flash, and their \textbf{capabilities}, including \textbf{speed}, \textbf{capacity}, \textbf{density}, \textbf{writability} (whether they are writable, permanent, or reprogrammable), as well as their \textbf{size}, \textbf{volatility}, and \textbf{cost}. \\ \vspace*{5px}
\subsection{Functional Taxonomy of Memories}
\begin{center} \includegraphics[width=0.45\textwidth]{chapters/chapter1c/images/funct_tax.png} \end{center}
\begin{itemize}
    \item[] \textbf{Multiport} memory allows simultaneous access by multiple processors, while \textbf{single-port} memory supports only one at a time.
    \item[] \textbf{Non-Random Access memories}
    \begin{itemize}
        \item \textbf{Adsociative} memories enable fast data retrieval by content rather than address, making it useful for cache memory, pattern recognition, and efficient lookups in large datasets.
        \item In \textbf{Implicit addressing} the address of the data to be operated on is inferred directly by the operation code (opcode), without explicitly specifying the address in the instruction.
    \end{itemize}
\end{itemize}


\subsection{Taxonomy of Random Access Memories}
\begin{center}
\includegraphics[width=0.45\textwidth]{chapters/chapter1c/images/ram_tax.png}
\end{center}
\newpage
\subsection{Basic Structure}
\textit{Remember, a Data Flip Flop, stores a 1 bit value by updating the output value to the input value at the rising edge of the clock signal.} \\ \vspace*{5px}
\begin{minipage}[htp]{0.35\textwidth}
\begin{center}
    \begin{tabular}{|c|c|}
        \hline
        \textbf{Address} & \textbf{Value} \\
        \hline
        \texttt{0} & 12 \\
        \hline
        \texttt{1} & 6 \\
        \hline
        \texttt{2} & 4 \\
        \hline
        \texttt{3} & 1 \\
        \hline
        \texttt{4} & 0 \\
        \hline
        \texttt{5} & 3 \\
        \hline
        \texttt{6} & 1 \\
        \hline
        \texttt{7} & 13 \\
        \hline
        \texttt{8} & 15 \\
        \hline
        \texttt{9} & 9 \\
        \hline
        \texttt{10} & 3 \\
        \hline
        \texttt{11} & 5 \\
        \hline
        \texttt{12} & 0 \\
        \hline
        \texttt{13} & 0 \\
        \hline
        \texttt{14} & 0 \\
        \hline
        \texttt{15} & 0 \\
        \hline
        \end{tabular}
\end{center}
\end{minipage}
\hfill
\vline
\hfill
\begin{minipage}[htp]{0.35\textwidth}
\textbf{16 x 4 Memory Cells (~Special DFFs (Data Flip-Flops))} \\
\begin{center}
    \includegraphics[width=0.45\textwidth]{chapters/chapter1c/images/structure.png}
\end{center}
\end{minipage} \\
\vfill
\begin{minipage}[htp]{0.45\textwidth}
    \subsection{Write Operations}
    \textit{The D is connected to the Data outside of the system and at the risiing edge it updates the value of the DFF.} \textbf{The AND gate ensures that the write signal is high when the clock signal is high.} \\ \vspace*{5px}
    \begin{center}
        \includegraphics[width=0.45\textwidth]{chapters/chapter1c/images/write.png}
    \end{center}
\end{minipage}
\hfill
\vline
\hfill
\begin{minipage}[htp]{0.45\textwidth}
    \subsection{Read Operations}
    \textit{D is still connected to the Data, remember the tri-state driver is active when it's enable signal is active (so when the wr is off and the operation signal is sent.).} \\ \vspace*{5px}
    \begin{center}
        \includegraphics[width=0.45\textwidth]{chapters/chapter1c/images/read.png}
    \end{center}
\end{minipage}

\vspace*{5px}
\subsection{Practical SRAMs}
\textbf{DISCLAIMER !!: Combinational loops are prohibited as they can lead to unstable behavior, unpredictable timing, simulation and synthesis issues, excessive power consumption, and lack of a defined reset state, making them unsuitable for reliable digital circuit design.} \\ \vspace*{5px}
\textit{While the type of memory we've juste seen is small, and very fast, SRAM memories uses 6 transitors per cell (less than the previous design). We've also seen (in Taxonomy) that SRAM is \textbf{static} meaning it doesn't require periodic refresh.} \\ \vspace*{5px}
\begin{minipage}[htp]{0.45\textwidth}
    \begin{center}
        \includegraphics[width=0.55\textwidth]{chapters/chapter1c/images/ram.png}
    \end{center}
    \end{minipage}
    \hfill
    \vline
    \hfill
    \begin{minipage}[htp]{0.45\textwidth}
    \begin{center}
        \includegraphics[width=0.55\textwidth]{chapters/chapter1c/images/sram.png}
    \end{center}
    \end{minipage}

\subsection{DRAMs}
\textit{Dynamic RAMS(DRAMs) are the densest and cheapest type of RAM memory, it stores information as charge in small capacitors. This makes the DRAM need periodic refresh otherwise the charge might leak off (~60ms) the capacitor due to parasitic resistances and the information lost} \\ \vspace*{5px}

\begin{minipage}[htp]{0.45\textwidth}
    \textbf{Refresh means, we come back before the end of a charge (~60ms) and we rewrite the value, if there is still some charge, we add charge, if there's no charge and we keep as is.} \\ \vspace*{5px}
    \textit{Personal Remark: Dynamic = Bad, data dissapears and needs refresh}
\end{minipage}
\hfill
\vline
\hfill
\begin{minipage}[htp]{0.45\textwidth}
    \begin{center}
        \includegraphics[width=0.5\textwidth]{chapters/chapter1c/images/dram.png}
    \end{center}
\end{minipage}

\subsection{Ideal Random Access Memory}
\textit{A memory array uses an \(n\)-to-\(2^n\) decoder to select a word line based on the input address, enabling data to be read or written through the bit lines.}
\begin{center}
    \includegraphics[width=0.45\textwidth]{chapters/chapter1c/images/ideal_ram.png}
\end{center}
\subsection{Physical Organisation }
\begin{center}
    \includegraphics[width=0.45\textwidth]{chapters/chapter1c/images/organisation.png}
\end{center}
\textit{Out of all physical organizations, the squared one is the best one as it has the best performance. This layout facilitates faster access times and simplified wiring, resulting in improved computational efficiency and system scalability.}
\subsection{Realistic ROM Array}
\textit{ROMs are Read-Only Memories, they are used to store the program of the computer, they are non-volatile and can't be written to.} \\ \vspace*{5px}
\begin{center}
    \includegraphics[width=0.45\textwidth]{chapters/chapter1c/images/rom.png}
\end{center}

\subsection{Static Ram Typical Interface}
\textit{This a typical interface of a SRAM, it has a 16-bit data input/output, a 16-bit address input, a write enable signal, and a circuit select signal.}
\begin{center}
    \includegraphics[width=0.45\textwidth]{chapters/chapter1c/images/sram_interface.png}
\end{center}

\section{Typical Asynchronous SRAM Read Cycle}
\textit{The read cycle of an asynchronous SRAM is initiated by the address input, which is decoded to select the word line, enabling the data to be read from the memory array and output to the data bus.} \\ \vspace*{5px}
\textit{Here, Tcyc is the cycle time, Tacc is the access time, and Ten is the enable time.}

\begin{minipage}[htp]{0.45\textwidth}
    \begin{center}
        \includegraphics[width=0.45\textwidth]{chapters/chapter1c/images/sram_read.png}
    \end{center}
\end{minipage}
\hfill
\vline
\hfill
\begin{minipage}[htp]{0.45\textwidth}
    \begin{center}
        \includegraphics[width=0.45\textwidth]{chapters/chapter1c/images/sram_read2.png}
    \end{center}
\end{minipage}

\subsubsection{Read Cycle}
\textit{Latency} defined as the number of cycles between the address asserted and data available \\ \vspace*{5px}
\begin{center}
    \includegraphics[width=0.45\textwidth]{chapters/chapter1c/images/read_cycle.png}
\end{center}
\subsubsection{Write Cycle}
\textit{Writes on the edge of the clock signal, as a DFF} \\ \vspace*{5px}
\begin{center}
    \includegraphics[width=0.45\textwidth]{chapters/chapter1c/images/write_cycle.png}
\end{center}


\section{Where is Memory in the Processor?}
\textit{In the processor we have memory in the Data memory component and in the Instruction memory component.} \\ \vspace*{5px}
\begin{center}
    \includegraphics[width=0.45\textwidth]{chapters/chapter1c/images/processor.png}
\end{center}
\subsection{Arithmetic and Logic Instructions}
\textit{The register file can only contain a limited number of registers making it difficult to handle more complex computations and managing data input/output efficiently.}
\begin{center}
    \includegraphics[width=0.45\textwidth]{chapters/chapter1c/images/arith_logic.png}
\end{center}
\subsubsection{Load Instructions}
\begin{center}
    \includegraphics[width=0.45\textwidth]{chapters/chapter1c/images/load.png}
\end{center}
\subsubsection{Load and Store: The RiSC-V Way}
\textit{This instruction would never work for example because the adress is too big to be sent as an immediate value :} \texttt{lw x5, (x7)} \\ \vspace*{5px}
\begin{center}
    \includegraphics[width=0.45\textwidth]{chapters/chapter1c/images/load.png}
\end{center}
\subsubsection{A Load/Store Architecture}
\textit{A feature of RISC-V is that it's a Load/Store architecture, meaning that the only way to access memory is through load and store instructions. Also, instructions reading and writing in memory do exactly that and nothing else, contrary to more complex instruction set architectures (CISC), where instructions may combine memory access with other operations like arithmetic or logic. This simplicity in RISC-V's instruction set helps with streamlining the pipeline and improving performance efficiency.} \\ \vspace*{5px}

\begin{center}
    \begin{tabular}{|c|c|c|c|c|}
        \hline
        \multicolumn{2}{|c|}{\textbf{Load}} & \textbf{I} & 0x2 & 0x03 \\
        \hline
        \texttt{lw} & \texttt{rd,imm(rs1)} & \multicolumn{3}{c|}{\texttt{rd $\leftarrow$ mem[rs1 + sext(imm)]}} \\
        \hline
        \multicolumn{2}{|c|}{\textbf{Store}} & \textbf{S} & 0x2 & 0x23 \\
        \hline
        \texttt{sw} & \texttt{rs2,imm(rs1)} & \multicolumn{3}{c|}{\texttt{mem[rs1 + sext(imm)] $\leftarrow$ rs2}} \\
        \hline
    \end{tabular}
\end{center}

\section{More Addressing Modes? Not in RISC-V!}
\vspace*{-10px}
\begin{center}
\resizebox{1.1\textwidth}{!}{
\begin{tabular}{|l|l|l|}
\hline
\textbf{Addressing Mode}          & \textbf{Instruction}                                      & \textbf{Description}                                                                                   \\ \hline
\textbf{Register}                 & \texttt{add x0, x1, x2}                                   & Adds the value of \texttt{x1} and \texttt{x2}, stores the result in \texttt{x0}.                         \\ \hline
\textbf{Immediate}                & \texttt{add x0, x1, 123}                                  & Adds the value of \texttt{x1} and the immediate constant 123, stores the result in \texttt{x0}.          \\ \hline
\textbf{Direct or Absolute}       & \texttt{add x0, x1, (1234)}                               & Adds the value of \texttt{x1} and the value at memory address 1234, stores the result in \texttt{x0}.    \\ \hline
\textbf{Register Indirect}        & \texttt{add x0, x1, (x2)}                                 & Adds the value of \texttt{x1} and the value in memory at the address held in \texttt{x2}, stores in \texttt{x0}. \\ \hline
\textbf{Displacement or Relative} & \texttt{add x0, x1, 123(x2)}                              & Adds the value of \texttt{x1} and the value in memory at \texttt{x2} plus the displacement 123, stores in \texttt{x0}. \\ \hline
\textbf{Base or Indexed}          & \texttt{add x0, x1, i5(x2)}                               & Adds the value of \texttt{x1} and the value in memory at \texttt{x2} plus index \texttt{i5}, stores in \texttt{x0}. \\ \hline
\textbf{Auto-increment/-decrement} & \texttt{add x0, x1, (x2+)}                               & Adds the value of \texttt{x1} and the value in memory at the address in \texttt{x2}, then increments \texttt{x2}, stores in \texttt{x0}. \\ \hline
\textbf{PC-Relative}              & \texttt{add x0, x1, 123(pc)}                              & Adds the value of \texttt{x1} and the value in memory at \texttt{pc} plus 123, stores in \texttt{x0}.    \\ \hline
\end{tabular}}
\end{center}

\textit{Syntax here looks like RISC-V but most of these instructions do not exist in RISC-V.}
\newpage
\subsection{Word Adressed Memory}
\textit{In a word addressed memory, the address is the index of the word in the memory.} \\
\textit{The letters inside the word are identified as eg. for Hello World, H:3980, E:3981, L:3982, \dots}.
\begin{center}
    \includegraphics[width=0.45\textwidth]{chapters/chapter1c/images/word_add.png}
\end{center}

\subsection{Loading Words (lw) and Instructions}
\textit{The lw instruction is used to load a word from memory into a register.} \\
\textit{The adress of such words would necessarly be a multiple of 4 meaning the two least significant bits must be 0s.(to the ensure the data is word aligned\dots)} \\
\begin{center}
    \includegraphics[width=0.45\textwidth]{chapters/chapter1c/images/lw.png}
\end{center}
\subsection{Loading Bytes (lb)}
\textit{The \texttt{lb} (Load Byte) instruction doesn't require alignment because it only loads 1 byte (8 bits), which can be accessed at any memory address, unlike \texttt{lw} which requires word alignment to efficiently load 4 bytes (32 bits).
}\textit{The lb instruction is used to load a byte from memory into a register.} \\

\begin{center}
    \includegraphics[width=0.45\textwidth]{chapters/chapter1c/images/lb.png}
\end{center}
\subsection{A Few More Load/Store Instructions}
\textit{Access bytes (and half-words) as if memory were made of bytes}
\begin{center}
    \includegraphics[width=0.45\textwidth]{chapters/chapter1c/images/lb.png}
\end{center}
\subsection{Access as it is more suitable}
\textit{For example storing the "Hello!"zero value in the memory would like this:} \\
\begin{minipage}[htp]{0.45\textwidth}
    \begin{center}
        \includegraphics[width=0.45\textwidth]{chapters/chapter1c/images/hello.png}
    \end{center}
\end{minipage}
\hfill
\vline
\hfill
\begin{minipage}[htp]{ .45\textwidth}
    \begin{center}
        \includegraphics[width=0.3\textwidth]{chapters/chapter1c/images/hello2.png}
    \end{center}
\end{minipage}
\subsubsection{Counting Characters in a String}
\textit{As an example, for counting the number of characters in a string, the load byte instruction would be more suitable as seeing the string as a sequence of bytes makes use of the memory as a sort of array.} \\ \vspace*{5px}

\begin{center}
    \includegraphics[width=0.15\textwidth]{chapters/chapter1c/images/hello2.png}
\end{center}
\begin{center}
    \begin{assembly}
strlen:
    mv t0, a0 # Copy the pointer (a0) into t0 to traverse the string
    li t1, 0 # t1 will hold the length (initialized to 0)
loop:
    lbu t2, 0(t0) # Load byte at address t0 into t2
    beq t2, zero, end # If t2 is 0 (null byte), we are done
    addi t1, t1, 1 # Increment the length counter (t1)
    addi t0, t0, 1 # Point to the next character in the string
j loop # Repeat the loop
end:
    mv a0, t1 # Move the length (t1) into a0 as the return value
    ret # Return to caller
    \end{assembly}
\end{center}

\textit{\texttt{lbu} is used here to ensure that the byte is treated as an unsigned value, which is the correct approach for processing characters in a string.
}
\newpage
In a word addressed memory view, the code would look like such: \\ \vspace*{5px}
\begin{center}
\begin{assembly}
strlen:
    li t1, 0           # t1 will hold the length (initialized to 0)
next_word:
    li t2, 4           # t2 will count the bytes in a loaded word (four)
    lw t3, 0(t0)       # Load four bytes at address t0 into t3
next_byte:
    andi t4, t3, 0xff  # Move the "little-end" in t4
    beq t4, zero, end  # If t4 is 0 (null byte), we are done
    addi t1, t1, 1     # Increment the length counter (t1)
    srli t3, t3, 8     # Prepare the next byte of the word in the "little-end" (t3)
    addi t2, t2, -1    # One byte left in the loaded word
    bnez t2, next_byte # If more bytes in t3, check the next
    addi a0, a0, 4     # Else point to the next word of characters in the string
    j next_word        # Repeat the loop
end:
    mv a0, t1          # Move the length (t1) into a0 as the return value
    ret                # Return to caller
\end{assembly}
\end{center}
\subsection{Loading Bytes (lb)}
\textit{Now, one may wonder in what ordering the bytes are stored in memory.} \\ \vspace*{5px}
\begin{center}
    \includegraphics[width=0.55\textwidth]{chapters/chapter1c/images/lb.png}
\end{center}
\subsubsection{Which Byte Where?}
\textit{Both ordering of bytes are valid the only thing we have to do is stick to one, the most generally used is little-endian as it's the RISCV default and the Intel x86/x64 default.} \\ \vspace*{5px}
\begin{minipage}[htp]{0.45\textwidth}
    \begin{center}
        \textbf{Little Endian} \\ \vspace*{5px}
        \includegraphics[width=0.55\textwidth]{chapters/chapter1c/images/bytes.png}
    \end{center}
\end{minipage}
\hfill
\vline
\hfill
\begin{minipage}[htp]{0.45\textwidth}
    \begin{center}
        \textbf{Big Endian} \\ \vspace*{5px}
        \includegraphics[width=0.55\textwidth]{chapters/chapter1c/images/bytes2.png}
\end{center}
\end{minipage} \\ \vspace*{5px}
\begin{center}
    \includegraphics[width=0.85\textwidth]{chapters/chapter1c/images/final_endian.png} \\
    \href{https://www.google.com/url?sa=i&url=https%3A%2F%2Fmedium.com%2Fmycsdegree%2Fsockets-in-c-little-and-big-endian-machines-23c9ed484c20&psig=AOvVaw1P8zdPW_G0ioJC2Ka6cOX5&ust=1731236021875000&source=images&cd=vfe&opi=89978449&ved=0CBQQjRxqFwoTCPinlvaKz4kDFQAAAAAdAAAAABAJ}{source}
\end{center}
\textit{Personal Remark : Mnemotechnic - Little Endian = Little End (The ending memory index takes the smallest(starting) data adress), Big Endian = Big End.}
\textit{Or, Little Endian = LSB in smallest index, Big Endian = MSB in smallest index.} 
\chapter{Part I(d) - ISA Arrays and Data Structures - W 2.2}
\section{Arrays}
\textit{In higher level languages, are written like follows :}
\begin{cc}
    
\end{cc}
\subsection{Different Ways to Store Arrays}

 

\chapter{Part I(e) - ISA Arithmetic - W 3.1, 3.2}
\section{Notation}
Before we start, let's define some notation:
\begin{itemize}
    \item[-] \textbf{Number representation (with a fixed number of digits/bits):}
    \[
    A = A^{(n)} = A^{(m)}
    \]
    
    \item[-] \textbf{Number in binary or decimal:}
    \[
    A = A_{10} = A_{2} = A_{2c}
    \]
    \textit{With $A_{2c}$ being the 2's complement representation.} \\
    \textit{And $A_{2}$ being the binary representation.}
    \item[-] \textbf{Individual digits or bits:}
    \[
    a_{n-1}, a_{n-2}, \dots, a_2, a_1, a_0
    \]
    
    \item[-] \textbf{Digit string representation:}
    \[
    \langle a_{n-1} a_{n-2} \dots a_2 a_1 a_0 \rangle
    \]
\end{itemize}

\section{Numbers}

Numbers in computing can be represented in different forms, each with specific use cases. \\
\vspace*{5px}
\textbf{Integers} can be either signed or unsigned, representing positive and negative values, or only non-negative values. Examples include:
\[
0, 1, 2, 3, 4294967295, -2147483648
\]

\textbf{Fixed-point} numbers are essentially integers with an implicit scaling factor (e.g., \(10^k\) or \(2^k\)) to handle fractional values. Common in applications like signal processing. Examples include:
\[
0.12, 3.14, 1073741823.75
\]

\textbf{Floating-point} numbers represent a wide range of values using a base and exponent, providing flexibility in precision. Examples include:
\[
3.14E3, -2.5E1, 1.0E0, 4.2E-2, -1.5E-3
\]

\subsection{Unsigned Integers}
Unsigned integers are:
\begin{itemize}
    \item[-] \textit{Weighted}: Each digit has a positional value.
    \item[-] \textit{Nonredundant}: Every number has a unique representation.
    \item[-] \textit{Based on a fixed-radix system}: Typically radix-10 (decimal) or radix-2 (binary).
    \item[-] \textit{Canonical}: Follows a standard form for representation.
\end{itemize}

\textbf{Definition:}
\[
A = \langle a_{n-1} a_{n-2} \dots a_2 a_1 a_0 \rangle = \sum_{i=0}^{n-1} a_i R^i
\]
where \(A\) is the unsigned integer, \(a_i\) are the digits, and \(R\) is the radix.

\subsection{Signed Integers}
We may distinguish between three methods for representing signed integers:
\begin{itemize}
    \item \textbf{Sign-and-Magnitude (SM)}: Uses the most significant bit (MSB) to represent the sign (0 for positive, 1 for negative), with the remaining bits representing the magnitude. This method has the drawback of two zeros (+0 and -0) (Redundant).
    \item \textbf{Two's Complement}(Specific True-and-Complement): The most common way to represent signed integers. It avoids the two-zero problem and simplifies arithmetic operations. Negative numbers are represented by flipping the bits and adding 1.
    \item \textbf{Biased Representation}: Primarily used in floating-point numbers, especially for the exponent part. A fixed bias is added to the actual value to avoid negative exponents. It's rarely used for integers but is another method for handling signed numbers.
\end{itemize}
\subsubsection{Sign and Magnitude}
In the sign-and-magnitude representation, the most significant bit (MSB) is used to represent the sign of the number. The remaining bits represent the magnitude. \\
\vspace*{7px}
\textbf{Definition} \\
\vspace*{3px}
\[
A = \langle s a_{n-2} a_{n-3} \dots a_2 a_1 a_0 \rangle = (-1)^s \cdot \sum_{i=0}^{n-1} a_i R^i
\]
where \(A\) is the signed integer, \(s\) the most significant bit of \(A\) representing the sign of the number, \(a_i\) the digits, and \(R\) the radix. \\
\vspace*{7px}
\textbf{Example (Signed 4-bit integer):} \\
\vspace*{3px}

Consider the 4-bit signed binary number \(1011_2\). In this case: \\
\begin{itemize}
    \item[1.] The MSB \(s = 1\), indicating the number is negative.
    \item[2.] The magnitude bits are \(011_2 = 3_{10}\).
    \item[3.] Therefore, the value of the number is \(-3\).
\end{itemize}

Thus, \(1011_2\) represents \(-3_{10}\) in sign-and-magnitude representation. \\
\vspace*{7px}

\subsection{Radix's Complement}
Radix's complement is a method used to represent signed numbers in different number systems. \\
It is a special form of \textit{true-and-complement} where the complement $C = R^n$, with $R$ being the radix (base) and $n$ the number of digits. \\
\vspace*{5px}
\begin{center}
    \includegraphics[width=0.65\textwidth]{chapters/chapter1e/images/twoscomplement.png}
\end{center}
\textbf{Definition} \\
\vspace*{2px}
A number $A$ in radix's complement is represented as:
\[
A = \langle a_{n-1}a_{n-2}\dots a_1a_0 \rangle = -a_{n-1}R^{n-1} + \sum_{i=0}^{n-2} a_i R^i
\]
where $a_{n-1}$ is the most significant bit, which also indicates the sign (negative for $a_{n-1} = 1$). \\
\vspace*{5px}
For binary numbers, radix’s complement is known as \textbf{two's complement}, which is the most commonly used method for representing signed numbers in digital systems. \\
\vspace*{5px}
\textbf{Binary (2's Complement) Representation} \\
\vspace*{2px}
Two's complement uses base $R = 2$ and has a fixed word length $n$. \\
Here is an example for an 8-bit number system:



\begin{center}
\begin{tabular}{|c|c|c|}
\hline
\textbf{Binary} & \textbf{Decimal} & \textbf{Range} \\
\hline
00000000 & 0   & \multirow{2}{*}{Positive range} \\
01111111 & 127 & \\
\hline
10000000 & -128 & \multirow{2}{*}{Negative range} \\
11111111 & -1   & \\
\hline
\end{tabular}
\end{center}

The two's complement system enables representation of both positive and negative numbers within a fixed bit length. \\
\vspace*{5px}
\textbf{Decimal (10's Complement) Representation} \\
\vspace*{2px}
In a decimal system with radix $R = 10$,\\
We use 10's complement to represent signed numbers. For instance: \\
\[
5,678_{(5)}^{10c} = 05,678_{10c} = +5,678_{10}
\]
This is a positive number representation in 10's complement. For a negative number: \\

\[
9,999,999_{(7)}^{10c} = -1_{10}
\]
Here, $9,999,999$ in 7 digits represents $-1$ in decimal form. \\
\vspace*{5px}
\textbf{Examples of Binary (2's Complement)} \\
\vspace*{2px}
Below are several examples of numbers in binary (2's complement) and their corresponding decimal values: \\
This is a positive binary number. \\
\[
0100,1101,0010_{(12)}^{2c} = 100,1101,0010_2 = +1,234_{10}
\]

This is a negative binary number in 8-bit representation.\\
\[
1111,1111_{(8)}^{2c} = -1_{10}
\]

This is a negative binary number in 12-bit representation.\\
\[
1011,0000,1110_{(12)}^{2c} = -1,234_{10}
\]
\vspace*{5px}

\subsection{Two's Complement Subtraction}

Consider the binary subtraction using the standard paper-and-pencil method:

\[
\begin{array}{cccccccccc}
\text{Borrow:} & -1 & -1 & -1 & & & & -1 & & \\
& 0 & 0 & 0 & 0 & 1 & 0 & 1 & 0 & \quad (10_{10}) \\
- & 0 & 0 & 0 & 1 & 0 & 0 & 0 & 1 & \quad (17_{10}) \\
\hline
  & 1 & 1 & 1 & 1 & 1 & 0 & 0 & 1 \\
\end{array}
\]


Since we had to borrow beyond the most significant bit, the result is negative. The binary result is:
\[
-1\ 1\ 1\ 1\ 1\ 0\ 0\ 1_2
\]

To find its decimal value:
\begin{center}
    $-2^7 + 2^6 + 2^5 + 2^4 + 2^3 + 2^0 = -128 + 64 + 32 + 16 + 8 + 1 =-7$ \\
    and \\
    $
    10_{10} - 17_{10} = -7_{10}
    $
\end{center}


\subsection{Addition Is Unchanged from Unsigned}

In arithmetic operations, addition remains consistent whether using signed or unsigned numbers. The following instructions are available for basic arithmetic operations:
\begin{center}
    \includegraphics[width=0.75\textwidth]{chapters/chapter1e/images/addition.png}
\end{center}
\begin{itemize}
    \item[-] \texttt{add rd, rs1, rs2}: Adds the values in \texttt{rs1} and \texttt{rs2}, and stores the result in \texttt{rd}.
    \item[-] \texttt{addi rd, rs1, imm}: Adds the value in \texttt{rs1} with the sign-extended immediate value \texttt{imm}, and stores the result in \texttt{rd}.
    \item[-] \texttt{sub rd, rs1, rs2}: Subtracts the value in \texttt{rs2} from \texttt{rs1}, and stores the result in \texttt{rd}.
\end{itemize}
 
Note that older architectures (e.g., MIPS) had distinct instructions for signed (\texttt{add}) and unsigned (\texttt{addu}) addition. However, this distinction is unnecessary as the hardware handles both identically. \\
\vspace*{5px}
Sign-and-magnitude addition presents unique challenges, making \textbf{two's complement} the standard for signed integers in modern architectures.

\subsection{Sign Extension}

In digital systems, sign extension is a technique used to increase the bit width of a binary number while preserving its value and sign. It is commonly used when converting a number from a smaller to a larger bit width in a way that maintains its original meaning, whether it's unsigned or in two’s complement format.


\subsubsection{Example: 4-bit to 8-bit Conversion}
Consider the 4-bit two’s complement number \( 1110_2 \), which represents \( -2_{10} \).\\
When extending this number to 8 bits, we replicate the MSB (which is 1 in this case) to fill the additional bits, as shown below: \\
\[
5_{10} = 0101_2 \quad \text{(4 bits)} \to \quad 00000101_2 \quad \text{(8 bits)}.
\] \\
while 
\[
-2_{10} = 1110_2 \quad \text{(4 bits)} \quad \to \quad 11111110_2 \quad \text{(8 bits)}.
\]

This ensures that the number remains \( -2_{10} \) even after increasing the bit width. \\
\vspace*{5px}


\textit{\textbf{Truncation} is allowed when reducing bit width, but only if the truncated bits are redundant (i.e., copies of the sign bit). For example, going from 8 bits back to 4 bits would result in \( 1110_2 \), preserving the value \( -2_{10} \).
}

\subsection{Signed and Unsigned Instructions}
In RISC-V, instructions differentiate between signed (s) and unsigned (u) operations:

\begin{center}
    \includegraphics[width=0.75\textwidth]{chapters/chapter1e/images/ref_card.png}
\end{center}
\begin{itemize}
    \item \textbf{Shift:} \texttt{sra}, \texttt{srai} (s) vs. \texttt{srl}, \texttt{srli} (u). 
    \begin{itemize}
        \item Signed shifts preserve the sign bit, while unsigned shifts insert zeroes.
    \end{itemize}
    
    \item \textbf{Compare:} \texttt{slt}, \texttt{slti} (s) vs. \texttt{sltu}, \texttt{sltiu} (u). 
    \begin{itemize}
        \item Signed comparisons use two's complement, unsigned comparisons ignore sign.
    \end{itemize}
    
    \item \textbf{Branch:} \texttt{blt}, \texttt{bge} (s) vs. \texttt{bltu}, \texttt{bgeu} (u). 
    \begin{itemize}
        \item Signed branches use two's complement; unsigned branches do not consider sign.
    \end{itemize}
    
    \item \textbf{Load:} \texttt{lb}, \texttt{lh} (s) vs. \texttt{lbu}, \texttt{lhu} (u). 
    \begin{itemize}
        \item Signed loads extend the sign bit, while unsigned loads extend with zeroes.
    \end{itemize}
\end{itemize}


\section{Overflow}

Overflow occurs when the result of an arithmetic operation exceeds the range of values that can be represented with a fixed number of bits. This can happen in both unsigned and signed arithmetic, though the detection method differs. In general, overflow results in an incorrect outcome that needs to be detected and handled.

\subsection{Overflow in 2's Complement}

In 2's complement arithmetic, overflow occurs when the result of an addition or subtraction operation falls outside the representable range for the number of bits. For an \(n\)-bit 2's complement system, the representable range is \(-2^{n-1}\) to \(2^{n-1} - 1\).

Overflow is detected by examining the carry into and out of the most significant bit (MSB). Specifically, overflow occurs if:

\[
\text{Overflow} = \text{Cout}_{n-1} \oplus \text{Cout}_n
\]

Where:
\begin{itemize}
    \item[-] \(\text{Cout}_{n-1}\) is the carry into the MSB.
    \item[-] \(\text{Cout}_n\) is the carry out of the MSB.
\end{itemize}

An overflow occurs when these two carry bits differ. This is because the sign of the result is incorrect if there is a mismatch, leading to an incorrect outcome.

\begin{center}
    \includegraphics[width=0.65\textwidth]{chapters/chapter1e/images/sum.png}
\end{center}

For example, if two large positive numbers are added and result in a negative value (or two negative numbers added result in a positive value), this indicates an overflow in 2's complement addition.

\subsection{Overflow in Software}

In many architectures, detecting overflow during arithmetic operations is a critical aspect of software implementation. Overflow occurs when the result of an addition or subtraction exceeds the capacity of the register used to store it. Detection methods vary depending on the type of architecture:

\begin{itemize}
    \item \textbf{Traditional architectures (e.g., x86):} These systems provide a \textit{carry bit} in a special register, known as a flag, that is set when an overflow occurs. Thus, overflow detection operates similarly to hardware-based overflow detection.
    
    \item \textbf{Modern architectures (e.g., RISC-V):} These architectures typically provide only the result of the addition or subtraction without a carry bit. Overflow detection must be handled in software, based on analyzing the sign and magnitude of the result.
\end{itemize}

\begin{center}
    \includegraphics[width=0.65\textwidth]{chapters/chapter1e/images/overflow.png}
\end{center}
Overflow detection can be based on the following observations:
\begin{itemize}
    \item[-] \textbf{Addition of opposite sign numbers:} The magnitude of the result decreases, making overflow impossible.
    \item[-] \textbf{Addition of same sign numbers:} Overflow is possible if the result exceeds the range representable by the register, leading to an incorrect sign in the result.
\end{itemize}
\subsection{Detect Addition Overflow in Software}
\begin{itemize}
    \item[-] Add two 32-bit signed integers and detect overflow
    \begin{itemize}
        \item At call time, \texttt{a0} and \texttt{a1} contain the two integers.
        \item On return, \texttt{a0} contains the result and \texttt{a1} must be nonzero in case of overflow.
    \end{itemize}
\end{itemize}

\begin{assembly}
srai a2, a0, 31       # a2 = sign of a0 (0 or -1)
srai a3, a1, 31       # a3 = sign of a1 (0 or -1)
xor  a4, a2, a3       # a4 = 0 if signs are same, -1 if different
add  a0, a0, a1       # compute sum in a0
srai a5, a0, 31       # a5 = sign of sum (0 or -1)
xor  a6, a2, a5       # a6 = 0 if sign of sum same as a0, -1 if different
and  a1, a4, a6       # a1 = -1 if overflow occurred, else 0
srli a1, a1, 31       # a1 = 1 if overflow occurred, else 0
\end{assembly}

\section{A Strange but Useful Property}
\textit{Personal Remark: don't mistake A and $\overline{A}$ as sets of elements which might confuse you. They are binary numbers.} \\
In binary arithmetic, there is a particularly useful property that can be expressed as follows:

\[
A + \overline{A} = -1
\]
or equivalently,
\[
-A = \overline{A} + 1
\]

\textbf{Proof:} Consider a binary number $A = a_{n-1}2^{n-1} + \sum_{i=0}^{n-2} a_i 2^i$, where $a_i \in \{0,1\}$ represents the binary digits of $A$. The complement of $A$, denoted $\overline{A}$, is given by replacing each $a_i$ with its complement $\overline{a_i}$.

\[
A + \overline{A} = \left( -a_{n-1}2^{n-1} + \sum_{i=0}^{n-2} a_i 2^i \right) + \left( -\overline{a_{n-1}} 2^{n-1} + \sum_{i=0}^{n-2} \overline{a_i} 2^i \right)
\]
\[
= -(a_{n-1} + \overline{a_{n-1}}) \cdot 2^{n-1} + \sum_{i=0}^{n-2} (a_i + \overline{a_i}) \cdot 2^i
\]
\[
= -2^{n-1} + \sum_{i=0}^{n-2} 2^i = -1
\]
\textit{Where $\overline{A}$ is the two's complement of $A$.} \\
\textbf{Intuition:} For each binary digit, adding $a_i$ and its complement $\overline{a_i}$ results in $1$. Therefore, $A + \overline{A}$ consists entirely of $1$s, representing $-1$ in two's complement.
\subsection{Two's Complement Subtractor}
Using the property of two's complement, we can create a subtractor circuit. The subtractor is implemented using an adder, where the number to be subtracted is inverted and incremented by 1.

\begin{itemize}
    \item[-] \textbf{Step 1: Inversion of Subtrahend (B)}\\
    The subtrahend $B$ is inverted using NOT gates, as shown in the diagram. This converts $B$ into its one's complement.
    
    \item[-] \textbf{Step 2: Addition of A and Inverted B}\\
    The full adders (FA) add each bit of the minuend $A$ to the inverted bits of $B$. The full adders also handle any carry-over from the previous addition.

    \item[-] \textbf{Step 3: Add 1 (Two's Complement)}\\
    To complete the two's complement operation, a carry-in of 1 is added to the least significant bit (LSB), which effectively adds 1 to the inverted $B$.

    \item[-] \textbf{Output:}\\
    The sum outputs $S$ ($s_0, s_1, s_2, ...$) represent the result of the subtraction $A - B$, while the final carry-out can be used to detect overflow.
\end{itemize}

\begin{center}
    \includegraphics[width=0.65\textwidth]{chapters/chapter1e/images/substractor.png}
\end{center}

\subsection{Two's Complement Add/Subtract Unit}
This circuit performs both addition and subtraction using two's complement arithmetic. The operation is selected based on the control input signal for subtraction. The unit consists of several key components:

\begin{itemize}
    \item[-] \textbf{Input Inversion:} Each bit of the subtrahend $B$ is passed through a XOR gate controlled by the `subtract` signal. When the `subtract` signal is high (logic 1), the bits of $B$ are inverted to form the two's complement of $B$, effectively switching the operation to subtraction.
    
    \item[-] \textbf{Addition:} The ripple-carry adder, represented by the ADDER block, performs binary addition of the bits from $A$ and $B$. The carry-in ($Cin$) to the least significant bit is used to add 1 when performing subtraction, completing the two's complement process.
    
    \item[-] \textbf{Overflow Detection:} The overflow generator block detects if the result of the addition/subtraction operation has exceeded the range representable by the fixed number of bits. The `overflow` output is asserted in such cases.
    
    \item[-] \textbf{Output:} The result of the operation is provided as the sum output ($S$), representing either the sum $A + B$ or the result of $A - B$, depending on the control signal.
\end{itemize}

\begin{center}
    \includegraphics[width=0.65\textwidth]{chapters/chapter1e/images/adder_substractor.png}
\end{center}


\section{Bounds Check Optimization}
\textit{Very, very, very useful.}
When working with signed integers (e.g., array indices), a common task is to ensure that the index remains within a valid range, typically \(0 \leq t0 < N\), where \(N\) is some predefined boundary. \\
This can be achieved efficiently using a single branch check that combines both lower and upper bound constraints. \\
\vspace*{7px}
\textbf{Single Branch Bound Check} \\
\vspace*{3px}
The instruction \texttt{bgeu} (branch if greater than or equal, unsigned) can perform two checks at once:
\[
\texttt{bgeu}\ t0, t1, \texttt{out\_of\_bound}
\]
Here, \(t0\) is the signed number to be checked, and \(t1 = N\) is the boundary. \\
\vspace*{7px}
\textbf{Explanation} \\
\begin{itemize}
    \item[-] If \(t0 \geq 0\), the behavior of \texttt{bgeu} mimics that of \texttt{bge} (branch if greater than or equal) for signed integers, thus effectively performing an upper bound check.
    \item[-] If \(t0 < 0\), since the comparison is unsigned, \(t0\) will appear as a very large positive value, hence automatically triggering the out-of-bound case.
\end{itemize}

This approach efficiently checks both the lower and upper bounds in one instruction, streamlining the bounds checking process.

\section{Floating Point Representation}

Floating point numbers are widely used in computing to represent real numbers in a way that supports a wide dynamic range. \\
\vspace*{5px}
They are composed of a \textit{significand} (or \textit{mantissa}) and an \textit{exponent} of the base. This representation allows for the approximation of very large and very small values, similar to the way scientific notation is used in everyday practices.\\
\vspace*{5px}
\textbf{Such as}
\begin{align*}
    0.18 \ \mu\text{m} & \quad \rightarrow \quad 0.18 \cdot 10^{-6} \ \text{m} \quad \rightarrow \quad 1.8 \cdot 10^{-7} \ \text{m} \\
    75 \ \text{km} & \quad \rightarrow \quad 75 \cdot 10^{3} \ \text{m} \quad \rightarrow \quad 7.5 \cdot 10^{4} \ \text{m}
\end{align*}


In floating point representation, a number \( X \) is expressed as:
\[
X = (-1)^s \cdot \left(\sum_{i=0}^{n-1} a_i \cdot 2^i \right) \cdot 2^{\left( - e_{m-1} 2^{m-1} + \sum_{j=0}^{m-2} e_j 2^j \right)}
\]
where:
\begin{itemize}
    \item[-] \( s \) is the sign bit,
    \item[-] \( a_i \) represents the bits of the significand (in sign-and-magnitude form),
    \item[-] \( e_j \) represents the bits of the exponent (in 2's complement form).
\end{itemize}

\subsubsection{Properties of Floating Point Numbers}
\begin{itemize}
    \item[-] \textbf{Large dynamic range}, but \textit{variable accuracy}.
    \item[-] Numbers are \textbf{redundant} unless \textit{normalized}.
    \item[-] Floating point operations are \textbf{not associative}, unlike real numbers.
    \item[-] Exponents are typically stored in a \textbf{biased signed representation}, making zero easier to represent and simplifying comparisons in hardware.
    \item[-] The \textbf{mantissa} (significand) is usually normalized such that \(1 \leq m < 2\), with a \textit{hidden bit} to store the leading 1.
\end{itemize}

\subsubsection{Standardization and Hardware Support}
Floating point representation is standardized by the IEEE 754 standard, which is widely adopted in modern computing systems:
\begin{itemize}
    \item[-] \textbf{x86/x64} architectures have supported floating point operations through SSE/AVX extensions since 1999.
    \item[-] \textbf{RISC-V} also includes support for floating point through ISA extensions.
\end{itemize}

\subsubsection{Example: Decimal to IEEE 754 Simple Precision (32 Bits) Conversion}
Convert \( -7.75 \) to IEEE 754 single-precision:

\begin{minipage}[t]{0.45\textwidth}
\vspace*{5px}
\textbf{Step 1: Sign Bit (1 Bit)}\\  
\vspace*{5px}
\( s = 1 \) (negative number). \\
\vspace*{5px} 
\textbf{Step 2: Binary Conversion}\\  
\vspace*{5px}
\( 7_{10} = 111_2 \), \( 0.75_{10} = 0.11_2 \), so \( 7.75_{10} = 111.11_2 \). \\
\textbf{Step 3: Normalize}\\
\vspace*{5px}  
\( 111.11_2 = 1.1111_2 \times 2^2 \). \\
\vspace*{5px}
\textbf{Step 4: Exponent (8 Bits)}\\
\vspace*{5px}  
\( E = 2 + 127 = 129 \), \( 129_{10} = 10000001_2 \).
\end{minipage}
\hfill
\vline
\hfill
\begin{minipage}[t]{0.45\textwidth}

\textbf{Step 5: Mantissa (23 Bits)}\\
\vspace*{3px}  
\begin{justify}
    Take the fractional part after the leading 1 and pad with zeros to make 23 bits:
\end{justify} 
\( 1111 \ 0000 \ 0000 \ 0000 \ 0000 \ 000 \) \\
(fractional part after the leading 1). \\
\vspace*{3px}
\textbf{Step 6: IEEE 754 Representation}\\
\vspace*{3px}  
\[
\boxed{
  \underbrace{1}_{\text{Sign bit}} \ 
  \underbrace{10000001}_{\text{Exponent bits}} \ 
  \underbrace{1111 \ 0000 \ 0000 \ 0000 \ 0000 \ 000}_{\text{Mantissa}}
}
\]
\end{minipage}

\subsection{Sign-and-Magnitude Addition} (Assembly)

In this exercise, we aim to write a function in RISC-V assembler to sum two 32-bit signed numbers represented in sign-and-magnitude (S\&M) format. The result should also be produced in the sign-and-magnitude format.

\begin{itemize}
    \item[-] The two operands are stored in registers \texttt{a0} and \texttt{a1} on entry.
    \item[-] The result should be placed in register \texttt{a0}.
    \item[-] Overflow cases should be ignored.
\end{itemize}

\subsubsection{Solution 1}

\textbf{Basic Algorithm:}
\begin{itemize}
    \item If the operands have the same sign:
    \begin{itemize}
        \item Add the absolute values
        \item Attach to the result the same sign as the operands
    \end{itemize}
    \item If the operands have different signs:
    \begin{itemize}
        \item Subtract the smallest absolute value from the largest one
        \item Attach to the result the sign of the largest value
    \end{itemize}
\end{itemize}

\begin{assembly}
add_sandm:
    lui     t1, 0x80000         # mask for sign bit
    and     t0, a0, t1          # check a0 sign
    beqz    t0, a0_positive     # if positive, skip
    xor     a0, a0, t1          # flip sign bit
    neg     a0, a0              # negate a0

a0_positive:
    and     t0, a1, t1          # check a1 sign
    beqz    t0, a1_positive     # if positive, skip
    xor     a1, a1, t1          # flip sign bit
    neg     a1, a1              # negate a1

a1_positive:
    add     a0, a0, a1          # add values
    and     t0, a0, t1          # check result sign
    beqz    t0, sum_positive    # if positive, skip
    neg     a1, a1              # negate a1
    xor     a0, a0, t1          # flip sign bit

sum_positive:
    ret                         # return result
\end{assembly}
\newpage
\subsubsection{Solution 2}

\textbf{Basic Algorithm:}
\begin{itemize}
    \item Convert the two operands from sign-and-magnitude to 2's complement
    \item Add the two operands
    \item Convert the result from 2's complement back to sign-and-magnitude
\end{itemize}

\begin{assembly}
convert_to_twos_comp:
    lui     t1, 0x80000         # mask for sign bit
    and     t0, a0, t1          # check a0 sign
    beqz    t0, a0_twos_comp    # if positive, skip
    xor     a0, a0, t1          # flip sign bit
    neg     a0, a0              # negate a0

a0_twos_comp:
    and     t0, a1, t1          # check a1 sign
    beqz    t0, a1_twos_comp    # if positive, skip
    xor     a1, a1, t1          # flip sign bit
    neg     a1, a1              # negate a1

add_twos_comp:
    add     a0, a0, a1          # add values

convert_to_sign_mag:
    lui     t1, 0x80000         # mask for sign bit
    and     t0, a0, t1          # check result sign
    beqz    t0, result_positive # if positive, skip
    xor     a0, a0, t1          # flip sign bit
    neg     a0, a0              # negate result

result_positive:
    ret                         # return result
\end{assembly}













 
\chapter{Part II(a) - I/O - Exceptions Multicycle Processor W - 3.2, 4.1}
\footnotesize
In this chapter we will be discussing how we can actually design a processor (subject of our LAB B). \\
\section{Processor}
\begin{minipage}[htp]{0.45\textwidth}
\footnotesize
(yes, one more time) A processor is composed of several fundamental components that work together to perform computations.  \\ \vspace*{5px}

- \textbf{Program Counter (PC):} Holds the address of the next instruction to be executed from the instruction memory. It increments after each instruction fetch or is updated based on control logic.   \\ \vspace*{5px}                

- \textbf{Instruction Memory:} Stores the instructions that the processor fetches and executes. Instructions are read sequentially unless altered by control logic. \\ \vspace*{5px}

- \textbf{Control Logic:} Manages the flow of data and the sequence of operations, including reading instructions, decoding them, and updating the program counter. \\ \vspace*{5px}

- \textbf{Register File:} A set of registers where data is temporarily stored. It allows the processor to access and manipulate values quickly. Each register has read/write capabilities.  \\ \vspace*{5px}

- \textbf{Arithmetic Logic Unit (ALU):} Performs arithmetic and logical operations. The inputs are provided by the register file, and the result is stored back into the registers or data memory. \\ \vspace*{5px}
    
- \textbf{Data Memory:} Stores data that can be written to or read from during program execution. It interacts with both the register file and the ALU for storing operands and results. \\ \vspace*{5px}
\end{minipage}
\hfill
\vline
\hfill
\begin{minipage}[htp]{0.45\textwidth}
    \begin{center}
        \includegraphics[width=1.2\textwidth]{chapters/chapter2a/images/processor.png}
    \end{center}
\end{minipage}

\subsection{Unified Memory}
\textit{In the image above, we see that the data memory and instruction memory are separate. However, a choice that is often made is to have a unified memory.}
\begin{center}
    \includegraphics[width=0.75\textwidth]{chapters/chapter2a/images/unified.png}
\end{center}
\subsection{Single-Cycle Processor}
At the end, like most circuits, a processor is just another Finite State Machine. The simplified state diagram of a single-cycle processor would like this:
\begin{center}
    \includegraphics[width=0.35\textwidth]{chapters/chapter2a/images/fsm.png}
\end{center}
\textit{Execute an instruction, move to the next, repeat.} \\
This simplified view doesn't reflect actual CPU design. In reality, instructions take different amounts of time due to complexity and \textbf{Propagation Time}—the delay in signal travel through the processor.

\section{Propagation Time}
Remember the difference \textbf{(this is absolutely critical to understand the rest of the course)} between combinational circuits and sequential circuits. \\ \vspace*{5px} 
As the name suggests, sequential circuits are built like a \textit{sequence}(mnemotechnic), meaning the current output depends on both the current input and the previous state. \\ \vspace*{5px} 
While combinational circuits, don't have a memory, they just take an input and give out an output.  \\ \vspace*{5px}
\begin{minipage}[htp]{0.45\textwidth}
The main thing to understand here is that, for our circuits to function as intended, the \textbf{propagation time} must allow the combinational circuits to complete before the next clock cycle (otherwise, it would lead to \textit{obvious bugs}). \\ \vspace*{5px}
This implies that we need to observe the \textbf{longest combinational path} and account for it when designing our circuits.\\ \vspace*{5px}

While this is the \textit{efficient approach}, one could, in theory, design a propagation time that is longer than the longest path. However, this would result in a \textit{waste of both time and resources}.\\ \vspace*{5px}
Remember, lower propagation time means higher clock frequency, which means faster processing.
\end{minipage}
\hfill
\vline
\hfill
\begin{minipage}[htp]{0.45\textwidth}
\begin{center}
    \includegraphics[width=1.2\textwidth]{chapters/chapter2a/images/prop_time.png}
\end{center}
\end{minipage}\\


\subsection{Increasing the Frequency}
To increase the frequency, we need to decrease the propagation time. This can be achieved by breaking down the combinational path into smaller parts. \\
For example, consider the `lw` instruction. This requires adding the offset to the base address (which involves addition, not completely trivial), and then reading the data from memory. This process can be broken down into two stages: first, the addition, and then the memory read. \\
\begin{center}
    \includegraphics[width=0.45\textwidth]{chapters/chapter2a/images/incr_freq.png}
\end{center}
By doing this, we can operate at twice the \textit{""speed""}(we'll see why this is wrong in a moment). \\

\subsection{Two-Cycle Processor}
However, what we quickly realize is that this approach doesn't result in a real performance gain. While the processor runs at twice the frequency, it also takes twice as long to complete the instruction, leading to no overall improvement. \\
\textit{Historically, Intel often used this strategy to persuade uninformed consumers that their processors were getting faster.} \\
\begin{center}
    \includegraphics[width=0.45\textwidth]{chapters/chapter2a/images/two_cyc_processor.png}
\end{center}

\subsection{Not All Paths Are Born Equal}
The reason we're discussing this is that not all paths are equal. Some instructions are faster to compute than others. \\
For example, the \texttt{andi} instruction is much faster than the \texttt{lw} instruction. \\
\begin{center}
    \includegraphics[width=0.65\textwidth]{chapters/chapter2a/images/paths.png}
\end{center}

\subsection{Asynchronous/Synchronous Memories}
Another reason why breaking down the combinational path could be beneficial is that certain memories are \textbf{Synchronous}, meaning they only read data from a valid memory address on the rising edge of the clock cycle. \\ 
On the other hand, \textbf{Asynchronous} memories read data as soon as a valid memory address is available, without waiting for the clock cycle.\\ \vspace*{5px}
So, for \textbf{Synchronous} memories, breaking down combinational paths into smaller segments allows us to increase the clock frequency, making memory updates faster. \\ \vspace*{5px}
\begin{minipage}[htp]{0.45\textwidth}
    \begin{center}
        \includegraphics[width=0.65\textwidth]{chapters/chapter2a/images/seq_memory.png}
    \end{center}
\end{minipage}
\hfill
\vline
\hfill
\begin{minipage}[htp]{0.45\textwidth}
    \begin{center}
        \includegraphics[width=0.85\textwidth]{chapters/chapter2a/images/seq_memory2.png}
    \end{center}
\end{minipage}

\section{Multicycle Processor}
\textit{Now let's try to construct a more convincing representation for our processor.}

The processor operates in two cycles: a faster path for simple instructions and a slower path for more complex ones. \\
\begin{minipage}[htp]{0.45\textwidth}
\vspace*{5px}
\footnotesize
\begin{justify}
        - \textbf{Fetch1/Fetch2}: 
        \begin{itemize}
        \item[] \textit{Simple}: Uses only Fetch1 for single-word instructions.
        \item[] \textit{Complex}: Uses Fetch2 to fetch additional data when needed (e.g., multi-word instructions).
        \end{itemize}
    
        - \textbf{Decode}: 
        \begin{itemize}
        \item[] \textit{Simple}: Quick decoding with fewer control signals.
        \item[] \textit{Complex}: More control signals and operands, requiring extra decoding time(and extra Optimization could be to introduce two Decoding stages for simple/complex instructions).
        \end{itemize}
    
        - \textbf{Execute}: 
        \begin{itemize}
        \item[] \textit{Simple}: Fast ALU operations like additions.
        \item[] \textit{Complex}: Involves branches or complex ALU operations.
        \end{itemize}
    
        - \textbf{Load1/Load2}: 
        \begin{itemize}
        \item[] \textit{Simple}: Skips Load stages if no memory access.
        \item[] \textit{Complex}: Memory operations use Load1 and Load2 to fetch and process data.
        \end{itemize}
\end{justify}
\end{minipage}
\hfill
\vline
\hfill
\begin{minipage}[htp]{0.45\textwidth}
    \begin{center}
        \includegraphics[width=0.70\textwidth]{chapters/chapter2a/images/multi_cyc.png}
    \end{center}
\end{minipage} \\
\vspace*{5px}
While this is an efficient design, it is not unique. The two things to keep in mind when designing a processor are: 
\begin{itemize} 
    \item[] not to have too many stages, \textit{meaning that having an excessive number of stages could increase the complexity and latency of the processor (this we will see later in the course).} 
    \item[] to have paths as balanced as possible, \textit{meaning that the duration of each stage should be similar to avoid bottlenecks that would slow down the overall process. The more balance we have the more we can profit from fast cases.} 
\end{itemize}
\newpage
\section{Mealy or Moore?}
\textit{Personal Remark (mnemotechnic)} \\
\textit{Moore - Output Only depends on state (double O like in Moore),}\\
\textit{Mealy - Output depends on state and input}\\
\begin{center}
    \includegraphics[width=1\textwidth]{chapters/chapter2a/images/moore_mealy.png}
\end{center}
\textit{It is generally preferable to use Moore state machines because their outputs depend only on the current state, making them simpler to design, debug, and predict, whereas Mealy machines depend on both state and input, introducing complexity and potential glitches. So unless the speicifcations requires us to do otherwise, we'll generally tend to represent our state machines as Moore machines.} \\
\section{Processor - Building the Circuit}
In this part, we will be incrementally adding the components needed to build our processor circuit.  \\
\vspace*{4px}
\begin{minipage}[htp]{0.45\textwidth}
    For now, we've added two components to our CPU:\\ \vspace*{4px}
        \textbf{Controller:} This component, although empty for now, will eventually manage the flow of data and sequence of operations within the CPU. It will control how data moves and instructions are executed. \\
        \vspace*{4px}
        \textbf{PC (Program Counter):} The PC holds the address of the next instruction to be executed from the instruction memory. It increments after each instruction fetch or is updated based on control logic. \\ \vspace*{5px}
            \textbf{Inputs} 
            \begin{itemize}
                \item \texttt{clk}: The clock input ensures the program counter updates synchronously with the system clock.
                \item \texttt{rst\_n}: An active-low reset signal that resets the program counter to a default value when low (0).
                \item \texttt{en}: The enable signal controls whether the PC updates its value (controlled by the Controller's FSM).
            \end{itemize}
            \textbf{Outputs}
            \begin{itemize}
                \item \texttt{addr}: The address output representing the next instruction to be fetched from memory.
            \end{itemize}
    \end{minipage}
\hfill
\vline
\hfill
\begin{minipage}[htp]{0.45\textwidth}
	\begin{center}
\includegraphics[width=1.2\textwidth]{chapters/chapter2a/images/p1.png}
\end{center}
\end{minipage}

\subsection{Adding the Instruction Register}
Now we're adding an Instruction Register.
\begin{center}
\includegraphics[width=0.75\textwidth]{chapters/chapter2a/images/p2.png}
\end{center}
In this step, we introduce the Instruction Register (IR) to our CPU: \\
\textbf{Instruction Register (IR):} The IR is responsible for storing the instruction fetched from memory. It captures the instruction ready to be decoded and executed. The Controller generates enable signals to control when the PC and IR should update their contents. \\
\noindent
\begin{minipage}[t]{0.3\textwidth}
    \footnotesize
    \textbf{PC (Program Counter)} \\ \vspace*{5px}
        \textbf{Inputs} 
        \begin{itemize}
            \item \texttt{clk}: The clock input ensures the program counter updates synchronously with the system clock.
            \item \texttt{rst\_n}: An active-low reset signal resets the program counter to a default value when low (0).
            \item \texttt{en}: The enable signal controls whether the PC updates its value. It is driven by the FSM in the Controller.
        \end{itemize}
         \textbf{Outputs}
        \begin{itemize}
            \item \texttt{addr}: The address output representing the next instruction to be fetched from memory.
        \end{itemize}
\end{minipage}
\hfill
\vline
\hfill
\begin{minipage}[t]{0.3\textwidth}
    \footnotesize
    \textbf{IR (Instruction Register)} \\ \vspace*{5px}
       \textbf{Inputs}
        \begin{itemize}
            \item \texttt{clk}: Ensures the instruction register captures the instruction at the correct clock edge.
            \item \texttt{rst\_n}: Active-low reset to reset the IR to its default state.
            \item \texttt{en}: The enable signal controls whether the IR updates its contents. It is activated when a new instruction is fetched from memory.
            \item \texttt{D}: The data input, which represents the instruction fetched from memory (\texttt{rdata}).
        \end{itemize}
        \textbf{Outputs}
        \begin{itemize}
            \item \texttt{Q}: The output of the instruction register, representing the stored instruction that will be decoded and executed.
        \end{itemize}

\end{minipage}
\hfill
\vline
\hfill
\begin{minipage}[t]{0.3\textwidth}
    \footnotesize
    \textbf{Controller} \\ \vspace*{5px}
        \textbf{Inputs}
        \begin{itemize}
            \item \texttt{clk}: The clock signal to ensure synchronization with other components.
            \item \texttt{rst\_n}: The active-low reset signal to reset the controller to its initial state.
        \end{itemize}
        \textbf{Outputs}
        \begin{itemize}
            \item \texttt{pc\_en}: The enable signal sent to the Program Counter (PC) to control when it should update its value.
            \item \texttt{ir\_en}: The enable signal sent to the Instruction Register (IR) to control when it should store a new instruction.
        \end{itemize}
\end{minipage}
\newpage
\subsection{Adding functionality}
Once an instruction is fetched from memory and stored in the Instruction Register (IR), it is crucial for the Controller to receive this instruction. The Controller needs the instruction to determine the next sequence of operations, as the next state of the system is dependent on the instruction being executed. \\
\begin{minipage}[htp]{0.45\textwidth}
    The \texttt{Q} output of the IR, which holds the stored instruction, is fed directly to the Controller. This connection allows the Controller to decode the instruction and control the subsequent operations of the CPU. \\
Specifically, the Controller will enable or disable other components, such as the Program Counter (PC), based on the instruction being processed.

\end{minipage}
\hfill
\vline
\hfill
\begin{minipage}[htp]{0.45\textwidth}
    \begin{center}
        \includegraphics[width=1.2\textwidth]{chapters/chapter2a/images/p3.png}
    \end{center}
        
\end{minipage}


\subsection{I-Type Instructions Need RF and ALU}
I-Type instructions such as \texttt{addi t0, t1, 1234} require both the register file (RF) and the Arithmetic Logic Unit (ALU) for execution. The operation consists of an addition between a register value and an immediate value, and the result is stored back into a register.
\begin{center}
    \includegraphics[width=0.5\textwidth]{chapters/chapter2a/images/p4.png}
\end{center}
\noindent
\begin{minipage}[t]{0.45\textwidth}
    \footnotesize
    \textbf{ALU (Arithmetic Logic Unit)} \\ \vspace*{5px}
        \textbf{Inputs} 
        \begin{itemize}
            \item \texttt{a}: First operand input from the register file (e.g., \texttt{t1}).
            \item \texttt{b}: Second operand input, typically the immediate value for I-type instructions (e.g., \texttt{1234}).
            \item \texttt{op\_alu}: Control signal from the controller specifying the operation to perform (e.g., addition for the \texttt{addi} instruction).
        \end{itemize}
         \textbf{Outputs}
        \begin{itemize}
            \item \texttt{alu\_out}: The result of the operation performed by the ALU (e.g., the sum of \texttt{t1} and the immediate value).
        \end{itemize}
\end{minipage}
\hfill
\vline
\hfill
\begin{minipage}[t]{0.45\textwidth}
    \footnotesize
    \textbf{Register File} \\ \vspace*{5px}
       \textbf{Inputs}
        \begin{itemize}
            \item \texttt{clk}: The clock input that ensures register updates are synchronous with the system clock.
            \item \texttt{aa}: The address of the first register (e.g., \texttt{t1}) from which data will be read.
            \item \texttt{ab}: The address of the second register (for other instruction types).
            \item \texttt{aw}: The address of the destination register (e.g., \texttt{t0}) to which the result will be written.
            \item \texttt{wren}: Write enable signal that allows data to be written into the destination register.
            \item \texttt{wrdata}: The data to be written into the destination register (e.g., the result from the ALU).
        \end{itemize}
        \textbf{Outputs}
        \begin{itemize}
            \item \texttt{a}: The data from the first register (e.g., the value stored in \texttt{t1}).
            \item \texttt{b}: The data from the second register (for other instruction types).
        \end{itemize}

\end{minipage}


\subsection{R-Type Instructions and Second Operand Selection}

R-Type instructions, such as \texttt{add t0, t1, t2}, require two register operands and involve several components for execution. The instruction specifies two source registers (\texttt{t1} and \texttt{t2}) and a destination register (\texttt{t0}), with the second operand selected from the register file rather than using an immediate value. The multiplexer plays a key role in selecting the correct second operand based on the instruction type. \\
\begin{center}
    \includegraphics[width=0.75\textwidth]{chapters/chapter2a/images/p5.png}
\end{center}
\noindent

\begin{minipage}[t]{0.3\textwidth}
    \footnotesize
    \textbf{Register File} \\ \vspace*{5px}
    \textbf{Inputs}
    \begin{itemize}
        \item \texttt{clk}: The clock input that ensures register updates are synchronous with the system clock.
        \item \texttt{aa}: The address of the first register (e.g., \texttt{t1}) from which data will be read.
        \item \texttt{ab}: The address of the second register (e.g., \texttt{t2}) from which data will be read.
        \item \texttt{aw}: The address of the destination register (e.g., \texttt{t0}) where the result will be written.
        \item \texttt{wren}: Write enable signal that allows data to be written into the destination register.
        \item \texttt{wrdata}: Data to be written into the destination register (e.g., the result from the ALU).
    \end{itemize}
    \textbf{Outputs}
    \begin{itemize}
        \item \texttt{a}: The data from the first register (e.g., the value stored in \texttt{t1}).
        \item \texttt{b}: The data from the second register (e.g., the value stored in \texttt{t2}).
    \end{itemize}
\end{minipage}
\hfill
\vline
\hfill
\begin{minipage}[t]{0.3\textwidth}
    \footnotesize
    \textbf{Multiplexer (sel\_b)} \\ \vspace*{5px}
    \textbf{Inputs}
    \begin{itemize}
        \item \texttt{b}: The second operand, which can either be the register value (\texttt{t2}) or an immediate value, depending on the instruction type.
        \item \texttt{sel\_b}: The select signal from the controller, determining whether the second operand is a register value (\texttt{t2}) or an immediate value.
    \end{itemize}
    \textbf{Outputs}
    \begin{itemize}
        \item \texttt{selected\_b}: The selected operand output, which forwards either the register value (\texttt{t2}) or the immediate value to the ALU as the second operand.
    \end{itemize}
\end{minipage}
\hfill
\vline
\hfill
\begin{minipage}[t]{0.3\textwidth}
    \footnotesize
    \textbf{ALU (Arithmetic Logic Unit)} \\ \vspace*{5px}
    \textbf{Inputs}
    \begin{itemize}
        \item \texttt{a}: First operand input from the register file (e.g., \texttt{t1}).
        \item \texttt{b}: Second operand input, selected by the multiplexer, from the register file (e.g., \texttt{t2}).
        \item \texttt{op\_alu}: Control signal from the controller specifying the operation to perform (e.g., addition for the \texttt{add} instruction).
    \end{itemize}
    \textbf{Outputs}
    \begin{itemize}
        \item \texttt{alu\_out}: The result of the operation performed by the ALU (e.g., the sum of \texttt{t1} and \texttt{t2}), which is written back into the destination register.
    \end{itemize}
\end{minipage}

\subsection{And More, and More...}
\textit{After these few additions, you basically get the point, we keep adding, block by block, the components we need for the full use of our processor, professor also goes pretty quickly over this (you'll also see the full implementation of this in LAB B.)}
\\  
The rest of the additions being : \\
\begin{itemize}
    \item[-] U-Type Instructions Write an Immediate
    \item[-] Load and Stores Produce a Memory Address
    \item[-] Loads Write the Read Data into the RF
    \item[-] Stores Send an Operand to Memory
    \item[-] Branches Need to Write an Offset to the PC
    \item[-] jal Needs to Store PC + 4 in the RF
    \item[-] Jumps Need to Write an Address to the PC
\end{itemize}
The processor after all of this looks like this:
\begin{center}
    \includegraphics[width=0.75\textwidth]{chapters/chapter2a/images/pend.png}
\end{center}

\subsection{Guidelines for Writing Verilog}
Before beginning to write Verilog code, it is crucial to follow certain guidelines to ensure clarity and correctness in your hardware design. Verilog and VHDL are Hardware Description Languages (HDLs) that require a clear and structured approach. \\
\textit{Anything that's complicated is a Module, anything that is trivial, we need to know if it's sequential or combinational.}
\begin{itemize}
    \item \textbf{Clarity and Preparation:} 
    \begin{itemize}
        \item Ensure that you have drawn a diagram, as demonstrated in previous examples.
        \item Clearly distinguish between \textbf{combinational} and \textbf{sequential} blocks in your design.
    \end{itemize}

    \item \textbf{Decomposition of Complex Sequential Blocks:}
    \begin{itemize}
        \item Break down complex sequential blocks into simpler, well-defined elements. For instance, sequential blocks should primarily consist of simple registers (e.g., Instruction Register - IR).
        \item Continue refining your hierarchical diagrams until all sequential blocks become trivial to implement.
    \end{itemize}

    \item \textbf{Adopt a Hierarchical Approach:}
    \begin{itemize}
        \item Use a hierarchical approach, similar to programming practices, and employ your diagrams to guide the creation of modules, such as the Program Counter (PC).
    \end{itemize}
\end{itemize}

\textit{For example, for our processor, identifying that a register file is sequential while a PC is combinational is crucial before starting to write Verilog.}

\subsection{Detailing Complex Combinational Modules (ALU)}
When designing complex combinational modules, it is essential to clearly define and break down each component to ensure accurate and efficient implementation. The following steps outline the process of detailing these modules:
\begin{center}
    \includegraphics[width=0.55\textwidth]{chapters/chapter2a/images/ALU.png}
\end{center}
\begin{itemize}
    \item \textbf{ALU (Arithmetic Logic Unit) Overview:}
    \begin{itemize}
        \item The ALU receives inputs \( A \), \( B \), and an operation code (\textit{op}), and produces an output \( S \).
        \item It contains multiple submodules, such as add/subtract, comparator, logical unit, and shift unit.
    \end{itemize}

    \item \textbf{Add/Subtract Unit:}
    \begin{itemize}
        \item The add/subtract unit performs addition and subtraction operations based on the control signal \textit{sub}. 
        \item It includes circuitry to handle carry and zero detection, essential for arithmetic operations.
    \end{itemize}

    \item \textbf{Hierarchical Design:}
    \begin{itemize}
        \item The ALU is composed hierarchically, where each submodule (e.g., add/subtract, comparator) performs specific functions and connects to the overall ALU structure.
        \item Such a design allows for easier debugging, maintenance, and understanding of each module’s role within the ALU.
    \end{itemize}
\end{itemize}

\subsection{Verilog - Sticking to Basic Paterns}
When writing Verilog, it is essential to adhere to basic patterns for describing combinational and sequential logic. This section provides guidelines on structuring Verilog code efficiently.

\begin{center}
    \begin{minipage}{0.45\textwidth}
        \textbf{Combinational Logic} \vspace{0.5em} \\
        Combinational logic blocks should be described using the \texttt{always @(*)} construct. This approach ensures that outputs are updated whenever the inputs change.
        \begin{verilog}
always @(*) begin
    if (a) 
        y = \~b;
    else 
        y = b;
end
        \end{verilog}

        Complex combinational blocks, such as the next state in a finite state machine (FSM), can also be described using this pattern.
    \end{minipage}
    \hfill
    \vline
    \hfill
    \begin{minipage}{0.45\textwidth}
\textbf{Sequential Logic} \vspace{0.5em} \\
Sequential logic blocks should be described using the \texttt{always @(posedge clk)} construct. This pattern is suitable for describing registers and counters.

\begin{verilog}
always @(posedge clk) begin
    if (reset == 1) 
        q <= 0;
    else if ((enable1 == 1) && (enable2 == 1)) 
        q <= d;
end
        \end{verilog}

        Use \texttt{posedge clk} to trigger updates on the rising clock edge.
    \end{minipage}
\end{center}

For detailed guidelines, refer to the Verilog guidelines provided in Moodle.
 
\chapter{Part II(b) - Processor, I/Os, and Exceptions W - 4.2}
\section{The CPU}

The CPU is a very sequential component responsible for executing instructions in a controlled manner. The CPU interacts with the memory through a defined memory interface, which includes various control signals and data pathways.
\begin{center}
    \includegraphics[width=0.45\textwidth]{chapters/chapter2b/images/cpu.png}
\end{center}
\begin{itemize}
    \item[-] \textbf{Control Signals (ctrl)}: These signals manage the behavior of memory access, indicating whether to read or write data.
    \item[-] \textbf{Address (addr)}: Specifies the memory address where the CPU wants to read or write data. The width of the address bus is typically 32 bits or more.
    \item[-] \textbf{Read Data (rdata)}: A 32-bit pathway through which the CPU receives data from the memory.
    \item[-] \textbf{Write Data (wdata)}: A 32-bit pathway through which the CPU sends data to be stored in memory.
\end{itemize}

The memory interface is also controlled by two important signals:
\begin{itemize}
    \item[-] \textbf{Circuit Enable (CE)}: Validates the address, indicating that the address provided is active and the operation should proceed.
    \item[-] \textbf{Write Enable (WE)}: Indicates that the current access is a store operation, allowing data to be written into memory.
\end{itemize}
\textbf{From now on, the clock signal, which drives the sequential behavior, may be omitted for simplicity.}
This interface design allows for a clear and structured method of communication between the CPU and memory, ensuring reliable execution of instructions and data management.



 
\chapter{Part II(c) - Interrupts W - 5.1 - 5.2}

\section{I/O Polling}

I/O Polling is a method used by the CPU to check if any peripheral devices, such as a keyboard or network interface, have data to provide. The CPU continuously monitors each connected I/O device at regular intervals to see if they need attention.
\begin{center}
    \includegraphics[width=0.65\textwidth]{chapters/chapter2c/images/IOP.png}
\end{center}
\begin{itemize}
    \item[] \textbf{How It Works:} The CPU keeps visiting each I/O device in a loop to check for input or status changes. This is known as "polling" the devices.
    \item[] \textbf{Drawbacks:} This approach can be very resource-intensive. If a device operates at high speed and requires immediate handling, the CPU must check it frequently, which can consume significant processing time.
\end{itemize}

\section{I/O Interrupts}
\textit{Instead of continuously checking the status of peripherals, it is more efficient to have them \textit{request attention} when needed. This approach minimizes CPU usage by eliminating the need for constant polling.}
\begin{itemize}
    \item[-] \textbf{Polling Method:} The CPU checks the status of a peripheral device by repeatedly executing a loop to monitor the peripheral register. This approach requires continuous CPU attention, which can be inefficient in systems with multiple peripherals. \\
    \item[-] \textbf{Interrupt Method:} In an interrupt-driven approach, peripherals alert the CPU only when they need attention. The CPU executes an interrupt service routine (ISR) to handle the request. This method allows the CPU to focus on other tasks until interrupted, improving efficiency.
\end{itemize}

\begin{center}
    \includegraphics[width=0.65\textwidth]{chapters/chapter2c/images/IOI.png}
\end{center}

\subsection{The Basic Concept of I/O Interrupts}

I/O interrupts provide a mechanism for a controller to handle external requests efficiently by temporarily diverting program execution. \\
\textbf{This is not the actual implementation, but basic concept for you to help you understand what we're aiming for.}
\begin{center}
    \includegraphics[width=0.65\textwidth]{chapters/chapter2c/images/basic_idea.png}
\end{center}

\begin{itemize}
    \item \textbf{Interrupt Request (IRQ):} An interrupt signal is triggered, typically from an I/O device, to request immediate attention from the controller.

    \item \textbf{Program Counter (PC) Preservation:} The current value of the Program Counter (PC), which holds the address of the next instruction, is saved to allow resumption of normal execution after the interrupt is handled.

    \item \textbf{Interrupt Service Routine (ISR):} The controller redirects the PC to the address of the interrupt handler function, denoted here as \texttt{read\_adc}. This function processes the interrupt by executing specific instructions related to the I/O request.

    \item \textbf{Instruction Memory Access:} The \texttt{Instruction Memory} is accessed to fetch instructions at the new PC address, executing the ISR for the interrupt.

    \item \textbf{Resuming Program Execution:} Once the interrupt has been serviced, the controller restores the saved PC value, allowing the program to continue from the point it was interrupted.
\end{itemize}

\textbf{Considerations for I/O Interrupt Handling} \\
When managing multiple I/O interrupts, several issues must be addressed:

\begin{itemize}
    \item \textbf{Identifying the Source of the Interrupt:} In systems with multiple peripherals, it is essential to determine which device triggered the interrupt. This can be achieved through:
    \begin{itemize}
        \item \textit{Polling:} After an interrupt, the software checks each peripheral sequentially.
        \item \textit{Identification by the Peripheral:} The I/O peripheral itself sends an identification signal.
    \end{itemize}

    \item \textbf{Handling Different Priorities:} Some interrupts may require immediate attention, while others can be delayed. Assigning priorities ensures critical interrupts are serviced promptly, while less urgent ones may wait.

    \item \textbf{Impact on Current Execution:} The system must decide whether to allow the current instruction(s) to complete before handling the interrupt or to pause immediately. This decision impacts program flow and execution timing.
\end{itemize}
\newpage
\subsection{Interrupt Cycle Description}

The interrupt cycle is a sequence where a peripheral device signals an interrupt to the processor, which responds by acknowledging the interrupt and reading the device identifier from the data bus. The following signals are involved in this process:
\begin{center}
    \includegraphics[width=0.65\textwidth]{chapters/chapter2c/images/interrupt.png}
\end{center}
\begin{itemize}
    \item \textbf{Clock Signal}: The clock signal provides the timing for synchronization between the processor and peripherals.
    
    \item \textbf{Interrupt Request (IREQ)}: A peripheral asserts this signal to request service from the processor. When IREQ goes high, the processor detects an interrupt request.
    
    \item \textbf{Interrupt Acknowledge (IACK)}: In response to IREQ, the processor sends an acknowledgment signal (IACK) to the peripheral. This signal indicates that the processor is ready to handle the interrupt.
    
    \item \textbf{Data Bus}: After the IACK signal is asserted, the peripheral places its device identifier on the data bus, allowing the processor to identify the source of the interrupt.
\end{itemize}

The interrupt cycle proceeds as follows:
\begin{enumerate}
    \item The peripheral raises the \textbf{IREQ} line to signal the interrupt request.
    \item The processor detects the interrupt and, after some clock cycles, responds by asserting the \textbf{IACK} line.
    \item The peripheral then places its \textbf{Device Identifier} on the data bus.
    \item The processor reads the device identifier to determine the source of the interrupt and proceeds with the appropriate interrupt service routine.
\end{enumerate}

This cycle ensures that the processor can handle asynchronous requests from peripheral devices in an organized and timely manner.
\newpage
\subsection{I/O Interrupt Priorities: Daisy Chain Arbitration}

Daisy Chain Arbitration is a basic method used to manage I/O interrupt priorities. The process operates as follows:

\begin{itemize}
    \item \textbf{Request Placement:} Any device can initiate a request to access the bus, indicated by signals such as \texttt{IREQ} (Interrupt Request).
    \item \textbf{Acknowledgment Line:} An acknowledgment signal, referred to as \texttt{IACK} or \texttt{Grant}, is sequentially passed from one device to the next.
    \item \textbf{Signal Interception:} The first device that requires access intercepts the acknowledgment signal, preventing it from being passed to devices further down the chain.
\end{itemize}

This method, while simple and easy to implement, has some limitations:
\begin{itemize}
    \item \textbf{Slow Performance:} Due to the sequential nature of signal passing, response times can be slower as the chain length increases.
    \item \textbf{Fixed Priorities:} Devices closer to the bus arbiter have higher priority by design, leading to a rigid priority structure.
\end{itemize}

\begin{center}
    \includegraphics[width=0.75\textwidth]{chapters/chapter2c/images/bus_arbiter.png}
\end{center}

In this setup, the bus arbiter, which acts as the processor or a proxy for the processor, grants access in a priority chain from the highest-priority device to the lowest. This method is suitable for systems where simplicity is valued over flexibility and speed.

\section{Direct Memory Access (DMA)}

Direct Memory Access (\textit{DMA}) is an efficient mechanism designed to offload the processor from managing repetitive and resource-intensive data transfers. Key considerations include:

\begin{itemize}
    \item \textbf{Interrupts Efficiency:} \textit{Interrupts} save the processor from continuously polling Input/Output (I/O) devices, allowing it to focus on computation.
    \item \textbf{Large Data Transfers:} Despite the use of interrupts, the processor may still spend considerable time transferring large chunks of data to and from high-throughput peripherals (e.g., disks, networks).
    \item \textbf{Solution:} A dedicated peripheral, known as the \textit{DMA Controller}, is introduced. This controller autonomously handles data transfers between memory and peripherals (read/write operations), freeing the processor to focus on more critical tasks.
\end{itemize}

\vspace{0.5cm}
\noindent
DMA significantly enhances system performance by reducing processor overhead during data transfer operations.

\vspace{0.8cm}
\begin{center} 
    \includegraphics[width=0.65\textwidth]{chapters/chapter2c/images/DMA.png}
\end{center}
\vspace{0.5cm}

The diagram above illustrates a key inefficiency: \textit{using the processor—a complex and expensive machine—to handle simple data transfer operations}. This inefficiency forms the basis for introducing DMA. 

\vspace{1cm}
\begin{center}
    \includegraphics[width=0.65\textwidth]{chapters/chapter2c/images/DMA2.png}
\end{center}
\vspace{0.5cm}

When initiating a data transfer, the \textbf{CPU} communicates with the \textbf{DMA Controller} to start the operation with a specific peripheral. The \textbf{DMA Controller} then handles communication with the \textbf{I/O device}, transferring the data to or from memory. 

\vspace{0.5cm}
\noindent
However, this introduces a new challenge. Previously, the \textit{CPU} was the sole \textbf{master of the BUS}. With the \textbf{DMA Controller} capable of two-way communication with the BUS (on its A input), it also becomes a master of the BUS. This requires a mechanism to manage BUS access between the processor and the DMA Controller.
\vspace{1cm}
\begin{center}
    \includegraphics[width=0.65\textwidth]{chapters/chapter2c/images/DMA3.png}
\end{center}
\vspace{0.5cm}
\noindent
This is achieved using a \textbf{Tri-State Buffer}. However, during the data transfer, the processor is temporarily \textit{disconnected from the BUS}, meaning it cannot track the progress of the transfer. Once the transfer is complete, an \textbf{interrupt} is required to notify the processor, ensuring it can resume operations with the updated data.
\newpage
\subsection{Timer and Interrupt Mechanism}

A \textbf{timer} is a critical hardware component used to manage periodic tasks in embedded systems. The timer operates by incrementing a \texttt{count} register until it reaches a programmable \texttt{max} value, at which point it generates an \textbf{interrupt request (IRQ)}. The key features of this mechanism are outlined below:

\begin{center}
    \includegraphics[width=0.65\textwidth]{chapters/chapter2c/images/timer.png}
\end{center}
\begin{itemize}
    \item \textbf{Programmable Frequency:} The \texttt{max} value is configurable, allowing the processor to adjust the interrupt frequency based on system requirements.
    \item \textbf{Interrupt Handling:} Upon reaching the \texttt{max} value, the timer sends an IRQ signal to the processor. This allows the processor to execute specific tasks at regular intervals.
    \item \textbf{System Integration:} The timer interacts with the processor, memory, and peripherals via the system bus, ensuring synchronized operation.
    \item \textbf{Task Management:} Without a timer, it would be impossible for a processor to manage multiple tasks simultaneously, as there would be no mechanism to divide time between different operations. The timer enables multitasking by providing precise time slicing for task scheduling.
\end{itemize}

This mechanism is essential for time-sensitive operations such as task scheduling, event triggering, and real-time control in embedded systems, enabling efficient multitasking and coordination between components.

 
\chapter{Part II(d) - Processor, I/Os, and Exceptions}

\section{Exceptions, Interrupts, Faults, Traps, and Checks}

\paragraph{Control Flow}
Under normal circumstances, the \textit{control flow}—the sequence of instructions executed by a program—is fully determined by the programmer. This includes the use of jumps, branches, and procedure calls.

\paragraph{Exceptions}
Exceptions represent a deviation from the normal control flow. They are triggered by \textbf{special conditions} that are not explicitly defined in the program. When an exception occurs, the control flow changes unexpectedly, and the program must respond accordingly.

\paragraph{Exception Handlers}
To manage exceptions, \textit{exception handlers} are invoked. These are specialized functions designed to take appropriate actions when an exception arises. An example of this is \textbf{I/O interrupts}, which signal specific events related to input/output operations.

\paragraph{Naming Conventions}
The terminology for exceptions and related events varies widely across systems. For clarity, we adopt the following convention based on RISC-V and the COD:
\begin{itemize}
    \item \textbf{Exceptions:} A general term encompassing all control flow deviations.
    \item \textbf{Interrupts:} A specific type of exception generated outside the processor.
\end{itemize}
Thus far, interrupts are the only form of exception encountered.

\subsection{Undefined Instruction}

Undefined instructions are instructions that the controller does not recognize, as they do not correspond to any valid operation in the Instruction Register (IR). These scenarios require special handling to ensure system stability and proper exception processing.

\vspace{0.5cm}
\begin{minipage}[htp]{0.35\textwidth}
- \textbf{Detection:} When an undefined instruction is detected in the IR, the controller generates a signal (\texttt{undef}) indicating the presence of an invalid operation. \\ 
- \textbf{Exception Handling:} The Program Counter (PC) is updated to the address of the Exception Handler to manage the undefined instruction. This involves: \\
\begin{itemize}
\item Saving the current PC for potential recovery.
\item Redirecting the control flow to the exception handler's address using multiplexer logic.
\end{itemize}
- \textbf{Control Logic:} The system leverages the Next PC Logic to determine whether the next instruction comes from the regular PC logic or the exception handler, based on the \texttt{undef} signal or an external interrupt (IRQ).
\end{minipage}
\hfill
\vline
\hfill
\begin{minipage}[htp]{0.55\textwidth}
    \begin{center}
        \includegraphics[width=1.2\textwidth]{chapters/chapter2d/images/undefined.png}
    \end{center}
\end{minipage} \\
\vspace{0.5cm}
- \textbf{Synchronous Nature:} These exceptions occur at a specific point in the program, precisely where the undefined instruction resides. This predictable behavior ensures that if the program is re-executed from the same initial state, the exception will occur at the exact same point, making debugging more straightforward. \\ \vspace{0.5cm}
- \textbf{Immediate Handling:} Serving the exception before executing the next instruction allows advanced features, such as efficient error recovery and the potential to extend system capabilities.
\vspace{0.5cm}
This mechanism ensures that undefined instructions do not disrupt the execution flow and are handled systematically, enabling robust error recovery and system stability.

\subsection{Optional \texttt{fadd.s} Instruction}

Suppose we want to include a floating-point addition instruction, denoted as:
\begin{assembly}
fadd.s rd, rs1, rs2
\end{assembly}

- Some processors might include a specialized ALU to support this instruction, whereas \textbf{cheaper processors do not}. \\ \vspace{7px}
- For processors that lack support for this instruction, its execution would trigger an \textit{undefined instruction exception}, which invokes a handler. \\ \vspace{7px}
- The handler can \textbf{emulate} the behavior of the \texttt{fadd.s} instruction, ensuring compatibility across processors. \\ \vspace{7px}

\subsection{Outline of an Undefined Instruction Handler}
To handle an undefined instruction, such as \texttt{fadd.s}, the following steps wouls be executed:
\begin{itemize}
    \item[] \textbf{Save all registers} on the stack that the handler or its callees might modify.
    \begin{itemize}
        \item Note: Standard calling conventions do not apply.
    \end{itemize}
    \item[] \textbf{Retrieve the problematic instruction}:
    \begin{itemize}
        \item If the program counter (PC) is saved, load the instruction from the corresponding address.
    \end{itemize}
    \item[] \textbf{Decode the instruction} in software and identify it as \texttt{fadd.s}.
    \item[] \textbf{Read the source registers} (operands) and either:
    \begin{itemize}
        \item Call a library function, or
        \item Implement the floating-point addition in software.
    \end{itemize}
    \item[] \textbf{Store the result} in the destination register.
    \item[] \textbf{Update the program counter (PC)} to point to the next instruction.
    \item[] \textbf{Jump to the updated PC} to resume execution.
\end{itemize}

\section{Exceptions and Interrupts}
Exceptions, interrupts, and related mechanisms handle critical events during execution. Key use cases include:
\begin{itemize}
    \item[] \textbf{I/O Requests:} Processing data or new inputs.
    \item[] \textbf{Timer Interrupts:} Handling time-based events.
    \item[] \textbf{Undefined Instructions:} E.g., unsupported floating-point operations.
    \item[] \textbf{Arithmetic Faults:} Errors like division by zero.
    \item[] \textbf{Memory Violations:} Unauthorized access to restricted memory.
    \item[] \textbf{Debugging:} Breakpoints and execution control.
    \item[] \textbf{Hardware Failures:} Malfunctions such as power loss.
\end{itemize}
\subsection{A Possible Classification of Exceptions}
\begin{center}
    \begin{tabular}{|l|l|l|l|}
    \hline
    \textbf{Type}                     & \textbf{Synchronous?} & \textbf{Coerced?}      & \textbf{Resume?} \\ \hline
    I/O request                       & Asynchronous          & Coerced               & Resume           \\ \hline
    Invoke OS                         & Synchronous           & User requested        & Resume           \\ \hline
    Trace instruction                 & Synchronous           & User requested        & Resume           \\ \hline
    Breakpoint                        & Synchronous           & User requested        & Resume           \\ \hline
    Page fault                        & Synchronous           & Coerced               & Resume           \\ \hline
    Misaligned access                 & Synchronous           & Coerced               & Resume           \\ \hline
    Memory protection violation       & Synchronous           & Coerced               & Terminate        \\ \hline
    Bus error                         & Synchronous           & Coerced               & Terminate        \\ \hline
    Arithmetic fault                  & Synchronous           & Coerced               & Terminate        \\ \hline
    Undefined instruction             & Synchronous           & Coerced               & Terminate        \\ \hline
    Hardware malfunction              & Asynchronous          & Coerced               & Terminate        \\ \hline
    Power failure                     & Asynchronous          & Coerced               & Terminate        \\ \hline
    \end{tabular}
\end{center}

\begin{itemize}
    \item \textbf{Synchronous?} Indicates whether the exception occurs as a direct result of the execution flow (synchronous) or independently of it (asynchronous).
    \item \textbf{Coerced?} Specifies whether the exception is forced by the system (coerced) or triggered by a user request.
    \item \textbf{Resume?} Denotes whether the system can continue executing after handling the exception (resume) or must terminate.
\end{itemize} 
\chapter{Part II(e) - Processor, I/Os, and Exceptions - Example W - 6.1}
\section{Part Ia: Connecitng an Input Peripheral}
Consider a hypothetical processor with the following buses and control signals:
\begin{itemize}
    \item[-] \textbf{A[31:0]}: Address bus
    \item[-] \textbf{D[31:0]}: Data bus
    \item[-] \textbf{AS} (Address Strobe): Active when a valid address is present on \textbf{A[31:0]}.
    \item[-] \textbf{WR} (Write): Active along with \textbf{AS} during a write cycle.
\end{itemize}

The input peripheral consists of 10 buttons numbered from 0 to 9, where:
\begin{itemize}
    \item[] Each button outputs a logic ‘1’ when pressed and ‘0’ otherwise.
    \begin{center}
        \includegraphics[width=0.10\textwidth]{chapters/chapter2e/images/button.png}
    \end{center}
    \item[] The processor reads the \textbf{state of the buttons} from memory location \texttt{0xFFFF'FFF0}:
    \begin{itemize}
        \item A value of ‘0’ indicates no button is pressed.
        \item A value of ‘1’ indicates at least one button is pressed.
    \end{itemize}
    \item[] The processor reads the \textbf{number of the button pressed} from memory location \texttt{0xFFFF'FFF4}.
\end{itemize}

\section{Bus Protocol}
 \begin{center}
     \includegraphics[width=0.45\textwidth]{chapters/chapter2e/images/bus.png}
 \end{center}
\section{Assembling the Circuit}
\textbf{Looking at the timing diagram is really really recommended when assembling a circuit.}
\begin{center}
    \includegraphics[width=0.65\textwidth]{chapters/chapter2e/images/circuit.png}
\end{center}

The input peripheral circuit connects 10 buttons to the processor, allowing it to detect button presses and identify which button is pressed. Below are the components of the circuit and their purposes:
\begin{itemize}
    \item[-] \textbf{Buttons:} Represent physical inputs numbered 0 to 9. Each button outputs a logic `1` when pressed and `0` otherwise.
    \item[-] \textbf{2\textsuperscript{n}-to-n Encoder:} Converts the 10 individual button signals into a 4-bit output, representing the button number.
    \item[-] \textbf{Address Decoders:} Determine the memory location being accessed (\texttt{0xFFFF'FFF0} or \texttt{0xFFFF'FFF4}) based on the address bus (\textbf{A[31:0]}).
    \item[-] \textbf{Latches (Q):} Store the state of the buttons and the number of the button pressed, enabling stable data retrieval by the processor.
    \item[-] \textbf{Control Signals (\textbf{AS}, \textbf{WR}):} 
    \begin{itemize}
        \item[-] \textbf{AS} (Address Strobe): Ensures the address on \textbf{A[31:0]} is valid.
        \item[-] \textbf{WR} (Write Enable): Activates during write cycles to store data.
    \end{itemize}
    \item[-] \textbf{Data Bus (\textbf{D[31:0]}):} Transfers data between the processor and the peripheral.
\end{itemize}

\section{Part 1b: Reading the Input Ports}
Write a RISC-V program named \texttt{buttons} to poll the state of the input buttons. The program must meet the following requirements:

\begin{itemize}
    \item Every time a button is pressed, the program should call the function \texttt{ShowIt}.
    \item Register \texttt{a0} must contain the ASCII code of the character corresponding to the button pressed. For example:
    \begin{itemize}
        \item Button ``0'' $\rightarrow$ ASCII code 48
        \item Button ``1'' $\rightarrow$ ASCII code 49
        \item Button ``2'' $\rightarrow$ ASCII code 50
    \end{itemize}
    \item The function \texttt{ShowIt} is provided, and you do not need to implement it.
\end{itemize}

\subsection{Software: buttons}
\begin{assembly}
    li s0, 0xFFFFFFF0
poll:    
    lw t0, 0(s0)
    beqz t0, 0(s0)
    lw t0, 4(s0)
    jal showIt
    j poll
\end{assembly}

\section{Part 2a - Connecting an Output Peripheral}
\begin{center}
    \includegraphics[width=0.15\textwidth]{chapters/chapter2e/images/seg8.png}
\end{center}
\begin{itemize}
    \item[] The peripheral receives an 8-bit signal, \texttt{SEG}, where each bit corresponds to a segment of the display:
    \begin{itemize}
        \item Bit 0 $\rightarrow$ Segment \texttt{a}
        \item Bit 1 $\rightarrow$ Segment \texttt{b}
        \item Bit 2 $\rightarrow$ Segment \texttt{c}, and so on.
    \end{itemize}
    A bit value of \texttt{1} indicates that the corresponding segment is lit.
    \item[] The processor writes a digit to the display by performing a write operation to the memory location \texttt{0xFFFF'FFF8}.
\end{itemize}
\section{Assembling everything}
\begin{center}
    \includegraphics[width=0.65\textwidth]{chapters/chapter2e/images/circuit2.png}
\end{center}

\section{Part 3a: Use Interrupts}
The processor includes three \textbf{Interrupt Priority Level} input pins, denoted as \texttt{IPL[2:0]}, which allow I/O devices to initiate interrupts. The interrupt handling mechanism functions as follows:

\begin{itemize}
    \item[] The binary value on the \texttt{IPL[2:0]} pins determines the \textbf{Priority Level} of the interrupt:
    \begin{itemize}
        \item[-] \texttt{0}: No interrupt request.
        \item[-] \texttt{1}: Lowest priority.
        \item[-] \texttt{7}: Highest priority.
    \end{itemize}
    \item[] An interrupt is \textbf{served only if} its priority level exceeds the current level stored in the processor's special status register.
    \item[] Interrupts at \textbf{Priority Level 7} are always served, as they are non-maskable.
    \item[] Modify the button interface to generate an interrupt with priority level 3 and identifier 0x45 when a button is pressed.
    \item[] The port at memory location 0xFFFF'FFF0 is no longer used.
\end{itemize}

\subsection{Interrupt Acknowledgement Process}

The interrupt acknowledgement process involves a coordinated sequence between the processor and peripheral devices. The key steps are outlined as follows:

\begin{center}
    \includegraphics[width=0.45\textwidth]{chapters/chapter2e/images/interrupt_ack.png}
\end{center}
\begin{itemize}
    \item[] The peripheral device asserts an \textbf{Interrupt Priority Level Request} on the \texttt{IPL[2:0]} lines, indicating the interrupt priority level.
    \item[] The processor recognizes the interrupt and initiates an \textbf{Interrupt Acknowledge (IACK)} signal to acknowledge the request.
    \item[] The processor evaluates the priority level of the interrupt:
    \begin{itemize}
        \item[-] If the priority level matches or exceeds the current threshold, the interrupt is serviced.
        \item[-] Otherwise, the interrupt is ignored or deferred.
    \end{itemize}
    \item[] Once the interrupt is acknowledged, the processor asserts the \textbf{Address Strobe (AS)} signal to identify the interrupting device.
    \item[] The peripheral responds by placing the \textbf{Device Identifier} on the \texttt{D[7:0]} data lines for the processor to process.
\end{itemize}
\newpage
\subsection{Solution}
\textit{Final circuit solution with interrupt handling mechanism:} \\
\vspace{10px}
\begin{minipage}[htp]{0.40\textwidth}
    \footnotesize
    \begin{itemize}
        \item \textbf{Retaining the Interrupt Request (IREQ):} 
        The \texttt{IREQ} signal is maintained active until the interrupt is served. This ensures that the request is not lost if the processor is handling a lower-priority task at the time of the interrupt.
    
        \item \textbf{Required Priority Level:} 
        The updated implementation includes a multiplexer that selects the required priority level for the interrupt. This allows the processor to dynamically assess whether the priority of the interrupt request is sufficient to preempt the current task.
    
        \item \textbf{Device Identifier:} 
        The device identifier (\texttt{0x45}) is explicitly encoded and transmitted over the data bus. This provides a clear and efficient way for the processor to identify the source of the interrupt.
    
        \item \textbf{Integration with the Processor:} 
        The processor interacts with the interrupt handling circuitry through the \texttt{IPL[2:0]}, \texttt{IACK}, and \texttt{AS} signals. These signals ensure that the interrupt servicing process aligns with the required priority levels and device identification.
    
        \item \textbf{New Comparator Mechanism:} 
        A comparator checks whether the least significant bits (LSBs) of the address match the expected value (\texttt{=3}). This additional check further validates the interrupt source and enhances system robustness.
    \end{itemize}
\end{minipage}
\hfill
\vline
\hfill
\begin{minipage}[htp]{0.55\textwidth}
    \begin{center}
        \includegraphics[width=0.85\textwidth]{chapters/chapter2e/images/circuit3.png}
    \end{center}
\end{minipage}

 
\chapter{Part III(a) - Processor, I/Os, and Exceptions An Example of I/Os and Exceptions W - 6.2} 
\chapter{Part III(a) - Memory Hierarchy - Virtual Memory - W.7.2}
\section{Segmentation Fault: Understanding the Cause}

Segmentation faults occur when a program attempts to access a memory location that is either invalid or restricted. Consider the following C code snippet:

\begin{center}
    \begin{tikzpicture}
        \node[rounded corners=30pt, draw=none] {\includegraphics[width=0.65\textwidth]{chapters/chapter3c/images/terminal.png}};
    \end{tikzpicture}
\end{center}

In this example:
\begin{itemize}
    \item The pointer \texttt{p} is assigned the value \texttt{1234}, which is an arbitrary and invalid memory address.
    \item When the program attempts to dereference \texttt{p} using \texttt{*p} to access the value at memory address \texttt{1234}, it triggers a segmentation fault because the program does not have permission to access this memory.
\end{itemize}

\textbf{Why This Happens:}
\begin{itemize}
    \item Modern operating systems enforce memory protection, disallowing access to memory that the program does not explicitly allocate or own.
    \item Hardcoding arbitrary addresses, like \texttt{1234}, is unsafe and violates these protections.
\end{itemize}

\textbf{Assembly Analysis:}
The following assembly instructions illustrate how the invalid memory access occurs:
\begin{itemize}
    \item \texttt{li t0, 1234} -- Load the immediate value \texttt{1234} into register \texttt{t0}.
    \item \texttt{lw a1, 0(t0)} -- Attempt to load a word from address \texttt{1234}.
    \item This results in a segmentation fault because address \texttt{1234} is not valid or accessible.
\end{itemize}

We need to be able to protect memory and prevent such invalid accesses. This is where memory protection mechanisms come into play.

\subsection{Overview - Problems to Solve}
Three main problems need to be addressed:

\begin{enumerate}
    \item \textbf{Memory Protection:} How can we protect memory so that each program (or process) running simultaneously in the system can only access its own data? How can processes be isolated from each other?
    \item \textbf{Insufficient Main Memory:} What happens if the main memory (DRAM) is not sufficient for the execution of a program? Can we utilize the disk to address this limitation? If so, how?
    \item \textbf{Running Multiple Programs:} How can we run several programs (processes) simultaneously? How can multiple programs be loaded into memory efficiently, and where should they be stored?
\end{enumerate}

\section{Relocation at Load Time}

Relocation at load time is a fundamental memory management process that adjusts memory addresses to align a program’s instructions with the actual memory layout during execution. This technique ensures that all address references within the program are accurate, allowing it to run seamlessly in its allocated memory space.

\textbf{Address Adjustments:} During the relocation process, address references within instructions are updated from placeholder values (e.g., \texttt{0x0000}) to their correct memory addresses (e.g., \texttt{0x1270} or \texttt{0x1248}). This adjustment applies to all memory references, including branches and jumps such as \texttt{beq} or \texttt{j}, which depend on accurate target addresses to function correctly.
\begin{center}
    \includegraphics[width=0.65\textwidth]{chapters/chapter3c/images/relocate.png}
\end{center}

By performing these updates, the program adapts to the memory layout without requiring additional runtime computations. This process, though effective, operates primarily at the binary level rather than at the assembly code level.

\subsubsection{Binary-Level Adjustments}

Relocation at load time involves modifying the binary code directly. Specific fields within the machine instructions are updated based on relocation tables, which specify the exact memory addresses to be adjusted. These tables play a crucial role in streamlining the relocation process and ensuring correctness.
\begin{center}
    \includegraphics[width=0.65\textwidth]{chapters/chapter3c/images/relocate2.png}
\end{center}

For instance, consider a binary program initially loaded at a base address of \texttt{0x0000}. During relocation, placeholders in the binary instructions are replaced with actual memory addresses derived from the relocation table, as illustrated in the figure. This guarantees that memory references resolve correctly during execution.

\subsubsection{Memory Utilization and Limitations}

While relocation at load time simplifies memory address management, it is not without its drawbacks. As programs are loaded and terminated, gaps in memory may form, leading to inefficient utilization. This fragmentation becomes particularly problematic in systems with limited memory or dynamic memory allocation needs.
\begin{center}
    \includegraphics[width=0.65\textwidth]{chapters/chapter3c/images/relocate3.png}
\end{center}

\textbf{Limitations:}
\begin{itemize}
    \item \textit{High Overhead:} Relocation requires considerable computational effort at load time to allocate and adjust memory segments accurately.
    \item \textit{Inflexibility:} Once memory is allocated, it cannot be dynamically reconfigured, making it challenging to adapt to changing program requirements.
    \item \textit{Fragmentation Constraints:} Memory fragmentation caused by terminated programs can prevent loading a new program if its size exceeds the largest available gap, even when total free memory is sufficient (garbage collector\dots).
\end{itemize}
\subsection{Relocation in Hardware: Base and Bounds MMU}
Memory relocation is an essential process in modern computer systems to map virtual addresses to physical addresses. The \textbf{Base and Bounds Memory Management Unit (MMU)} facilitates this process dynamically by ensuring secure and efficient address translation.
\begin{center}
    \includegraphics[width=0.65\textwidth]{chapters/chapter3c/images/MMU.png}
\end{center}
\begin{itemize}
    \item[-] \textbf{Base Register:} Holds the starting physical address of the process's memory.
    \item[-] \textbf{Bounds Register:} Defines the limit or size of the memory allocated to the process.
    \item[-] \textbf{Translation Process:}
    \begin{enumerate}
        \item The processor generates a \textit{virtual address}.
        \item The MMU checks if the virtual address exceeds the value in the \textit{Bounds Register}.
        \begin{itemize}
            \item If it exceeds, a \textbf{fault} is raised, preventing illegal memory access.
            \item Otherwise, the MMU adds the \textit{Base Register} value to the virtual address, producing the \textit{physical address}.
        \end{itemize}
        \item The \textit{physical address} is used to access the main memory.
    \end{enumerate}
\end{itemize}

This mechanism ensures process isolation and protects the system against unauthorized memory access, as only addresses within the defined bounds can be accessed. The dynamic nature of this relocation is key to supporting multitasking and efficient memory utilization.
\newpage
\subsection{Memory Management Unit (MMU)}
The \textbf{Memory Management Unit (MMU)} is a critical hardware component that facilitates the translation of \textit{virtual addresses} generated by the processor into \textit{physical addresses} used by the memory. This process is essential for efficient memory management in modern computer systems.

\begin{center}
    \includegraphics[width=0.65\textwidth]{chapters/chapter3c/images/MMU2.png}
\end{center}
\begin{itemize}
    \item[-] \textbf{Virtual Address:} Generated by the processor, representing a logical view of memory.
    \item[-] \textbf{Physical Address:} The actual address in the main memory where data is stored.
    \item[-] \textbf{Key Components:}
    \begin{enumerate}
        \item \textit{Configuration Registers:} Store information such as base addresses, bounds, and page tables.
        \item \textit{Translation Tables:} Facilitate the mapping of virtual to physical addresses.
    \end{enumerate}
    \item[-] \textbf{Process Overview:}
    \begin{enumerate}
        \item The processor generates a \textit{virtual address}.
        \item The MMU uses its configuration registers and translation tables to map the virtual address to a physical address.
        \item The mapped physical address is used to access data in the main memory.
    \end{enumerate}
\end{itemize}

\subsection{Program Relocation with Virtual Memory}
Virtual memory provides a powerful abstraction that decouples a program's logical memory view from the physical memory of the system. This flexibility is achieved by isolating programs from the underlying physical memory addresses, allowing dynamic relocation of programs without affecting their execution.
\begin{center}
    \includegraphics[width=0.65\textwidth]{chapters/chapter3c/images/virtual.png}
\end{center}
\begin{itemize}
    \item \textbf{Independent Address Spaces:} Each program is assigned its own virtual address space, creating the illusion that it has exclusive access to the entire memory. The program is unaware of its actual location in physical memory.
    \item \textbf{Program Relocation:} Since a program only interacts with its virtual address space, the operating system can freely move its physical location in memory as needed. This movement, known as \textit{relocation}, can occur during execution or at load time.
    \item \textbf{Benefits of Relocation:}
    \begin{itemize}
        \item \textit{Efficient Memory Use:} Programs can be compacted to free up contiguous physical memory for other processes.
        \item \textit{Load Balancing:} Active programs can be repositioned to optimize memory access speed or reduce fragmentation.
        \item \textit{Seamless Execution:} Since the memory translation is handled by the hardware (e.g., the Memory Management Unit, MMU), the program remains unaware of any changes in its physical location.
    \end{itemize}
\end{itemize}

\section{Relocation in Hardware: Base and Bounds MMU}
The Base and Bounds MMU (Memory Management Unit) is a hardware mechanism designed to facilitate memory relocation. It operates by performing two key actions on a virtual address provided by a process:
\begin{center}
    \includegraphics[width=0.65\textwidth]{chapters/chapter3c/images/relocation.png}
\end{center}
\begin{itemize}
    \item \textbf{Bounds Check:} The virtual address is compared with the \texttt{Bounds} register to ensure it is within the allowable range. If the virtual address exceeds the bounds, a fault is triggered, preventing unauthorized access.
    \item \textbf{Offset Addition:} If the virtual address is valid, it is added to the value in the \texttt{Base} register. This offset addition translates the virtual address into a physical address, which is then used to access the main memory.
\end{itemize}

\subsection{Preventing Overreach in Virtual and Physical Memory}
In systems using virtual memory, it is critical to ensure that programs remain within their allocated address spaces. If a program accesses memory beyond its assigned virtual boundaries, it risks overlapping with another program's memory. This can lead to severe security and stability issues.
\begin{center}
    \includegraphics[width=0.65\textwidth]{chapters/chapter3c/images/fault.png}
\end{center}
\begin{itemize}
    \item[-] \textbf{Virtual Memory Overreach:} Each program is given its own virtual address space. However, if a program attempts to access an address outside its bounds, it could inadvertently access data or instructions from another program. This can compromise the integrity of both programs.
    \item[-] \textbf{MMU Checks:} To prevent such overreach, the Memory Management Unit (MMU) enforces strict bounds checking. It ensures that all memory accesses fall within the allowed range defined by the program's base and bounds registers. If a memory access is outside these bounds, the MMU generates a fault, preventing the access.
    \item[-] \textbf{Physical Memory Isolation:} Even though programs use virtual addresses, these are translated to physical addresses by the MMU. Proper isolation ensures that physical memory regions assigned to different programs do not overlap, maintaining system stability.
\end{itemize}

This mechanism of bounds checking and fault generation safeguards against unintended interactions between programs and ensures that no program can overwrite or access another program's data.

\subsection{Base and Bounds MMU}

\begin{center} \includegraphics[width=0.65\textwidth]{chapters/chapter3c/images/bb.png} \end{center}

The Base and Bounds mechanism in the MMU defines the memory allocation for a process as follows:

\begin{itemize} \item Base: The starting address of the process's allocated memory. \item Bounds: The upper limit or size of the memory allocated to the process. \end{itemize}

\section{Needs of a Multiprogrammed System}

Multiprogrammed systems require efficient handling of memory and process management to ensure reliability and performance. The key requirements are as follows:

\begin{itemize}
    \item[-] \textbf{Relocation:} Programs must be written without prior knowledge of their location in memory. This ensures flexibility when allocating memory during execution.

    \item[-] \textbf{Protection:} Programs are restricted to access only their own data, safeguarding against interference from other programs. However, this protection mechanism can sometimes be crude, as it often limits each program to a single chunk of memory.

    \item[-] \textbf{Space Management:} When several programs run simultaneously, memory shortages may arise. Effective space allocation strategies are needed, which may involve techniques such as garbage collection and moving programs or data to optimize memory usage.
\end{itemize}

\section{Segmentation and Paging}
\begin{center}
    \includegraphics[width=0.65\textwidth]{chapters/chapter3c/images/segPage.png}
\end{center}
\paragraph{Segmentation:} Segmentation, an extension of the Base and Bounds technique, allows memory to be split exactly as needed by each program. Key characteristics include:
\begin{itemize}
    \item[-] Arbitrary starting point of a memory block.
    \item[-] Arbitrary length of memory blocks.
    \item[-] Multiple blocks can be allocated per application.
\end{itemize}

\paragraph{Paging:} Paging divides memory into equal-sized small blocks (e.g., 4–64~KiB) and assigns as many blocks as required to each program. This ensures uniformity in memory allocation.

\subsection{How do we Translate Now?}
Now we need to translate virtual addresses to physical addresses (with our new paging constraints). This process involves the following steps:
\begin{center}
    \includegraphics[width=0.65\textwidth]{chapters/chapter3c/images/translate.png}
\end{center}

For example,if the system needs to determine the physical memory location corresponding to the virtual address \texttt{0x2345} of \textbf{Program \#2}. The translation process follows these steps:

\begin{enumerate}
    \item Identify the \textbf{page number} and \textbf{offset} within the virtual address. Assuming page size is known, \texttt{0x2345} translates to a specific page and offset.
    \item Consult the \textbf{page table} for \textbf{Program \#2}, which maps virtual pages to physical pages. For instance:
    \[
    \text{\texttt{Program \#2 Page 0}} \rightarrow \text{\texttt{Physical Page 8}}, \quad
    \text{\texttt{Page 1}} \rightarrow \text{\texttt{Physical Page 9}}
    \]
    \item Use the mapping to locate the physical address. In this example, the virtual address resides in \texttt{Page 0}, which maps to \texttt{Physical Page 8}. Combining the physical page base address with the offset yields the final physical memory address.
\end{enumerate}

Sumarized, to find the physical address, we need to extract the virtual page number and the page offset from the virtual address. The virtual page number is used to look up the corresponding physical frame number in the page table. Finally, the physical address is computed using the formula:
\begin{center}
    \includegraphics[width=0.65\textwidth]{chapters/chapter3c/images/translate2.png}
\end{center}
\[
\text{Physical Address} = (\text{Physical Frame Number} \times \text{Page Size}) + \text{Page Offset}
\]

where the page offset is directly derived from the virtual address, and the physical frame number is obtained from the page table. The page size is often a power of 2 (for example, \(4 \, \text{KB} = 2^{12}\)), which makes extracting the page offset straightforward as it corresponds to the lower-order bits of the virtual address.
\newpage
\subsection{Virtual Adress Translation in a Paged MMU}
In a paged MMU, the virtual address generated by the processor is translated into a physical address using the page table stored in memory.
\begin{center}
    \includegraphics[width=0.65\textwidth]{chapters/chapter3c/images/pagedmmu.png}
\end{center}

\begin{itemize}
    \item[] \textbf{Page Table:}
    The page table, residing in main memory, contains:
    \begin{itemize}
        \item \textit{Control Bits:} Indicate the validity of a page and access permissions.
        \item \textit{Physical Page Numbers:} Map virtual pages to physical pages.
    \end{itemize}
\end{itemize}

\subsection{Memory Allocation is Easy Now}
Virtual memory systems simplify memory allocation by allowing noncontiguous physical memory to be mapped to contiguous virtual memory addresses. This enables efficient utilization of physical memory without requiring large, contiguous blocks.
\begin{center}
    \includegraphics[width=0.65\textwidth]{chapters/chapter3c/images/relocation_bis.png}
\end{center}
\begin{itemize}
    \item[] \textbf{Virtual Memory:} Each program operates in its own virtual address space, making it unaware of the physical memory layout. Virtual addresses are mapped to physical addresses using a page table.
    \item[] \textbf{Physical Memory:} Physical memory is divided into fixed-size blocks called \textit{pages}. Any empty page in physical memory can be allocated to a program's virtual page.
    \item[] \textbf{Advantages:}
    \begin{itemize}
        \item Programs can use noncontiguous memory regions without manual intervention.
        \item Memory fragmentation is minimized since any available physical page can be used.
        \item Programs are isolated from one another, enhancing security and stability.
    \end{itemize}
    \item[] In the diagram above, Program \#4's virtual memory is mapped to noncontiguous pages in physical memory (e.g., 0x2000, 0x8000, and 0xC000). This flexibility ensures efficient allocation.
\end{itemize}

The use of virtual memory significantly enhances system performance and simplifies memory management by abstracting physical memory complexities.

\subsection{Page Tables and Their Size}
Page tables in virtual memory systems can grow significantly in size, especially in cases with large memory spaces. For instance, a memory size of 64 GiB with 4 KiB pages requires $2^{24}$ entries, amounting to approximately 64 MiB of space.

\textbf{Challenges} \\
For programs that utilize only a few megabytes, the majority of these entries remain empty, leading to inefficient memory usage.

\textbf{Solutions} \\
Several approaches exist to mitigate this inefficiency, including:
\begin{itemize}
    \item \textbf{Hashed Tables:} An alternative structure to reduce unused entries.
    \item \textbf{Paged Segmentation:} A hybrid approach to manage memory.
    \item \textbf{Multilevel Page Tables:} A hierarchical design to handle sparse page tables efficiently.
\end{itemize}

\subsection{Multilevel Page Tables}

Multilevel page tables are a hierarchical solution to reduce memory overhead caused by storing a single large page table. This structure is particularly useful in virtual memory systems with large address spaces.

\begin{center}
    \includegraphics[width=0.65\textwidth]{chapters/chapter3c/images/multipage.png}
\end{center}
A virtual address is divided into three main parts:
\begin{itemize}
    \item \textbf{Virtual Page Number:} Determines the index within the page tables.
    \item \textbf{Page Offset:} Specifies the exact byte within the page.
\end{itemize}

The hierarchical organization involves:
\begin{enumerate}
    \item A \textbf{first-level page table}, indexed by the higher-order bits of the virtual page number. This table points to second-level page tables.
    \item \textbf{Second-level page tables}, indexed by the remaining bits of the virtual page number. These map to the physical page number.
\end{enumerate}

\textbf{Advantages:} \\
\begin{itemize}
    \item \textbf{Reduced memory usage:} Only necessary parts of the page tables are stored in memory.
    \item \textbf{Scalability:} Easily adapts to varying address space sizes.
\end{itemize}

\textbf{Key Steps in Address Translation:} \\
\begin{enumerate}
    \item Use the first-level index to locate the appropriate second-level page table.
    \item Use the second-level index to identify the physical page number.
    \item Combine the physical page number with the page offset to obtain the physical address.
\end{enumerate}

Multilevel page tables strike a balance between efficiency and memory overhead, making them a practical choice in modern operating systems.

 
\chapter{Comparch II - Part IV(a) - Instruction Level Parallelism Performance}
\textit{So far, we've only been building our processor, now it's about performance\dots}
\section{What is Performance ?}
\textit{Now what do we mean by performance ?, we need a metric to measure performance.}
\begin{itemize}
    \item[] Does processory frequency matter ? \textit{Is it better an Intel Core i7-7700K at 4.2 GHz or an AMD Ryzen 5 5600X at 3.7 GHz?}
    \item[] Memory Speed ? Cache efficiency ?
    \begin{itemize}
        \item[] Is it better to have 8 MiB of 4-way set-associative cache or 16 MiB of direct
        mapped cache?
        \item[] Is it better to have three levels of overall smaller caches or two levels of overall
        bigger caches?
    \end{itemize}
\end{itemize}
\subsection{Elapsed Time, CPU Time, \dots}
\textit{In reality, none of this matters in it self.}
What matters is the time it takes to perform a job a user needs.
\begin{center}
    \includegraphics[width=0.45\textwidth]{chapters/chapter4a/images/elapsed.png}
\end{center}

\begin{itemize}
    \item \textbf{Elapsed Time:} The total time taken for the job to complete, measured from start to finish. (e.g., 1.20 seconds)
    \item \textbf{System CPU Time:} The CPU time used by the operating system to execute instructions on behalf of the program. (e.g., 0.17 seconds)
    \item \textbf{User CPU Time:} The CPU time used to execute instructions for the program itself. (e.g., 0.79 seconds)
    \item 80.0\% of the Elapsed Time ($0.96\,s / 1.20\,s$) was spent on the job (the rest might be spent on system I/O, other jobs, and other users.)
    \item \textbf{Note:} \textit{User CPU Time} + \textit{System CPU Time} $\neq$ \textit{Elapsed Time}: The processor spent 0.96 seconds executing for the program, but the overall job took 1.20 seconds to complete.
\end{itemize}

\subsection{Relative Performance}
\textbf{Speedup}

The speedup metric quantifies how much faster system $X$ is compared to system $Y$. It is defined as:
\[
\text{Speedup} = \frac{\text{Performance}_X}{\text{Performance}_Y} = \frac{\text{Execution Time}_Y}{\text{Execution Time}_X}
\]

\textbf{Common Performance Indices}
Common benchmarks used to measure the speedups of systems relative to a standard system include: \newline
\textbf{SPEC CPU} (a classic CPU performance benchmark), \textbf{Geekbench} (a comprehensive cross-platform benchmark), \textbf{Cinebench} (a benchmark focusing on rendering performance), \textbf{LinPack HPL} (a high-performance computing benchmark), and \textbf{EEMBC (``Embassy'') CoreMark} (dedicated to benchmarking embedded processors).

\subsection{Relating Performance to Hardware Implementation}
In hardware design, time is measured by the \textit{clock period} or \textit{cycle}.

\subsubsection{Cycles per Instruction (CPI) and Instructions per Cycle (IPC)}
\begin{itemize}
    \item CPI: Average cycles needed per instruction
    \[
    \text{CPI} = \frac{\text{Total Cycles}}{\text{Total Instructions}} = \frac{\text{Execution Time}/\text{Clock Period}}{\text{Total Instructions}}
    \]
    \item IPC: Average instructions executed per cycle
    \[
    \text{IPC} = \frac{1}{\text{CPI}}
    \]
    \item Note: IPC $\leq$ 1 unless the processor can execute multiple instructions in parallel
\end{itemize}

\subsection{Improving Performance}
Performance is defined as the reciprocal of execution time:

\[
\text{Performance} = \frac{1}{\text{Execution Time}}
\]

By breaking down execution time, performance can be expressed as:

\[
\text{Performance} = \frac{f_{\text{clock}}}{\text{Instruction Count} \cdot \text{CPI}} = \frac{f_{\text{clock}}\cdot IPC}{\text{Instruction Count}}
\]

Where:
\begin{itemize}
    \item[] $f_{\text{clock}}$ is the clock frequency.
    \item[] $\text{CPI}$ is the cycles per instruction.
    \item[] $\text{Instruction Count}$ is the total number of instructions executed.
\end{itemize}

To improve performance, several strategies can be employed:

\begin{enumerate}
    \item \textbf{Increase Clock Frequency ($f_{\text{clock}}$):} Implement the processor using faster technology to achieve higher clock rates.
    \item \textbf{Reduce CPI:} Simplify instructions (RISC architecture) to lower the cycles per instruction. However, this may require more instructions to perform the same task.
    \item \textbf{Decrease Instruction Count:} Use fewer, more complex instructions (CISC architecture). This could increase the number of cycles per instruction or reduce the clock frequency.
    \item \textbf{Execute Instructions in Parallel:} Increase instructions per cycle (IPC) through techniques like pipelining or parallel execution.
\end{enumerate}
These trade-offs highlight the balance required when optimizing computer architecture for performance.

\subsection{Factors Influencing Performance}
Several factors can significantly impact system performance. Here are a few examples:

\begin{itemize}
    \item \textbf{Instruction Count and the Compiler:}
    \begin{itemize}
        \item The instruction count depends heavily on the compiler.
        \item A well-designed instruction set that the compiler can use effectively (i.e., best instructions for the job) is more important than having a highly reduced or overly complex instruction set. Otherwise, the instruction count may become unnecessarily large.
    \end{itemize}

    \item \textbf{Cycles Per Instruction (CPI) and Cache Performance:}
    \begin{itemize}
        \item CPI is influenced by the efficiency of the cache.
        \item As the overall code size increases, cache performance may degrade due to a higher number of cache misses.
    \end{itemize}

    \item \textbf{Clock Cycle Speed and Memory Access:}
    \begin{itemize}
        \item Faster clock cycles increase the demand for fetching instructions from memory.
        \item As a result, the performance of the cache becomes more critical in maintaining overall efficiency.
    \end{itemize}
\end{itemize}

\subsection{What to Improve to Increase Performance}

\textbf{Amdahl's Law (Law of Diminishing Returns)}:
The performance enhancement possible with a given improvement is limited by the amount the improved feature is used. \\

\textbf{Typical Software Situation:} \\
If a program spends 20\% of its time in subroutine $X$, the maximum reduction in execution time achievable by optimizing $X$ is 20\%. This corresponds to a speedup of:
\[
\text{Speedup} = \frac{1}{1 - 0.2} = 1.25
\]

\textbf{In a Processor:} \\
If an instruction $Y$ is used only 0.1\% of the time, is it worth optimizing it? It is more practical to focus on optimizing instructions or operations used more frequently, such as those taking 20\% of the time.

What often happens is that we start optimizing the most used instructions, but then we forget that once optimized, the less used instructions become the bottleneck. So, we need to start looking at the less used instructions and optimize them, and so on.
\begin{center}
\textbf{\textcolor{red}{Look for where most of the time goes!}}
\end{center}

\subsection{Benchmarks}

\textbf{Performance Indices:} \\
Benchmarks such as SPEC CPU, Geekbench, Cinebench, LinPack HPL, and EEMBC ("Embassy") CoreMark require a \textbf{precise definition of the user job(s)} to be executed. \\

\textbf{Benchmark Suites:} \\
Serious \textbf{benchmark suites} consist of collections of large and representative user programs, spanning various areas of typical use. These suites are often agreed upon by manufacturers to ensure standardization. \\

\textbf{Key Features:} \\
Benchmark suites do not only define the programs (e.g., written in C, C++, FORTRAN, or Java), but also specify: \\
\begin{itemize}
    \item How the programs should be compiled.
    \item What data should be used during execution.
    \item The conditions under which the programs should be run.
\end{itemize}

\subsubsection{SPEC CPU2006 Integer Benchmarks}
The SPEC CPU2006 benchmark suite evaluates the performance of computer processors by running a set of standardized workloads. Below is a comparison of integer benchmarks between SPEC CPU2000 and SPEC CPU2006. The benchmarks are categorized by their description, language, and reference time (RT).


\begin{center}
    \includegraphics[width=0.65\textwidth]{chapters/chapter4a/images/spec.png}
\end{center}

\textbf{Key Notes:}
\begin{itemize}
    \item \textbf{Reference Time (RT):} Measured on a Sun Ultra 5 with a 300MHz UltraSPARC III and 256KB L2 cache, corresponding to 100 SPEC2000.
    \item \textbf{Benchmark Complexity:} The runtime for integer benchmarks was 36.6 hours on a relatively old machine.
    \item SPEC CPU2006 introduced more complex workloads and higher reference times, demonstrating a significant evolution in benchmarking standards.
\end{itemize}
 
\chapter{Part Part IV(b) - Instruction Level Parallelism - Basic Pipelining}
\section{Circuit Timing and Performance}

Most of the time, we have discussed circuits at a higher level of timing abstraction, focusing on what happens during each cycle:
\begin{itemize}
    \item[-] \textbf{Finite State Machines:} $\texttt{state} \gets \texttt{next\_state}$
    \item[-] \textbf{Functional Units and Memory Elements:} Perform one operation over a small number of cycles, e.g., a combinational ALU performs an addition per cycle.
\end{itemize}

To design faster circuits, it is essential to delve deeper into the concepts of \textbf{signal propagation} and timing limitations.
\subsection{Signal Propagation}

In sequential circuits, the edges of the \texttt{clock} signal are pivotal for proper operation. They govern:

\begin{enumerate}
    \item \textbf{Data Capture}: Determining when \textbf{new data} is latched into the combinational logic.
    \item \textbf{Data Stability}: Ensuring that \textbf{processed data} (i.e., the previous input) has fully propagated through the combinational logic and is ready to be stored at the output.
\end{enumerate}

\begin{minipage}[htp]{0.45\textwidth}
    \begin{center}
        \includegraphics[width=0.45\textwidth]{chapters/chapter4b/images/prop_time.png}
    \end{center}
\end{minipage}
\hfill
\vline
\hfill
\begin{minipage}[htp]{0.45\textwidth}
    \begin{center}
        \includegraphics[width=0.65\textwidth]{chapters/chapter4b/images/prop_diag.png}
    \end{center}
\end{minipage}
To guarantee reliable operation, the \texttt{clock} period must be at least as long as the circuit's \textbf{critical path delay}—the longest delay through the combinational logic. This ensures that all signal transitions complete before the next clock edge arrives.

\[
\texttt{Clock Period} \geq \texttt{Critical Path Delay}
\]
\[
T_{\texttt{clock}} \geq T_{\texttt{critical\_path}}
\]


\noindent \textbf{Example:} In the circuit shown, the critical path is highlighted, indicating the longest combinational delay that dictates the minimum \texttt{clock} period.

\newpage

\subsubsection{Adding Intermediate Registers}
\textbf{Intermediate Registers} can be added to break up the critical path into smaller segments, reducing the overall delay. This technique is known as \textbf{pipelining}.
\begin{center}
    \includegraphics[width=0.65\textwidth]{chapters/chapter4b/images/prop_pipe.png}
\end{center}
\textit{Here for example, we've divided our overall critical path into three smaller segments such that, on the first clock edge, the first segment is processed, and on the second clock edge, the second segment is processed, and so on.}
Now, this new circuit has a shorter critical path, allowing for a faster clock period.
$$T_{\texttt{clock, pipe}} \geq T_{\texttt{new\_critical\_path}} \approx \frac{T_{\texttt{critical\_path}}}{3}$$
While this makes clock periods shorter, it also increases the number of clock cycles required to complete the operation.

\noindent \textbf{Conclusion} \\
The system's functionality remains unchanged, but the clock can run N times faster due to reduced critical path length from intermediate registers, at the cost of requiring N cycles to compute results. This allows for a finer control over the system.

\subsection{Pipelining: Enhancing System Throughput}

Pipelining is a technique widely used in computer architecture to improve the throughput of a system by overlapping the execution of multiple operations. It achieves this by dividing a task into smaller stages, where each stage performs a portion of the overall operation. These stages are connected in a pipeline structure, allowing multiple operations to be processed simultaneously.

\subsubsection*{How Pipelining Works}
A pipeline is divided into distinct stages, each designed to execute a specific part of the operation. For example, in an arithmetic operation, the stages might include fetching data, decoding instructions, performing calculations, and writing results. Each stage operates independently and processes data sequentially.

To understand this, consider a factory analogy where a product goes through three steps:
\begin{itemize}
    \item[] \textbf{Step 1:} Assembly
    \item[] \textbf{Step 2:} Painting
    \item[] \textbf{Step 3:} Packaging
\end{itemize}

In a \textbf{non-pipelined factory}, one worker completes all three steps for one product before starting the next. If each step takes 1 minute, three products would require \( 3 \times 3 = 9 \) minutes.

\hspace{-10px}
\begin{center}
    \includegraphics[width=0.65\textwidth]{chapters/chapter4b/images/parallel.png}
\end{center}
In a \textbf{pipelined factory}, the work is divided among three workers:
\begin{itemize}
    \item[] \textbf{At minute 1}, Worker A starts assembling the first product.
    \item[] \textbf{At minute 2}, Worker A starts assembling the second product, while Worker B paints the first.
    \item[] \textbf{At minute 3}, Worker A starts assembling the third product, Worker B paints the second, and Worker C packages the first.
\end{itemize}
By minute 5, all three products are completed, and the pipeline produces one product per minute after it is full.

This overlapping of tasks ensures that all workers are continuously busy, reducing the overall time required to produce multiple products.

\subsubsection*{Advantages of Pipelining}
The key benefits of pipelining include:
\begin{itemize}
    \item[-] \textbf{Improved Throughput:} By overlapping tasks, the system produces results at a faster rate. For instance, once the pipeline is full, one result can be produced per cycle.
    \item[-] \textbf{Efficient Resource Utilization:} Each stage works concurrently on different parts of separate operations, preventing idle resources.
    \item[-] \textbf{Scalability:} Pipelining can accommodate larger workloads by increasing the number of stages, enabling more operations to be processed simultaneously.
\end{itemize}

\subsection{Latency and Throughput}
\noindent \textbf{Latency} \\
Latency refers to the time between the start of a computation and when the result becomes available. It is given by:
\begin{itemize}
    \item \textbf{Original Circuit:} \( T \)
    \item \textbf{Pipelined Circuit:} \( \frac{T}{N} \times N = T \)
\end{itemize}

\noindent \textbf{Throughput} \\
Throughput represents the number of results produced per unit time. It is defined as:
\begin{itemize}
    \item \textbf{Original Circuit:} \( \frac{1}{T} = f \)
    \item \textbf{Pipelined Circuit:} \( \frac{1}{T/N} = \frac{N}{T} = N \times f \)
\end{itemize}

\subsection{Practical Pipelining: Latency and Throughput}
\subsubsection*{Stages and Timing in Pipelining}
Consider a pipeline with $N$ stages, where each stage $i$ takes a time $T_i$ to complete. The pipeline is divided by registers (denoted by red dashed lines), which ensure data is synchronized between stages.
\begin{center}
    \includegraphics[width=0.45\textwidth]{chapters/chapter4b/images/practical.png}
\end{center}
The overall operation of the pipeline is governed by:
\begin{itemize}
    \item[-] \textbf{Clock Period ($T_\text{CLK,pipe}$):} This is determined by the slowest stage, $T_\text{CLK,pipe} = \max(T_i + T_\text{FF})$, where $T_\text{FF}$ accounts for flip-flop delays.
    \item[-] \textbf{Stage Timing:} Ideally, $T_i \approx T_\text{CLK,comb}/N$, where $T_\text{CLK,comb}$ is the clock period of the original non-pipelined design.
\end{itemize}

\subsubsection*{Latency and Throughput of a Pipeline}

\begin{itemize}
    \item[-] \textbf{Latency ($\lambda_\text{pipe}$):} The latency is the total time required for a single input to propagate through all $N$ stages of the pipeline. It is given by:
    \[
    \lambda_\text{pipe} = N \cdot \max(T_i + T_\text{FF}) = N \cdot T_\text{CLK,pipe}
    \]
    While pipelining increases latency compared to a non-pipelined system, the trade-off is improved throughput.

    \item[-] \textbf{Throughput ($\phi_\text{pipe}$):} Throughput measures how many operations the pipeline can complete in a given time. Once the pipeline is filled, results are produced every clock cycle. It is calculated as:
    \[
    \phi_\text{pipe} = \frac{1}{\max(T_i + T_\text{FF})} = f_\text{pipe}
    \]
    where $f_\text{pipe}$ is the pipeline operating frequency.
\end{itemize}

Pipelining is a practical approach to achieving high-speed operation in digital systems, particularly in processors and signal processing applications. By carefully designing stage timing and managing trade-offs, pipelining can achieve an optimal balance between latency and throughput.
 
\chapter{Part 4c. Instruction Level Parallelism}
In the last chapter, we've seen how pipelining can make it easier to parallelize indepent operations making it the overall process faster.

\subsection{Pipelining the Processor}
Pipelining in processors is a technique that splits the execution of an instruction into multiple stages, each handled in parallel by separate hardware units. By doing so, multiple instructions can be processed simultaneously, thereby increasing the overall throughput of the processor without increasing the clock frequency.

\begin{itemize}
    \item \textbf{Fetch (F)}: Retrieve the instruction from memory (often from the instruction cache).
    \item \textbf{Decode (D)}: Interpret the fetched instruction, identify operands, and configure the control signals for execution.
    \item \textbf{Execute (E)}: Perform the required operations (e.g., arithmetic, logic, load, store).
\end{itemize}

\noindent In a basic pipeline with three stages (F, D, E), each stage takes one clock cycle. While one instruction is being executed, a second instruction can be decoded, and a third can be fetched at the same time. This overlapping of tasks leads to a substantial improvement in instruction throughput.

%\begin{figure}[htbp]
%    \centering
%    % Replace 'placeholder_image1.png' with your actual figure file
%    \includegraphics[width=0.7\textwidth]{placeholder_image1.png}
%    \caption{Conceptual view of a three-stage pipeline. Each instruction proceeds through Fetch, Decode, and Execute stages in order.}
%    \label{fig:basic-pipeline}
%\end{figure}

\subsubsection*{Example Pipeline Schedule}
Consider a schedule where three instructions (\(i\), \(i+1\), and \(i+2\)) enter the pipeline. Each instruction occupies a unique pipeline stage in any given clock cycle. Figure~\ref{fig:pipeline-schedule} illustrates how each instruction advances one stage every cycle:
\[
\begin{array}{c|cccccc}
\text{Time} & t & t+1 & t+2 & t+3 & t+4 & t+5 \\ \hline
i     & F & D & E & - & - & - \\
i+1   & - & F & D & E & - & - \\
i+2   & - & - & F & D & E & - \\
\end{array}
\]

%\begin{figure}[htbp]
%    \centering
%    % Replace 'placeholder_image2.png' with your actual figure file
%    \includegraphics[width=0.7\textwidth]{placeholder_image2.png}
%    \caption{A pipeline schedule showing instruction overlap in each stage.}
%    \label{fig:pipeline-schedule}
%\end{figure}

\subsubsection*{Multi-Cycle Processor vs.\ Pipelined Processor}
A \emph{multi-cycle} processor might use multiple cycles to execute every instruction (e.g., separate cycles for Fetch, Decode, ALU, Memory Access, and Write Back), but only one instruction flows through the processor at a time. In contrast, a \emph{pipelined} processor allows the next instruction to begin its Fetch stage in parallel with the Decode stage of the previous instruction, greatly improving throughput.
\begin{center}
    \includegraphics[width=0.65\textwidth]{chapters/chapter4c/images/vs.png}
\end{center}
\subsubsection*{Key Observations for Pipelining}
\begin{enumerate}
    \item \textbf{Repetitive Activity}: Pipelining is effective only when the processor has a large number of instructions to execute.
    \item \textbf{Subactivities}: Each major task (Fetch, Decode, Execute, etc.) should be clearly separable into sub-stages to allow parallel operation.
    \item \textbf{Throughput Gain}: Once the pipeline is full, an instruction completes at the end of every cycle (in the ideal case), increasing throughput.
\end{enumerate}

\noindent Properly designing pipeline stages and handling hazards (such as data, control, and structural hazards) ensures that the pipeline delivers high performance without correctness issues.

\section{Hardware Reuse Across Processor Stages}

In processor design, the approach to hardware reuse varies significantly between multicycle and pipelined architectures. Understanding these differences is crucial for optimizing performance and resource utilization.

\subsection{Multicycle Processor Architecture}

A multicycle processor divides instruction execution into distinct \textbf{states}, allowing certain hardware components to be shared across these states. This sharing is feasible because the components are not required simultaneously, enabling efficient resource utilization.

\begin{itemize}
    \item \textbf{FETCH} State: Typically involves an \textit{adder} to increment the program counter (PC).
    \item \textbf{EXECUTE} State: Requires an \textit{Arithmetic Logic Unit} (ALU) to perform operations.
\end{itemize}

Since the \textit{adder} and the \textit{ALU} are not active concurrently, the ALU can be repurposed to increment the PC during the FETCH state. This reuse reduces the overall hardware complexity and cost.

\subsection{Pipelined Processor Architecture}

In contrast, a pipelined processor operates with multiple \textbf{stages} that are active simultaneously. Each stage performs a different part of the instruction execution process, necessitating dedicated hardware for each stage to avoid conflicts and ensure seamless parallelism.

\begin{itemize}
    \item All pipeline stages are \textit{active concurrently}, handling different instructions in each stage.
    \item Hardware components cannot be shared across stages since multiple instructions require access to the same resources simultaneously.
    \item Consequently, hardware must be \textit{replicated} where necessary to maintain pipeline efficiency and prevent bottlenecks.
\end{itemize}

The inability to share hardware across pipeline stages often leads to increased hardware requirements compared to multicycle processors. However, this replication is essential for achieving high instruction throughput and maximizing pipeline performance.

\section{Two Main Challenges in Processor Design}

Designing efficient processors involves addressing several challenges. Two prominent issues are the \textbf{CISC vs. RISC} debate and \textbf{instruction independence}.

\subsection{CISC vs. RISC}

\begin{enumerate}
    \item \textbf{Pipeline Efficiency in CISC vs. RISC}
    \begin{itemize}
        \item \textbf{Question}: Can we construct a pipeline for a Complex Instruction Set Computer (CISC) that matches the efficiency of a pipeline designed for a Reduced Instruction Set Computer (RISC)?
        \item \textbf{Implications}:
        \begin{itemize}
            \item RISC architectures typically use simpler, fixed-length instructions, which are easier to pipeline efficiently.
            \item CISC architectures have more complex, variable-length instructions, potentially complicating pipeline design and reducing efficiency.
            \item The distinction influences processor complexity, performance, and power consumption.
        \end{itemize}
    \end{itemize}
\end{enumerate}

Understanding the trade-offs between CISC and RISC is essential for making informed decisions about processor design, balancing factors such as instruction complexity, pipeline efficiency, and overall performance.

\subsection{Instruction Independence}

\begin{enumerate}
    \setcounter{enumi}{1}
    \item \textbf{Ensuring Correct Execution with Dependent Instructions}
    \begin{itemize}
        \item \textbf{Issue}: Instructions are often \textit{dependent} on the results of preceding instructions, violating the assumption of \textbf{instruction independence}.
        \item \textbf{Challenge}: Executing code correctly in the presence of such dependencies requires sophisticated mechanisms to handle hazards, such as data forwarding or pipeline stalls.
        \item \textbf{Considerations}:
        \begin{itemize}
            \item Ensuring correct program behavior without sacrificing pipeline throughput.
            \item Balancing hardware complexity with the ability to maintain high instruction-level parallelism.
        \end{itemize}
    \end{itemize}
\end{enumerate}

Addressing instruction dependencies is critical for maintaining the integrity of program execution while striving for optimal pipeline performance. Techniques such as out-of-order execution and speculative execution are often employed to mitigate the impact of these dependencies.
 
\chapter{Part IV(d) - Instruction Level Parallelism - Scheduling}
\section{Dynamic Scheduling and Out-of-Order Execution}
\label{sec:dynsched}

Modern processors often employ \textbf{dynamic scheduling} to exploit more instruction-level parallelism
(ILP). Instead of strictly following the program’s original instruction order
(\emph{in-order} execution), the processor can:

\begin{itemize}
  \item \textbf{Fetch and decode} instructions as early as possible, even if previous
        instructions have not completed.
  \item \textbf{Reorder} instruction execution to avoid idle functional units when
        long-latency instructions (e.g., divides) are in progress.
  \item \textbf{Ensure correctness} by respecting true data dependencies and properly
        writing results back in program order (via a \emph{Reorder Buffer (ROB)}).
\end{itemize}

\subsection{Motivating Example}
Consider the following sequence of floating-point operations:
\begin{verbatim}
  divd  $f0, $f2, $f4    # Long-latency division
  addd  $f10, $f0, $f8   # Depends on divd's result
  subd  $f12, $f8, $f14  # Independent of divd's result
\end{verbatim}

\begin{itemize}
  \item In a strict in-order pipeline, the \texttt{subd} could be stalled until
        \texttt{divd} completes its execution (because \texttt{addd} is waiting
        on \$f0).
  \item With dynamic scheduling, the processor can reorder
        \texttt{subd} before \texttt{addd} as soon as it identifies that
        \texttt{subd} does \emph{not} depend on \texttt{divd}.
  \item This reordering utilizes available resources without violating correctness.
\end{itemize}

\subsection{Breaking the Rigidity of Basic Pipelines}
In a standard five-stage pipeline (Fetch, Decode, Execute, Memory, Writeback),
stalls are common when earlier instructions hold resources or have unresolved
data dependencies. Dynamic scheduling mitigates these stalls by:
\begin{enumerate}
  \item \textbf{Continuing to fetch and decode} new instructions (even if some are
        stalled in execution).
  \item \textbf{Deferring writeback} until resources become available or dependencies
        are resolved, using dedicated hardware structures (e.g., reservation
        stations, reorder buffers).
  \item \textbf{Allowing out-of-order completion}: instructions finish as soon as they
        can, but their results are committed in-order to preserve program semantics.
\end{enumerate}

\subsection{Dynamically Scheduled Processor Overview}
A typical dynamically scheduled CPU integrates:
\begin{itemize}
  \item \textbf{Fetch/Decode} units that feed instruction information into
        \emph{reservation stations} (RS).
  \item Multiple functional units (ALU, FPU, Memory pipelines) operating in
        parallel.
  \item A \textbf{Reorder Buffer (ROB)} to track instruction completion and to
        ensure in-order retirement (commit) of results.
  \item \textbf{Forwarding paths} to provide operands directly to waiting instructions
        without requiring all results to be written back to the register file first.
\end{itemize}

By decoupling instruction fetch/decode from their actual execution, a
\textbf{dynamically scheduled processor} allows more effective use of hardware
resources and improves overall performance, particularly for workloads with
long-latency operations or frequent stalls in in-order pipelines.
\subsection{Reservation Stations}

Reservation stations are hardware queues that temporarily hold instructions before they are sent to an execution unit. They are a key component in enabling \textbf{out-of-order execution} within a processor. This mechanism allows the CPU to execute instructions as soon as their necessary resources are available, rather than strictly adhering to the program's original order.

\subsubsection{How Reservation Stations Work}

Reservation stations facilitate several critical functions in the CPU:

\begin{enumerate}
    \item \textbf{Check Operand Availability:}
    They verify that all input operands required by an instruction have been computed and are available, either in the register file or through bypassing from another execution unit.

    \item \textbf{Prevent Structural Hazards:}
    They ensure that the targeted execution unit is free and ready to accept a new operation, thereby avoiding conflicts over shared functional units.

    \item \textbf{Enable Dynamic Scheduling:}
    They allow instructions to be dispatched to available execution units as soon as both their operands and the necessary functional resources are ready, rather than waiting for earlier instructions to complete.
\end{enumerate}

\subsubsection{Components of a Reservation Station}

Each entry within a reservation station typically contains the following elements:

\begin{itemize}
    \item \textit{Operation Code} (e.g., \texttt{add}, \texttt{sub}, \texttt{mul}): Specifies the operation to be performed.
    \item \textit{Operands or Tags}: Contains the actual operands needed for the operation or tags that reference future instructions responsible for producing those operands.
    \item \textit{Status Bits}: Indicate whether the operands are ready and whether an appropriate execution unit has been reserved.
\end{itemize}

\subsubsection{Execution Process}

\begin{enumerate}
    \item \textbf{Issuing Instructions:} When an instruction enters a reservation station, it waits until both its operands are available and the required execution unit is free.
    \item \textbf{Executing Instructions:} Once these conditions are met, the reservation station issues the instruction to the execution unit.
    \item \textbf{Broadcasting Results:} After execution, the result is forwarded to all reservation stations that are waiting for that particular value, updating their entries to reflect that the operand is now valid.
\end{enumerate}

\begin{center}
    \includegraphics[width=0.65\textwidth]{chapters/chapter4d/images/reservation_table.png}
\end{center}

\subsubsection{Analogy: Kitchen Order Management}

Think of reservation stations as a \textbf{kitchen's order management system} in a restaurant:

\begin{itemize}
    \item \textbf{Orders as Instructions:} Each customer's order is an instruction that needs to be prepared.
    \item \textbf{Ingredients as Operands:} The ingredients required for each dish represent the operands. An order can only be prepared if all necessary ingredients are available.
    \item \textbf{Chefs as Execution Units:} The chefs are the execution units that prepare the dishes.
\end{itemize}

\textbf{Process Flow:}
\begin{enumerate}
    \item \textbf{Taking Orders:} Orders are placed in the reservation station (waiting area) as they come in.
    \item \textbf{Checking Ingredients:} The system verifies that all ingredients for an order are available.
    \item \textbf{Assigning to Chefs:} If ingredients are ready and a chef is available, the order is handed off to the chef for preparation.
    \item \textbf{Updating Availability:} Once a dish is prepared, the result is available to fulfill other orders that might depend on it.
\end{enumerate}

\subsubsection{Summary}
Reservation stations decouple the dispatching of instructions from the availability of their operands and the execution units. By doing so, they enhance the processor's ability to execute instructions out of order, thereby improving overall performance and efficiency.

\subsection{Register Renaming and Data Dependencies}
Register renaming is a technique used to eliminate \textbf{Write-After-Write (WAW)} and \textbf{Write-After-Read (WAR)} dependencies, collectively referred to as \textit{name dependencies}. These dependencies occur because registers are reused across multiple instructions, despite the absence of actual data flow between them.

\subsubsection{Pipeline Hazards and Dependency Types}

Pipeline execution is prone to the following dependencies:
\begin{itemize}
    \item \textbf{Read-After-Write (RAW):} True data dependence, where an instruction requires the output of a previous one.
    \item \textbf{Write-After-Write (WAW):} Name dependence, resolved by renaming.
    \item \textbf{Write-After-Read (WAR):} Name dependence, resolved by renaming.
\end{itemize}

In dynamic pipelines, out-of-order execution can introduce hazards like WAW and WAR. Renaming ensures correctness by maintaining unique register identifiers, enabling both in-order and out-of-order pipelines to avoid conflicts.

\subsubsection{Example}

Consider the following instructions:\\
\begin{minipage}[t]{0.45\textwidth}
    \begin{verbatim}
        divd  $f0, $f1, $f2
        addd  $f3, $f0, $f4
        subd  $f4, $f5, $f6
        adddi $f0, $f5, 10
    \end{verbatim}
    - \texttt{addd} has a \textbf{RAW} dependence on \texttt{divd}. \\
    - \texttt{subd} has a \textbf{WAR} dependence on \texttt{addd}. \\
    - \texttt{adddi} has a \textbf{WAW} dependence on \texttt{divd}. \\
\end{minipage}
\hfill
\vline
\hfill
\begin{minipage}[t]{0.45\textwidth}
    By renaming, these dependencies are resolved:
    \begin{verbatim}
        divd  $f0, $f1, $f2
        addd  $f3, $f0, $f4
        subd  $f30, $f5, $f6
        adddi $f29, $f30, 10
    \end{verbatim}
\end{minipage} \\ \vspace{10px}
\textbf{This ensures the pipeline executes efficiently without conflicts, improving instruction throughput.}

\newpage
\section{Dynamically Scheduled Processor}
A \emph{dynamically scheduled processor} uses hardware mechanisms to exploit
\emph{out-of-order} execution, allowing instructions to proceed as soon as
their operands become available. This contrasts with a strictly pipelined MIPS
design, where all instructions flow in \emph{order} through the five pipeline
stages (F, D, E, M, W). By dynamically scheduling instructions, the
processor can reduce stalls and more effectively utilize hardware resources.

\begin{center}
    \includegraphics[width=0.65\textwidth]{chapters/chapter4d/images/complete.png}
\end{center}
\begin{itemize}
  \item[-] \textbf{Instruction Fetch \& Decode Unit:}
    Fetches and decodes instructions, dispatching them to the appropriate
    \emph{reservation stations} once the instruction type is identified.

  \item[-] \textbf{Reservation Stations:}
    Buffers that hold instructions waiting for the required operands or
    execution unit to become available. Each functional unit (e.g., ALU,
    floating-point, branch, or load/store) typically has its own set of
    reservation stations.

  \item[-] \textbf{ALU, FP Unit, Branch Unit, Load/Store Unit:}
    Execution units where instructions are actually carried out. The Load/Store
    Unit also manages memory operations. Because these units operate in parallel,
    multiple independent instructions can be serviced simultaneously.

  \item[-] \textbf{Register File:}
    Stores the architectural registers. Instructions read from and write to
    this file (potentially out of program order), but the final states are
    committed in order, preserving correct program semantics.

  \item[-] \textbf{Commit Unit:}
    Also referred to as the \emph{retirement} or \emph{write-back stage}. It
    ensures that the processor's \emph{architectural state} is updated in the
    correct program order, even though internal execution may be out of order.
\end{itemize}

\subsubsection*{Example Execution}
Suppose we have the following sequence of instructions:
\[
\begin{aligned}
&I1: \text{R1} \leftarrow \text{R2} + \text{R3}\\
&I2: \text{R4} \leftarrow \text{R1} \times \text{R5}\\
&I3: \text{R6} \leftarrow \text{R7} + \text{R8}
\end{aligned}
\]
In a strictly pipelined MIPS processor, \(\text{I2}\) would have to stall
while waiting for \(\text{R1}\) (produced by \(\text{I1}\)) to be written back.
However, a dynamically scheduled processor can place \(\text{I2}\) into a
reservation station, and simultaneously issue \(\text{I3}\) to the ALU,
because \(\text{I3}\) does not depend on \(\text{I1}\) or \(\text{I2}\).
This allows overlapping execution and reduced idle cycles.
\newpage
\subsection{Precise vs.\ Imprecise Exceptions}
Exceptions occur when the processor encounters an event requiring special handling (e.g., page fault, unsupported instruction).
They can be categorized as \emph{precise} or \emph{imprecise} based on whether the exact instruction that caused the exception---and the architectural state associated with that point in the instruction stream---can be precisely identified.

\begin{itemize}
  \item \textbf{Precise Exceptions}
  \begin{itemize}
    \item The processor enforces an in-order view of instruction completion at the point of the exception.
    \item This implies that all instructions before the faulting instruction have completed, and none of the subsequent instructions have begun or altered the architectural state.
    \item Reordering or out-of-order execution may still happen internally, but when an exception occurs, the processor ``commits'' instructions in a way that appears strictly in-order.
    \item This behavior simplifies error handling, as the operating system (OS) or exception handler knows exactly where the problem occurred and which instructions completed.
  \end{itemize}
  \item \textbf{Imprecise Exceptions}
  \begin{itemize}
    \item Out-of-order execution becomes visible to the user (or OS), meaning the faulting instruction might not be clearly identified at the time of the exception.
    \item The OS or programmer must assume that instructions have partially or fully executed in a different order than expected.
    \item Correcting the architectural state becomes more complex; the system may need to re-execute an entire subroutine to ensure correctness.
    \item Modern architectures generally avoid imprecise exceptions because of these complexities (especially for critical features such as virtual memory or I/O).
  \end{itemize}
\end{itemize}

\subsubsection{Out-of-Order Commitment and Exceptions}
Dynamic (out-of-order) execution complicates exception handling because the processor may complete some instructions after the faulting instruction if it issued them early. In a precise exception model, the hardware automatically rolls back or defers the effects of later instructions so that:
%
\begin{enumerate}
  \item Everything before the faulting instruction is guaranteed to have completed.
  \item No instructions after the faulting instruction have committed any state.
\end{enumerate}

This exact commitment model allows the exception handler to identify the precise location of the fault. When exceptions are imprecise, the program may need to be restarted from an earlier point to restore correct state, making it challenging for both the hardware and software to manage.

\newpage
\subsection{Reordering Instructions at Writeback}
A \emph{reorder buffer} (ROB) is used in out-of-order processors to maintain correct program order when writing back results to the architectural register file and memory. While instructions may execute in parallel or out of order, the ROB ensures that their visible effects (writes to registers/memory) occur in the original program order. This mechanism preserves the logical behavior of the program while also taking advantage of pipeline parallelism.
\begin{center}
    \includegraphics[width=0.65\textwidth]{chapters/chapter4d/images/rob.png}
\end{center}
\subsubsection{High-Level Overview.}
\begin{itemize}
  \item \textbf{Out-of-Order Execution:} After fetching and decoding, instructions are dispatched to execution units as soon as their operands become available. This enables the processor to exploit instruction-level parallelism.
  \item \textbf{Reorder Buffer (ROB):} Each fetched instruction is allocated an entry in the ROB. The entry holds:
    \begin{enumerate}
      \item The instruction’s \emph{program counter} (PC) and a unique \emph{tag}.
      \item The \emph{destination} (register or memory address).
      \item The \emph{result} value (once the execution unit produces it).
      \item An \emph{exception status} field to indicate whether an exception occurred.
    \end{enumerate}
  \item \textbf{Writeback \& Commit:} Although execution finishes out of order, the ROB enforces an \emph{in-order} commit. The oldest (head) entry in the ROB is checked first:
    \begin{enumerate}
      \item If its result is ready and no exception has occurred, the commit unit writes the result to the destination register or memory location.
      \item The ROB entry is then freed, and the \emph{head} pointer moves to the next instruction.
    \end{enumerate}
  \item \textbf{Preserving Program Correctness:} If an exception flag is set for the head entry, the pipeline can be flushed and the exception handled in program order, ensuring correct state recovery.
\end{itemize}

\subsubsection{Execution Steps.}
\begin{enumerate}
  \item \emph{Fetch \& Decode:} Instructions are fetched in program order and assigned entries in the ROB. Each ROB entry records necessary metadata (PC, destination, etc.).
  \item \emph{Dispatch to Execution Units:} As soon as sources for an instruction are ready, it can be sent to an available execution unit. Meanwhile, the ROB entry remains allocated to that instruction.
  \item \emph{Receive Results in ROB:} Once an execution unit finishes, it writes the result (along with the instruction’s tag) back to the ROB. The destination register or memory is \emph{not} updated yet.
  \item \emph{Commit in Program Order:} The reorder buffer’s head entry is checked:
    \begin{itemize}
      \item If the head instruction’s result is available and no exception is flagged, that value is \emph{committed} to the architectural register file or memory in correct program order.
      \item The head pointer is advanced, retiring the entry from the ROB.
      \item This process repeats as subsequent instructions at the head become ready and valid.
    \end{itemize}
\end{enumerate}

\subsubsection{Why It Improves Performance.}
Even though instructions are effectively \emph{reordered} before the final writeback, the pipeline overlaps multiple steps of different instructions. The reorder buffer allows:
\begin{itemize}
  \item[-] \textbf{Parallel Execution:} Independent instructions execute simultaneously in different pipeline stages, reducing overall latency.
  \item[-] \textbf{Hazard Resolution:} The ROB tracks which instructions have completed and can manage data hazards by forwarding results as soon as they are produced.
  \item[-] \textbf{In-Order Commit Guarantee:} The programmer-visible state updates in strict order (the ROB’s head-to-tail sequence), preserving correct semantics without stalling earlier instructions for later ones.
\end{itemize}

Thus, the reorder buffer provides the illusion of in-order execution while allowing out-of-order performance gains. As soon as an instruction completes, its result is available for subsequent instructions; however, to ensure correctness, final commitment of these results to the architectural state occurs strictly in the order of the original program.
\newpage
\section*{Dynamically Scheduled Processor: Step-by-Step Execution}
\vspace{-5px}
This outlines the operation of a dynamically scheduled processor, broken down into modular stages that can be adapted for various design choices (e.g., with or without forwarding paths, reservation stations, etc.). Each step is presented using a structured \textbf{if-then-else} format and emphasizes how instructions flow through the pipeline under different conditions. \\
\textit{Basically an algorithm to answer this chapter's exercises.}
\vspace{-15px}
{
  % Set spacing adjustments for lists
  \setlist{noitemsep, topsep=1pt}
  % Reduce font size locally
  \small
  \subsubsection*{1. Instruction Fetching}
  \vspace{-5px}
  \begin{enumerate}
      \item \textbf{Fetch Attempt}
      \begin{itemize}
          \item \textbf{IF} the instruction cache (I-cache) is ready to serve a new instruction \textbf{THEN}
          \begin{itemize}
              \item Fetch the next instruction address from the Program Counter (PC).
              \item Send the address request to the I-cache.
              \item Increment or update PC for the next instruction (or branch target if known).
          \end{itemize}
          \item \textbf{ELSE} (\emph{I-cache miss} or \emph{pipeline stall condition})
          \begin{itemize}
              \item Stall the fetch stage until the I-cache responds, or the pipeline is cleared.
          \end{itemize}
      \end{itemize}
      \item \textbf{Branch Misprediction Handling}
      \begin{itemize}
          \item \textbf{IF} a branch misprediction is detected \textbf{THEN}
          \begin{itemize}
              \item Flush the fetched instructions after the mispredicted branch.
              \item Update PC with the correct branch target.
              \item Re-fetch instructions from the correct location.
          \end{itemize}
          \item \textbf{ELSE}
          \begin{itemize}
              \item Continue normal fetching.
          \end{itemize}
      \end{itemize}
  \end{enumerate}
  \vspace{-15px}
  \subsubsection*{2. Instruction Decoding}
  \vspace{-5px}
  \begin{enumerate}
      \item \textbf{Decode Phase}
      \begin{itemize}
          \item \textbf{IF} the decode (or dispatch) hardware and any necessary pipeline registers are available \textbf{THEN}
          \begin{itemize}
              \item Read the fetched instruction.
              \item Decode the opcode and identify operands and destination register(s).
          \end{itemize}
          \item \textbf{ELSE}
          \begin{itemize}
              \item Stall decode until resources become free.
          \end{itemize}
      \end{itemize}
      \item \textbf{Hazard Checks}
      \begin{itemize}
          \item \textbf{IF} there is a structural hazard (e.g., decode hardware busy) \textbf{THEN}
          \begin{itemize}
              \item Stall the decode stage until the hazard is cleared.
          \end{itemize}
          \item \textbf{IF} there is a data hazard (register not yet available or pending in the Reorder Buffer) \textbf{THEN}
          \begin{itemize}
              \item Mark the instruction as needing operands from future writes or forwarding paths.
          \end{itemize}
          \item \textbf{ELSE}
          \begin{itemize}
              \item Proceed to place the instruction into an appropriate reservation station.
          \end{itemize}
      \end{itemize}
  \end{enumerate}
  \vspace{-15px}
  \subsubsection*{3. Reservation Stations}
  \vspace{-5px}
  \begin{enumerate}
      \item \textbf{Instruction Buffering and Dependency Resolution}
      \begin{itemize}
          \item \textbf{IF} a reservation station matching the instruction type (ALU, FP, Branch, or Load/Store) is free \textbf{THEN}
          \begin{itemize}
              \item Place the instruction in the station, along with operand tags or values.
              \item Check which operands are currently valid (available in the register file, or forwarded) and which are pending.
          \end{itemize}
          \item \textbf{ELSE}
          \begin{itemize}
              \item Stall the instruction until a reservation station becomes available.
          \end{itemize}
      \end{itemize}
      \item \textbf{Operand Availability}
      \begin{itemize}
          \item \textbf{IF} all input operands are ready \textbf{THEN}
          \begin{itemize}
              \item Dispatch the instruction to the appropriate execution unit immediately.
          \end{itemize}
          \item \textbf{ELSE}
          \begin{itemize}
              \item Wait for forwarding signals or for the Reorder Buffer (ROB) to broadcast the result.
          \end{itemize}
      \end{itemize}
  \end{enumerate}
  \newpage
  \subsubsection*{4. Execution Units}
  \vspace{-5px}
  \begin{enumerate}
      \item \textbf{Instruction Execution}
      \begin{itemize}
          \item \textbf{IF} the execution unit is free and all operands are valid \textbf{THEN}
          \begin{itemize}
              \item Execute the instruction (e.g., perform ALU operation, load/store, floating-point operation, or branch evaluation).
          \end{itemize}
          \item \textbf{ELSE}
          \begin{itemize}
              \item Stall in the reservation station until the execution unit is available and any missing operands are forwarded.
          \end{itemize}
      \end{itemize}
      \item \textbf{Forwarding and ROB Updates}
      \begin{itemize}
          \item \textbf{IF} the processor supports forwarding \textbf{THEN}
          \begin{itemize}
              \item Immediately broadcast the result on the Common Data Bus (CDB) so dependent instructions can receive the value without waiting for it to be written to the register file.
          \end{itemize}
          \item \textbf{ELSE}
          \begin{itemize}
              \item Write the result into the reorder buffer and/or register file.
              \item Dependent instructions must wait until the write is complete to read the result.
          \end{itemize}
      \end{itemize}
  \end{enumerate}
  \vspace{-15px}
  \subsubsection*{5. Commit Unit}
  \vspace{-5px}
  \begin{enumerate}
      \item \textbf{Reorder Buffer (ROB) and In-Order Commit}
      \begin{itemize}
          \item \textbf{IF} the instruction is at the head of the ROB and has completed execution with no exceptions \textbf{THEN}
          \begin{itemize}
              \item Commit the result to the architectural register file or memory (for store operations).
              \item Remove the instruction entry from the ROB.
          \end{itemize}
          \item \textbf{ELSE} (\emph{instruction not yet at head of ROB} or \emph{exception detected})
          \begin{itemize}
              \item Stall the commit stage until the head instruction is fully ready.
              \item \textbf{IF} an exception or misprediction is detected \textbf{THEN}
              \begin{itemize}
                  \item Flush instructions in the ROB after the faulting or mispredicted instruction.
                  \item Recover architectural state from the last known good state or from checkpoints.
              \end{itemize}
          \end{itemize}
      \end{itemize}
  \end{enumerate}
  \vspace{-15px}
  \subsubsection*{6. Register File}
  \vspace{-5px}
  \begin{enumerate}
      \item \textbf{Register Access and Dynamic Scheduling Support}
      \begin{itemize}
          \item \textbf{IF} the result is not yet committed (i.e., it resides in the ROB) \textbf{THEN}
          \begin{itemize}
              \item Dependent instructions obtain the result from forwarding paths or by listening to the ROB broadcast.
          \end{itemize}
          \item \textbf{ELSE}
          \begin{itemize}
              \item Read the committed value from the register file in the usual manner.
          \end{itemize}
      \end{itemize}
  \end{enumerate}
  \vspace{-15px}
  \subsubsection*{7. Execution Scenarios (Decision Tree)}
  \vspace{-5px}
  \noindent
  \textbf{Scenario 1: With Forwarding Paths}
  \begin{enumerate}
      \item \textbf{IF} an instruction completes in the execution unit
      \begin{itemize}
          \item \textbf{THEN} broadcast the result immediately to all reservation stations listening for that tag.
          \item Dependent instructions that had this operand pending can now proceed to execution in the next cycle (if their other operands are also ready and an execution unit is free).
      \end{itemize}
  \end{enumerate}
  \noindent
  \textbf{Scenario 2: Without Forwarding Paths}
  \begin{enumerate}
      \item \textbf{IF} an instruction completes in the execution unit
      \begin{itemize}
          \item \textbf{THEN} the result is first written to the reorder buffer (and eventually to the register file).
          \item Dependent instructions wait until the value is visible in the register file or the ROB can broadcast the commitment.
      \end{itemize}
  \end{enumerate}

  \textit{Lastly, general rule, if something is busy, or if it doesn't have enough space to add an instruction/operation etc\dots, Stall.}\vspace*{-10px}
  \subsubsection*{8. Performance Comparison}
  \vspace{-5px}
  \begin{itemize}
      \item Dynamic scheduling allows multiple instructions to be \textit{in-flight}, decoding and executing out of order as their operands become available.
      \item \textbf{Compared to a simple in-order pipeline}:
      \begin{itemize}
          \item More instructions can execute in parallel if they do not depend on each other.
          \item Hazards are resolved dynamically, leading to fewer pipeline stalls.
          \item Forwarding paths further reduce stalls by providing immediate data to dependent instructions.
      \end{itemize}
      \item Overall, the utilization of hardware resources is improved and throughput (instructions per cycle) is increased.
  \end{itemize}
} 
\chapter{Part 4f: Instruction Level Parallelism (ILP) Besides and Beyond Superscalars}

Instruction-Level Parallelism (ILP) is the measure of how many operations in a computer program can be performed simultaneously. The goal of ILP is to exploit parallel execution within a single thread to improve performance. Traditional \emph{superscalar} architectures achieve ILP by issuing multiple instructions in one clock cycle, but new challenges arise when hardware and program behavior limit the amount of parallelism that can be extracted.

\section{Superscalar Execution}
Modern high-performance processors often implement \emph{superscalar} execution, where multiple instructions are issued (i.e., started) in the same clock cycle to increase throughput. However, to realize sustained parallelism, several requirements must be met:

\begin{center}
    \includegraphics[width=0.45\textwidth]{chapters/chapter4f/images/superscalar.png}
\end{center}

\begin{itemize}
  \item \textbf{Fetch more instructions per cycle.}\\
  An instruction cache with sufficient bandwidth is needed to supply multiple instructions each cycle. Without adequate fetch capacity, the pipeline stalls and cannot exploit superscalar capabilities.

  \item \textbf{Commit more instructions per cycle.}\\
  To retire (commit) multiple instructions per cycle, the \emph{reorder buffer} (ROB) and the \emph{register file} must have enough ports and resources. A lack of commit bandwidth creates a bottleneck, undoing the benefits of parallel execution in earlier stages.

  \item \textbf{Obey data and control dependencies.}\\
  Even if hardware can fetch and commit multiple instructions per cycle, \emph{data hazards} and \emph{control hazards} must be respected. Modern superscalar designs typically use out-of-order (dynamic) scheduling to track dependencies and reorder instructions for higher throughput while maintaining correctness.
\end{itemize}

Despite advanced hardware techniques, data and control hazards impose fundamental limits on how much ILP can be extracted. In practice, balancing fetch, execute, and commit bandwidth with careful hazard management remains the central challenge of superscalar execution.

\section{Dealing with Control Hazards}
Superscalar processors can issue multiple instructions each cycle, but they are especially sensitive to branching instructions that disrupt the flow of fetched instructions. This section outlines several techniques that help mitigate the performance penalties of branches.

\subsection{Dynamic Branch Prediction}
\label{sec:branch-pred}
Branches create uncertainty: the processor needs to know which instruction addresses to fetch next. Two main problems arise when exploiting ILP:
\begin{itemize}
    \item \textbf{True data dependencies}, which are unavoidable since some instructions inherently depend on the results of others.
    \item \textbf{Branches}, which determine \emph{where} to look for further instructions.
\end{itemize}

\noindent
\textbf{Static Prediction}\\
Early branch prediction methods (e.g., always-taken, never-taken, or simple compiler hints) often fail to anticipate dynamic behavior accurately because they cannot adapt to runtime conditions.

\noindent
\textbf{Dynamic Prediction}\\
Dynamic branch prediction learns from past behavior. By using hardware structures that track how often a branch was taken, the predictor can adapt and refine its guesses over time, increasing overall accuracy and reducing wasted work from mispredicted branches.

\subsection{Branch History Table (BHT)}
A key component in many dynamic branch predictors is the \emph{Branch History Table (BHT)}, which uses part of the \textit{Program Counter} (PC) as an index. Each BHT entry stores one or more bits that represent the likelihood of a branch being taken or not taken.

\begin{center}
    \includegraphics[width=0.45\textwidth]{chapters/chapter4f/images/bht.png}
\end{center}

\begin{itemize}
    \item \textbf{Address Indexing:} The lower bits of the PC (e.g., bits 7:0) index into the BHT.
    \item \textbf{Prediction Storage:}
    \begin{itemize}
        \item \emph{One-bit Prediction:} Each entry stores a single bit:
              \begin{itemize}
                \item \texttt{0}: Not taken
                \item \texttt{1}: Taken
              \end{itemize}
        \item \emph{Two-bit Prediction:} Each entry stores two bits:
              \begin{itemize}
                \item \texttt{00}: Strongly not taken
                \item \texttt{01}: Weakly not taken
                \item \texttt{10}: Weakly taken
                \item \texttt{11}: Strongly taken
              \end{itemize}
    \end{itemize}
    \item \textbf{Update Mechanism:} Predictions are updated based on whether the branch was actually taken, helping the hardware adapt to program behavior.
\end{itemize}

\subsection{Speculative Execution}
Speculative execution is a performance optimization that issues and executes instructions \emph{before} a branch outcome is resolved. This reduces pipeline bubbles that occur when the processor must otherwise wait.

\begin{itemize}
  \item \textbf{Dynamic Branch Prediction:} The processor fetches and decodes along the predicted path, using minimal resources so that incorrect speculations are easily discarded.
  \item \textbf{Results Handling:} Computed results remain \emph{uncommitted} until the branch outcome is confirmed.
  \item \textbf{Recovery:} If a misprediction occurs, the processor \emph{squashes} any partially executed instructions from the wrong path and resumes from the correct path.
\end{itemize}

\subsection{Branches in the Reorder Buffer}
Modern out-of-order processors track branch instructions and their outcomes in a \emph{Reorder Buffer (ROB)}. Because branches are speculative, the ROB must handle both correct and incorrect predictions gracefully.

\begin{center}
    \includegraphics[width=0.45\textwidth]{chapters/chapter4f/images/rob.png}
\end{center}

\begin{itemize}
    \item \textbf{Correctly Predicted Branches:}
    When the actual branch target matches the prediction, the branch instruction is marked \emph{resolved} and can be retired normally.
    \item \textbf{Mispredicted Branches:}
    If the actual branch target differs from the prediction, all instructions following that branch in the ROB are invalidated (\emph{squashed}), and fetch restarts from the correct target address.
\end{itemize}

These mechanisms ensure the processor can continue running instructions \emph{speculatively}, reaping the benefits of ILP while safeguarding correctness.

\section{Beyond Superscalars: Simultaneous Multithreading (SMT)}
When a superscalar pipeline is unable to find sufficient parallelism within a single thread, an alternative is to keep the hardware busy by issuing instructions from multiple threads. \emph{Simultaneous Multithreading (SMT)} does exactly this, allowing multiple independent threads to utilize the same execution resources in parallel.

\begin{center}
    \includegraphics[width=0.45\textwidth]{chapters/chapter4f/images/smt.png}
\end{center}

\subsection*{SMT vs.\ Single-Thread Superscalar}
A dynamically scheduled superscalar typically manages only one thread at a time. It has:
\begin{itemize}
  \item A single set of \emph{reservation stations} to track operand availability.
  \item One \emph{reorder buffer} to handle out-of-order execution and in-order retirement.
\end{itemize}

\subsection*{Adding SMT Support}
An SMT processor can simultaneously issue instructions from multiple threads in the same cycle:
\begin{itemize}
  \item \textbf{Multiple PCs:} Each hardware thread needs its own program counter.
  \item \textbf{Extended Register File:} Either a separate set of registers per thread or a unified file with thread IDs.
  \item \textbf{Multiple or Extended ROBs:} Each thread must track and retire its instructions in order; thread IDs are often added to each entry.
\end{itemize}

\begin{center}
    \includegraphics[width=0.55\textwidth]{chapters/chapter4f/images/smt_proc.png}
\end{center}

Because thread instructions share the same functional units, SMT can achieve higher resource utilization, especially when one thread stalls or has limited ILP on its own. However, hardware becomes more complex, and caches and other shared structures must be carefully managed.

\section{Memory Considerations: Nonblocking Caches}
Even with a highly parallel core, performance may be bottlenecked by slow memory operations. If a load instruction misses in the cache, superscalar and SMT designs benefit from continuing other work rather than stalling immediately. \emph{Nonblocking caches} help achieve this.

\subsection*{Example of Memory Stall}
\begin{assembly}
lw   $t2, 0($t0)    # t2 = mem[t0]
lw   $t3, 0($t1)    # t3 = mem[t1]
addi $t3, $t3, 123
andi $t3, $t3, 0xff
\end{assembly}
If the first load (\texttt{lw $t2, 0($t0)}) results in a cache miss, a simple blocking cache would stall the pipeline until the data returns from memory. However, a \emph{nonblocking cache} lets subsequent instructions continue if they do not depend on the missing data.

\subsection*{Hit Under Miss and Miss Under Miss}
\begin{itemize}
  \item \textbf{Hit Under Miss:} Serves new cache requests from different addresses while a miss is being resolved.
  \item \textbf{Miss Under Miss:} Handles multiple outstanding misses, overlapping latencies from the memory system.
\end{itemize}

These mechanisms greatly improve throughput by allowing the processor to exploit ILP (and multiple threads in SMT) while memory requests are pending.

\newpage
\section{VLIW: Very Long Instruction Word Architecture}

VLIW (Very Long Instruction Word) architecture exploits \emph{instruction-level parallelism} (ILP) by bundling multiple operations into a single long instruction word. Unlike pipelined processors, which execute instructions sequentially (albeit overlapped in time), VLIW delegates the responsibility of scheduling parallel operations to the compiler. The compiler groups independent operations that can be executed simultaneously into fixed-width instruction bundles.
\begin{center}
    \includegraphics[width=0.65\textwidth]{chapters/chapter4f/images/VLIW.png}
\end{center}
\subsection{Core Concepts}
\begin{itemize}
    \item \textbf{Fixed Instruction Format:} Each instruction consists of multiple slots, where each slot is assigned to a specific functional unit (e.g., arithmetic logic unit (ALU), memory unit).
    \item \textbf{Static Scheduling:} The compiler analyzes dependencies in the code and schedules instructions into bundles. Dynamic dependency checks and scheduling hardware are not needed.
    \item \textbf{Compiler-Driven:} The compiler must ensure that operations within one bundle are independent (or safe) to execute in parallel.
\end{itemize}

\subsection{Example: VLIW vs. Pipelined Execution}

Consider the following simple code fragment in a C-like pseudocode:
\begin{verbatim}
a = b + c;   // Operation 1
d = e - f;   // Operation 2
g = h * i;   // Operation 3
x = arr[j];  // Operation 4 (memory load)
\end{verbatim}

\subsubsection*{Pipelined Processor Execution}
In a pipelined processor, these instructions may be overlapped in execution stages (fetch, decode, execute, etc.). However, they are still issued one after another. For example:
\begin{enumerate}
    \item \textbf{Cycle 1:} Fetch Operation 1
    \item \textbf{Cycle 2:} Decode Operation 1, Fetch Operation 2
    \item \textbf{Cycle 3:} Execute Operation 1, Decode Operation 2, Fetch Operation 3
    \item \textbf{Cycle 4:} Execute Operation 2, Decode Operation 3, Fetch Operation 4
    \item \textbf{Cycle 5:} Execute Operation 3, Execute Operation 4
\end{enumerate}
The pipeline overlaps different stages of separate instructions but does not execute multiple operations simultaneously in the same cycle.

\subsubsection*{VLIW Processor Execution}
Assume a simple VLIW processor with 3 functional units:
\begin{itemize}
    \item ALU1: Arithmetic operations.
    \item ALU2: Arithmetic operations.
    \item MEM: Memory load/store operations.
\end{itemize}

The VLIW compiler analyzes dependencies and groups independent operations into a single long instruction. Given the code, the compiler may schedule as follows:

\paragraph{VLIW Instruction Format:}
\[
\texttt{| ALU1 \quad|\quad ALU2 \quad|\quad MEM |}
\]

\paragraph{Scheduled VLIW Instructions:}
\begin{enumerate}
    \item \textbf{Cycle 1:}
    \begin{itemize}
        \item \texttt{ALU1:} \texttt{ADD r1, r2, r3} \quad (compute \texttt{a = b + c})
        \item \texttt{ALU2:} \texttt{SUB r4, r5, r6} \quad (compute \texttt{d = e - f})
        \item \texttt{MEM:} \texttt{LOAD r7, [arr + r8]} \quad (perform memory load for \texttt{x = arr[j]})
    \end{itemize}
    \item \textbf{Cycle 2:}
    \begin{itemize}
        \item \texttt{ALU1:} \texttt{MUL r9, r10, r11} \quad (compute \texttt{g = h * i})
        \item \texttt{ALU2:} \texttt{NOP} \quad (no operation)
        \item \texttt{MEM:} \texttt{NOP} \quad (no operation)
    \end{itemize}
\end{enumerate}

\subsubsection*{Key Differences}
\begin{itemize}
    \item \textbf{Instruction Issue:}
    \begin{itemize}
        \item \emph{Pipelined:} Issues one instruction per cycle and overlaps the stages.
        \item \emph{VLIW:} Issues a bundle of operations in one cycle, executing them concurrently.
    \end{itemize}
    \item \textbf{Scheduling:}
    \begin{itemize}
        \item \emph{Pipelined:} Hardware manages instruction overlapping.
        \item \emph{VLIW:} The compiler statically schedules independent instructions into a single, long instruction word.
    \end{itemize}
\end{itemize}

\subsection{Summary}
VLIW architectures shift complexity from hardware to the compiler, allowing multiple operations to execute in parallel within a single clock cycle. This enables efficient exploitation of ILP but requires careful compile-time analysis to ensure that parallel execution is both possible and correct.

 
\chapter{Part 5a. Multiprocessors Cache Coherence}
\textit{In the last chapter, we focused on how to get the most out of a single processor by exploring advanced parallelism techniques. Now, we’re moving to systems with multiple processors, where keeping their caches in sync is key to making everything work smoothly. This chapter introduces the basics of cache coherence and why it’s so important in shared-memory systems.}
\section{Flynn's Taxonomy (1966)}
Flynn's Taxonomy classifies computer architectures based on the number of instruction streams and data streams they support. It is divided into the following categories:
\begin{itemize}
    \item \textbf{SISD (Single Instruction, Single Data):} Represents uniprocessors where a single instruction stream operates on a single data stream. This is the traditional architecture of most early computers.

    \item \textbf{SIMD (Single Instruction, Multiple Data):} A single program executes on multiple data sets simultaneously. Classic examples include vector architectures used in high-performance computing, which are now less common. Modern x86 architectures support SIMD through various Instruction Set Architecture (ISA) extensions such as:
    \begin{itemize}
        \item MMX (1996)
        \item SSE (1999--2008)
        \item AVX (2011--2016)
    \end{itemize}

    \item \textbf{MIMD (Multiple Instruction, Multiple Data):} The general form of parallelism where each processor executes its own program on its own data. This is the most flexible and widely used parallel computing model.
\end{itemize}

\subsection{Shared-Memory Multiprocessors (UMA)}

Uniform Memory Access (UMA) is a shared-memory multiprocessor architecture where all processors have equal access time to the shared memory. This architecture is characterized by:
\begin{center}
    \includegraphics[width=0.65\textwidth]{chapters/chapter5a/images/smm.png}
\end{center}
\begin{itemize}
    \item \textbf{Uniform memory access:} Every path to memory is essentially the same, ensuring consistent performance across all processors.
    \item \textbf{Simple interconnect:} Typically, a bus or a basic interconnect is used to connect CPUs, caches, memory, and I/O devices.
    \item \textbf{Limited scalability:} UMA systems generally support 4 to 16 processors due to bottlenecks in the interconnect and memory access.
    \item \textbf{Traditional design:} UMA represents a simple and fairly traditional multiprocessor architecture suitable for small-scale parallel systems.
\end{itemize}

\subsection{Distributed-Memory Multiprocessors (NUMA)}
Nonuniform Memory Access (NUMA) is a distributed-memory multiprocessor architecture where each processor has its own local memory.
\begin{center}
    \includegraphics[width=0.65\textwidth]{chapters/chapter5a/images/dmm.png}
\end{center}
\begin{itemize}
    \item \textbf{Local memory access:} Each processor accesses its local memory much faster than the memory of other processors, leading to nonuniform memory access times.
    \item \textbf{Interconnection network:} Processors are connected through an interconnection network, which is often implemented as a real network, to enable communication and data exchange.
    \item \textbf{Scalability:} NUMA systems offer a more scalable and cost-effective way to grow the memory system, making them suitable for larger parallel systems.
    \item \textbf{Complex communication:} Communication between processors is more complex and involves higher latency compared to UMA systems.
\end{itemize}

\subsection{Programming Paradigms}
Parallel programming paradigms are classified based on how data is exchanged between processors. The two primary paradigms are:

\begin{itemize}
    \item \textbf{Shared-Memory:}
    \begin{itemize}
        \item Data is exchanged \emph{implicitly} through shared variables in a common memory space.
        \item Standard libraries (e.g., \texttt{OpenMP}) simplify programming.
        \item Well-suited for shared-memory architectures (e.g., SMP, NUMA).
        \item Can be implemented as \emph{Distributed Shared Memory (DSM)} on systems with physically distributed memory, using virtual memory abstractions (e.g., \texttt{TreadMarks} for DSM; \texttt{Apache Spark} for DSM-like abstractions in big data).
    \end{itemize}

    \item \textbf{Message Passing:}
    \begin{itemize}
        \item Data is exchanged \emph{explicitly} by sending and receiving messages over a network or interconnect.
        \item Standard libraries (e.g., \texttt{MPI}) are widely used.
        \item Natural for distributed-memory systems with private memory per processor.
        \item Can also be implemented on shared-memory systems (e.g., NUMA), although it may introduce unnecessary overhead compared to native shared-memory programming.
    \end{itemize}
\end{itemize}
\newpage
\subsection{Why (Hardware) Shared Memory?}

Shared memory provides a mechanism for parallel computing where multiple processors access a common memory space. Its advantages and disadvantages are as follows:

\begin{itemize}
    \item \textbf{Advantages:}
    \begin{itemize}
        \item Applications perceive it as a multitasking uniprocessor.
        \item Requires only evolutionary extensions for operating systems.
        \item Enables communication without relying on the operating system.
        \item Simplifies software development by allowing correctness to be prioritized over performance.
    \end{itemize}

    \item \textbf{Disadvantages:}
    \begin{itemize}
        \item Communication is implicit, making optimization more challenging.
        \item Synchronization between processors is complex.
        \item Places implementation demands on hardware designers.
    \end{itemize}

    \item \textbf{Result:}
    \begin{itemize}
        \item \emph{Symmetric Multiprocessors (SMPs):} Once the foundation of early supercomputers, SMPs have been largely replaced by distributed-memory message-passing systems due to scalability limitations.
        \item \emph{Chip Multiprocessors (CMPs) or Multicore Processors:} These dominate modern parallel computing, driving multibillion-dollar markets.
    \end{itemize}
\end{itemize}

\subsection{Cache Coherence and the Multi-Processor Problem}
\label{subsec:cache-coherence}

Cache coherence ensures that in a system with multiple processors,
all caches have a \emph{consistent} view of shared data. Without
coherence mechanisms, processors could read or write stale data in
their caches, leading to erroneous computation results.

\medskip

\noindent
\textbf{Example Scenario:}
\begin{center}
    \includegraphics[width=0.55\textwidth]{chapters/chapter5a/images/incoherence.png}
\end{center}
\begin{enumerate}
  \item \textbf{Processor P1 loads a value:} Suppose a shared variable
  \texttt{x} in main memory holds an initial value. Processor~P1 loads
  \texttt{x} into its cache, so it now has a local copy of \texttt{x}.
  \item \textbf{Processor P2 also accesses \texttt{x}:} Later, P2 tries
  to read \texttt{x} but experiences a cache miss (since \texttt{x} is
  not yet in P2's cache). It fetches \texttt{x} from main memory, storing
  this same value in its own local cache.
  \item \textbf{P2 modifies \texttt{x}:} Processor~P2 then updates
  \texttt{x} directly in its cache (for instance, \texttt{st x, r1}).
  Depending on the cache write policy (\emph{write-back} vs.\
  \emph{write-through}), P2 may or may not immediately update main memory.
  \textbf{However, even if it does write to memory, P1's cache copy remains
  \emph{unchanged}.}
  \item \textbf{P1 sees stale data:} Because P1 already has a cached
  copy of \texttt{x}, it continues reading that older value (i.e.,
  the one loaded earlier).
\end{enumerate}

\medskip

\noindent
\textbf{Why is this a problem?}
The issue here is that multiple cached copies of the same data
must be kept in sync. If updates are performed in one cache without
informing other caches, the system can quickly become \emph{incoherent}.
A processor might base calculations on an incorrect, outdated value of
\texttt{x}, leading to unpredictable behavior or incorrect program
outputs.

\medskip

\noindent
\textbf{Importance of Coherence Protocols:}
To prevent such inconsistencies, cache coherence protocols enforce
invalidation or updating of stale cache lines. When one processor
modifies \texttt{x}, coherence messages are sent over the interconnection
network so that other caches either invalidate their copies or update
them with the new data. This ensures that all processors always see
a consistent view of shared memory.

\subsection{Ensuring a Coherent Memory System}
A \emph{coherent memory system} guarantees that every processor observes
all shared-memory operations (reads and writes) in a manner that is
logically consistent with a single, shared view of memory. The goals of
such a system typically boil down to the following three properties:

\begin{enumerate}
  \item \textbf{Preservation of program order.} If a processor $P$
  writes to a location \texttt{X} and then (without any intervening
  writes from other processors) reads \texttt{X}, it must read back the
  value that it just wrote. This ensures that each processor's own
  writes are visible to itself in program order.

  \item \textbf{Coherent view (read values).} If processor $P_1$ writes
  \texttt{X}, and processor $P_2$ then reads \texttt{X}$\!$, with no
  other intervening writes to \texttt{X}, $P_2$ must see the value
  written by $P_1$. Essentially, a read in one processor cannot observe
  an older version of \texttt{X}$\!$ if a newer version exists in the
  system and there have been no conflicting writes.

  \item \textbf{Write serialization.} If multiple processors write to
  \texttt{X} (e.g., $P_1$ writes, then $P_2$ writes, etc.), all
  processors must observe these writes in the same order. Thus, if $P_1$
  writes to \texttt{X}$\!$ first and $P_2$ writes to \texttt{X}$\!$
  second, then no processor should be able to observe $P_2$'s write
  before $P_1$'s write.
\end{enumerate}

\medskip

\noindent
\textbf{How do we achieve coherence in practice?}

\begin{itemize}
  \item \textbf{Hardware Protocols:} Coherence is typically enforced via
  specialized hardware protocols (e.g., MESI, MOESI) that track and
  coordinate the states of cache lines. When a processor writes to
  \texttt{X}, the protocol ensures other copies of \texttt{X} are either
  invalidated or updated.

  \item \textbf{Snooping or Directory-Based Approaches:} In
  \emph{snooping} protocols, all caches monitor a shared bus to detect
  writes and invalidate outdated copies. In \emph{directory-based}
  protocols, a central directory keeps track of which caches hold each
  line, allowing precise invalidation or update messages.

  \item \textbf{Preserving Order:} A coherence protocol enforces the
  three properties above by establishing rules for when a cache line
  can be read, written, invalidated, or shared. This ensures every
  processor eventually sees writes in the correct order and never
  operates on stale data.
\end{itemize}

\noindent
By carefully orchestrating which cache copy is valid and who has the
authority to write to it, a coherent memory system prevents the classic
inconsistencies shown in our earlier example. Processors remain
synchronized on shared data values, avoiding stale reads and enabling
correct parallel execution.


\subsection{Snoopy Cache-Coherence Protocols}
Snoopy cache-coherence protocols rely on a shared bus to serialize memory transactions and ensure data consistency across multiple caches. Each cache controller \emph{snoops} all bus transactions and compares them against the cache lines it currently holds. \\

\noindent \textbf{Basic Operation:}
\begin{itemize}
  \item \textbf{Bus as a Serialization Point:} All memory requests issued by processors appear on the shared bus, providing a single global ordering.
  \item \textbf{Snooping:} Each cache controller listens (\emph{snoops}) to every bus transaction. If a transaction concerns a cache line that the controller contains, it takes steps to maintain coherence.
\end{itemize}

\noindent \textbf{Coherence Actions:}
When a transaction targets a cache line, the responsible cache controller can:
\begin{itemize}
  \item \textit{Invalidate} a stale copy of the line.
  \item \textit{Update} its local line if another cache provides new data.
  \item \textit{Supply value} to another cache or to memory.
\end{itemize}
The specific action taken depends on the protocol’s finite state machine (FSM), which tracks the line’s state (e.g., \textit{Modified}, \textit{Shared}, \textit{Invalid}, etc.). \\
\smallskip
\noindent \textbf{Simultaneous Controllers:} \\
Each cache operates its snooping logic independently but concurrently. Because they all observe the same bus traffic, conflicts and updates are detected quickly, preserving coherence across the system. \\

This bus-based \emph{snoopy} approach is conceptually simpler than directory-based methods and is effective for a moderate number of processors sharing a single bus. However, as system scale increases, the performance overhead of snooping and bus contention may become a limiting factor.


\subsection{Simple Invalidate Snooping Protocol}
The Simple Invalidate Snooping Protocol is a cache coherence protocol designed for write-through, write-no-allocate caches. It operates with two states: \textbf{Valid} and \textbf{Invalid}. \\
Transitions between these states are governed by processor and bus actions.

\begin{center}
    \includegraphics[width=0.3\textwidth]{chapters/chapter5a/images/basic.png}
\end{center}
\begin{itemize}
    \item[-] \textbf{Valid State}:
        \begin{itemize}
            \item A \texttt{PrWr} (Processor Write) results in a \texttt{BusWr} (Bus Write) operation.
            \item A \texttt{PrRd} (Processor Read) requires no bus action.
        \end{itemize}
    \item[-] \textbf{Invalid State}:
        \begin{itemize}
            \item A \texttt{PrRd} triggers a \texttt{BusRd} (Bus Read) to fetch data into the cache.
            \item A \texttt{PrWr} results in a \texttt{BusWr}.
        \end{itemize}
    \item[-] When another processor writes to the same cache line, a \texttt{BusWr} is broadcast, transitioning the state from \textbf{Valid} to \textbf{Invalid}.
    \item[-] A \texttt{BusRd} can transition the state from \textbf{Invalid} to \textbf{Valid}.
\end{itemize}

The protocol ensures coherence by invalidating or updating cache lines in response to snooped bus operations. This mechanism is essential in multiprocessor systems where caches are shared.

\textbf{Note:} The snooping mechanism actively monitors the bus to maintain coherence.

\subsection{3-State Write-Back Invalidation Protocol (MSI)}
The \textit{3-State Write-Back Invalidation Protocol (MSI)} is used to maintain cache coherence in multiprocessor systems. It introduces three states for cache lines:
\begin{itemize}
    \item \textbf{Modified (M):}
    \begin{itemize}
        \item The cache line contains the latest copy of the data.
        \item The memory is stale and not up-to-date.
        \item Only one cache can have this state for a given line.
    \end{itemize}

    \item \textbf{Shared (S):}
    \begin{itemize}
        \item The cache line contains a valid copy of the data.
        \item One or more caches may hold the same data in this state.
    \end{itemize}

    \item \textbf{Invalid (I):}
    \begin{itemize}
        \item The cache line is not valid and must be fetched from memory or another cache.
    \end{itemize}
\end{itemize}

\noindent\textbf{Features:}
\begin{itemize}
    \item Before entering the \textit{Modified} state, all other copies of the cache line must be invalidated.
    \item Ensures cache coherence by enforcing bus transactions to maintain order and perform invalidations.
\end{itemize}

\noindent\textbf{Comparison with 2-State Protocol:}
\begin{itemize}
    \item The 2-state protocol is simpler but less efficient, as every write operation requires a broadcast on the bus.
    \item MSI resolves coherence issues effectively, but it can lead to higher performance overhead due to bus transactions.
\end{itemize}

This protocol ensures coherence but can impact performance, particularly in systems with high contention for the memory bus.

\section{MSI Protocol}
%-----------------------------------------------------------------------
% Overview
%-----------------------------------------------------------------------
The \emph{Modified, Shared, Invalid} (MSI) protocol is a classic cache coherence protocol
used in multiprocessor systems with write-back caches. Its goal is to ensure that all
processors observe a consistent view of memory, even though multiple caches may hold
copies of the same memory block. In MSI, each cache block can be in exactly one of
three states at any time:

\begin{itemize}
  \item \textbf{M (Modified)}:
        The cache block holds the only valid (and most recent) copy of the data,
        and this copy has been modified with respect to main memory. Main memory
        is thus \emph{stale} until this block is written back.
  \item \textbf{S (Shared)}:
        One or more caches may contain valid copies of the data. The copy in main
        memory is also valid (i.e., the same as the cache blocks).
  \item \textbf{I (Invalid)}:
        The cache block is not valid. The data in this block must not be used
        without first fetching a valid copy from memory or another cache.
\end{itemize}

%-----------------------------------------------------------------------
% Diagram Placeholder
%-----------------------------------------------------------------------
\begin{center}
    \includegraphics[width=0.55\textwidth]{chapters/chapter5a/images/msi.png}
\end{center}

%-----------------------------------------------------------------------
% States and Transitions Explanation
%-----------------------------------------------------------------------
The protocol enforces coherence by requiring certain \emph{bus transactions} on
reads and writes, which can cause cache blocks to transition from one state
to another. Typical bus signals include:
\begin{itemize}
  \item \textbf{BusRd (Bus Read)}: A read request \emph{without} intent to modify.
  \item \textbf{BusRdX (Bus Read Exclusive)}: A read request \emph{with} intent to modify
        (also called \emph{Read For Ownership}); it invalidates any other copies.
  \item \textbf{BusWB (Bus Write Back)}: A cache writes its modified block back to memory
        (or supplies it to another cache) when it must give up ownership.
\end{itemize}

Below is a concise description of the main state transitions (processor actions are
prefixed with \texttt{Pr} and bus actions with \texttt{Bus}):

\begin{itemize}
  \item \textbf{I $\rightarrow$ S}:
    Occurs on a \texttt{PrRd}, which triggers \texttt{BusRd} if the block is not present
    in any cache (or must be fetched from memory). The cache obtains a shared copy.
  \item \textbf{I $\rightarrow$ M}:
    Happens on a \texttt{PrWr}, leading to \texttt{BusRdX}. All other caches invalidate
    their copies before one cache transitions to Modified.
  \item \textbf{S $\rightarrow$ M}:
    On a \texttt{PrWr} to a shared block, the cache issues \texttt{BusRdX}, invalidating
    other shared copies and gaining exclusive (modified) ownership.
  \item \textbf{M $\rightarrow$ S}:
    If another processor performs a read (\texttt{BusRd}) while the block is in M,
    the current cache must supply the data via \texttt{BusWB}, and the block transitions
    to S (now shared among caches).
  \item \textbf{M or S $\rightarrow$ I}:
    An \texttt{Invalidate} request (triggered by someone else's \texttt{BusRdX}) or a
    coherence miss can force the local copy to become Invalid.
\end{itemize}

By following these rules, the MSI protocol ensures that at most one cache
holds a \textbf{Modified} copy and that any other copies are either \textbf{Shared} or
\textbf{Invalid}. This guarantees coherence across all caches and maintains
the illusion of a single, consistent memory. 

\end{document}

